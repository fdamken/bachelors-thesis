\chapter{Invent some nice chapter title} \todo{Chapter title.}
	\section{Solution of the Harmonic Oscillator}
		\label{app:harmonicOscillatorSolution}
	
		To solve the differential motion equation
		\begin{align*}
			m\ddot{x} = -kx
		\end{align*}
		of the harmonic oscillator given in~\autoref{subsec:harmonicOscillator}, we use the solution approach
		\begin{align*}
			x(t) = c e^{\lambda t} \qquad \dot{x}(t) = \lambda c e^{\lambda t} \qquad \ddot{x}(t) = \lambda^2 c e^{\lambda t}
		\end{align*}
		and insert it into the differential equation:
		\begin{align*}
			m\ddot{x} = -kx \quad\implies\quad
			m \lambda^2 c e^{\lambda t} = -k c e^{\lambda t} \quad\iff\quad
			m \lambda^2 = -k \quad\iff\quad
			\lambda = \pm \sqrt{-\frac{k}{m}}
		\end{align*}
		As both \(k\) and \(m\) are defined to be positive, we get the complex solutions:
		\begin{align*}
			x_1(t) = e^{i t \sqrt{k / m}} \qquad x_2(t) = e^{-i t \sqrt{k / m}}
		\end{align*}
		Due to the superposition principle, also \( x_1 + x_2 \) and \( x_1 - x_2 \) are solutions. Hence, we get two real solutions by using Euler's identity \( e^{\varphi i} = \cos(\varphi) + i \sin(\varphi) \):
		\begin{align*}
			x_1 + x_2
				&= e^{i t \sqrt{k / m}} + e^{-i t \sqrt{k / m}} \\
				&= \cos\Big(t \sqrt{k / m}\Big) + i \sin\Big(t \sqrt{k / m}\Big) + \cos\Big(t \sqrt{k / m}\Big) - i \sin\Big(t \sqrt{k / m}\Big) \\
				&= 2 \cos\Big(t \sqrt{k / m}\Big) \\
			x_1 - x_2
				&= e^{i \sqrt{k / m} t} - e^{-i t \sqrt{k / m}} \\
				&= \cos\Big(t \sqrt{k / m}\Big) i \sin\Big(t \sqrt{k / m}\Big) - \cos\Big(t \sqrt{k / m}\Big) + i \sin\Big(t \sqrt{k / m}\Big) \\
				&=  2i \sin\Big(t \sqrt{k / m}\Big)
		\end{align*}
		This yields the following general solution:
		\begin{align*}
			x(t) = c_1 \cos\Big(t \sqrt{k / m}\Big) + c_2 \sin\Big(t \sqrt{k / m}\Big),\quad c_1, c_2 \in \C
		\end{align*}
		As both Sine and Cosine are Sinusoidal, different \( c_1 \neq c_2 \) only lead to a phase shift. Thus we can also write the solution as
		\begin{align*}
			x(t) = A \cos\Big(t \sqrt{k / m} + \varphi\Big)
		\end{align*}
		with the amplitude \(A\) and the phase \(\varphi\).
	% end
% end
