\begin{abstract}
	In control theory, we try to control a system to reach a specific goal, for example for a pendulum to stand upright. For linear dynamical systems, highly evolved control methods like the linear quadratic regulator exist. In the case of nonlinear systems, however, we do not have such control methods. This is the reason why linearization of a nonlinear system is quite appealing as it offers a way for us to apply linear control theory to nonlinear systems. Typical linearization approaches like small angle approximation have the disadvantage of only linearizing locally. Koopman theory offers a way to linearize a system globally by mapping the nonlinear behavior to an infinite-dimensional embedding using nonlinear observation functions. In this embedding, the dynamics behave linearly with an infinite-dimensional operator called the \emph{Koopman operator}, advancing the observations forward in time.

	Prior work has studied the Koopman operator and how to approximate it using a finite-dimensional linear embedding and to find the Koopman eigenfunctions. These functions only scale when applying the Koopman operator to it, spanning an invariant subspace of the embedding. Most of the prior work employed a deterministic view on the Koopman operator which makes gauging the uncertainty of the system impossible.

	We will build an algorithm based on the theory of \acl{lgds} by replacing the linear observations with nonlinear observations. This algorithm, the \emph{Koopman inference} algorithm, is an \acl{em} algorithm that alternates between estimating the linear latent state, and optimizing the latent state dynamics and the observations functions. This way we will be able to approximate the dynamics of nonlinear systems using a finite-dimensional linear embedding.

	We will see that our algorithm performs decently on common nonlinear environments like the pendulum, successfully finding an embedding where the dynamics behave linearly.

	% TODO: More results here?
\end{abstract}
