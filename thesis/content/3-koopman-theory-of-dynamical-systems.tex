\chapter{Koopman Theory of Dynamical Systems}
\label{c:koopmanTheoryOfDynamicalSystems}



Out first contribution is based on the need of a linearization technique that generalized globally. We have seen this need in the previous chapter when looking at simple nonlinear systems and small angle approximations. This brings us directly to Koopman theory, original introduced by B.~Koopman in 1931~\cite{koopmanHamiltonianSystemsTransformation1931} in the context of Hamiltonian systems and transformations in Hilbert spaces. Considering a nonlinear discrete-time dynamical system
\begin{equation*}
	\vec{s}_{t + 1} = \vec{F}(\vec{s}_t),\quad \vec{F} : \R^k \to \R^k
\end{equation*}
and observations \( \vec{g} : \R^p \to \R^p \) of this system, \ac{ie} \( \vec{g}_t \coloneqq \vec{g}(\vec{s}_t) \), the infinite-dimensional \emph{Koopman operator} \(\mathcal{K}\) advances all of the measurements forward in time. This relation can also be expressed as the composition
\begin{equation*}
	\mathcal{K} \vec{g} \coloneqq \vec{g} \circ \vec{F} \qquad\iff\qquad \mathcal{K} \vec{g}(\vec{s}_t) = \vec{g}\big( \vec{F}(\vec{s}_t) \big) = \vec{g}(\vec{s}_{t + 1})
\end{equation*}
This relation is true for every possible measurement function \( \vec{g} \) at any point of the system space \( \R^k \)~\cite{bruntonKoopmanInvariantSubspaces2016}. All possible measurement functions \( \vec{g} \) span an infinite-dimensional Hilbert space \( \mathcal{G} \). A finite set of measurements \( \vec{g}_1, \vec{g}_2, \cdots, \vec{g}_p \) that span an invariant subspace \( \mathcal{G}' \subset \mathcal{G} \), \ac{ie} applying the Koopman operator to a linear combination of these functions keeps them in the subspace
\begin{align*}
	\vec{g} &= \alpha_1 \vec{g}_1 + \alpha_2 \vec{g}_2 + \cdots + \alpha_p \vec{g}_p \\
	\mathcal{K} \vec{g} &= \beta_1 \vec{g}_1 + \beta_2 \vec{g}_2 + \cdots + \beta_p \vec{g}_p
\end{align*}
can be considered as a basis of that subspace. If that is possible, we can restrict the Koopman operator onto \( \mathcal{G}' \). Functions that both span the subspace \(\mathcal{G}'\) and do only scale when applying the Koopman operator, \ac{ie} \( \mathcal{K} \vec{\varphi} = \lambda \vec{\varphi} \) are called \emph{Eigenfunctions} of the Koopman operator. Finding these Eigenfunctions is extremely desirable, as it allows us to get a finite Koopman operator \( \mat{K} \) globally linearizing the dynamical system \( \vec{F} \).

For this thesis, we do not focus on finding the actual Eigenfunctions but to just approximate them.
