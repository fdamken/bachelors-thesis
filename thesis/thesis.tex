% !TeX spellcheck = en_US
% !TeX program = lualatex


% Do not surround the equal signs with spaces, it causes errors!
\documentclass[
	ruledheaders=section,
	class=report,
	thesis={type=bachelor},
	accentcolor=1c,
	custommargins=false,
	marginpar=false,
	BCOR=0mm, % Bindekorrektur
	parskip=half-,
	fontsize=11pt,
	instbox=false,
	twoside
]{tudapub}

% Include preamble to not clutter this file.
% Core packages.
% Primary: English, Secondary: German.
\usepackage[main = english, ngerman]{babel}
% Other packages.
%\usepackage{graphicx}
\usepackage{tikz}
\usepackage{mathtools}
\usepackage{amssymb}
\usepackage{siunitx}
\usepackage{tabularx}
\usepackage{xcolor}
\usepackage[font = normalsize, labelfont = bf]{caption}
\usepackage{subcaption}
\usepackage{float}
\usepackage{wrapfig}
\usepackage[ruled, vlined, linesnumbered]{algorithm2e}
\usepackage{csquotes}
\usepackage{microtype}
\usepackage{physics}
\usepackage{cancel}
\usepackage{hyperref}
\usepackage{tabto}
\usepackage{eqparbox}
\usepackage{pgfplots}
\usepackage{multicol}
\usepackage{listings}
\usepackage{xspace}
\usepackage[bottom]{footmisc}
\usepackage{excludeonly}

\usetikzlibrary{arrows.meta, shapes, backgrounds, angles, calc, chains, scopes, decorations.pathmorphing, patterns, positioning, quotes}

% Debug packages.
\usepackage{comment}

\makeatletter

% Make LOA a section.
\newcommand{\listalgorithmname}{List of Algorithms}
\renewcommand\listofalgorithms{
	\section*{\listalgorithmname}
	\@mkboth{\MakeUppercase\listalgorithmname}{\MakeUppercase\listalgorithmname}
	\@starttoc{loa}
}
% Make LOF a section.
\renewcommand\listoffigures{
	\section*{\listfigurename}
	\@mkboth{\MakeUppercase\listfigurename}{\MakeUppercase\listfigurename}
	\@starttoc{lof}
}
% Make LOL a section.
\newcommand{\listlistingname}{List of Listings}
\newcommand\listoflistings{
	\section*{\listlistingname}
	\@mkboth{\MakeUppercase\listfigurename}{\MakeUppercase\listfigurename}
	\@starttoc{lol}
}
% Make LOT a section.
\renewcommand\listoftables{
	\section*{\listtablename}
	\@mkboth{\MakeUppercase\listtablename}{\MakeUppercase\listtablename}
	\@starttoc{lot}
}

\makeatother



% Style definitions.

% Description-list styling.
\SetLabelAlign{parright}{\parbox[t]{\labelwidth}{\raggedleft#1}}
\setlist[description]{style = multiline, leftmargin = 4cm, align = parright}

\MakeOuterQuote{"}

% Simplify command to center a caption.
\newcommand{\captioncenter}{\captionsetup{justification=centering}}

% Matrix/Vector notation.
\renewcommand{\vec}[1]{\boldsymbol{\mathrm{#1}}}
\newcommand{\mat}[1]{\boldsymbol{\mathrm{#1}}}
% Shorthands.
\renewcommand{\C}{\mathbb{C}}
\newcommand{\R}{\mathbb{R}}
\newcommand{\E}{\mathbb{E}}
\newcommand{\normal}{\mathcal{N}}
\newcommand{\gaussianMulti}[4]{\frac{1}{\left(2\pi\right)^{#4/2} \cdot \lvert #3 \rvert^{1/2}} \exp \bigg\{\! -\frac{1}{2} \left(#1 - #2\right)^T #3^{-1} \left(#1 - #2\right) \bigg\}}
\newcommand{\logGaussianMulti}[4]{-\frac{1}{2} \log \lvert #3 \rvert - \frac{#4}{2} \log\left(2\pi\right) - \frac{1}{2} \left(#1 - #2\right)^T #3^{-1} \left(#1 - #2\right)}
\newcommand{\subgiven}{\vert}
\newcommand{\given}{\,\vert\,}
\newcommand{\biggiven}{\,\big\vert\,}
\newcommand{\Biggiven}{\,\Big\vert\,}
\newcommand{\bigggiven}{\,\bigg\vert\,}
\newcommand{\Bigggiven}{\,\Bigg\vert\,}
\newcommand{\new}{\mathrm{new}}
\newcommand{\KL}[2]{D_\mathrm{KL}\big( #1 \,\Vert\, #2 \big)}
\newcommand{\SRC}{\mathit{SRC}}
\newcommand{\rangedots}{\,\cdots\!}
\newcommand{\train}{\mathrm{train}}
% Math operators.
\DeclareMathOperator{\const}{const}
\DeclareMathOperator{\Cov}{Cov}
\DeclareMathOperator{\diag}{diag}

\newcommand{\oversetfootnotemark}[1]{\stepcounter{footnote} \overset{\mathclap{(\thefootnote)}}{#1}}
\let\realfootnote\footnote
\let\realfootnotetext\footnotetext
\renewcommand{\footnote}[1]{\realfootnote{\, #1}}
\renewcommand{\footnotetext}[1]{\realfootnotetext{\, #1}}
\newcommand{\doublefootnotetext}[2]{\addtocounter{footnote}{-1} \footnotetext{#1} \stepcounter{footnote} \footnotetext{#2}}
\newcommand{\triplefootnotetext}[3]{\addtocounter{footnote}{-2} \footnotetext{#1} \stepcounter{footnote} \footnotetext{#2} \stepcounter{footnote} \footnotetext{#3}}

\parindent0pt

% Definitions, Theorems and Lemmata.
\newtheorem{theorem}{Theorem}[chapter]
\newtheorem{definition}[theorem]{Definition}
\newtheorem{lemma}[theorem]{Lemma}

% Do not include subsections and lower in TOC.
\setcounter{tocdepth}{1}

% BibTeX.
\renewcommand{\bibname}{References}
%\bibliographystyle{ieeetr}
\bibliographystyle{alpha}

% TikZ-related stuff.
\tikzset{> = { Latex[length = 2mm] }}

% algorithm2e.
\SetCommentSty{text}

% Listings.
\colorlet{changedpurple}{TUDa-11a}
\colorlet{lerrorred}{TUDa-9b}
\colorlet{lstcomments}{TUDa-4c}
\colorlet{lstkeywords}{TUDa-9d}
\colorlet{lstlinenumbers}{TUDa-0c}
\colorlet{lststrings}{TUDa-2c}
\lstdefinestyle{base}{
	moredelim = **[is][\color{errorred}]{@@!}{@@@},
	moredelim = **[is][\color{changedpurple}]{@@?}{@@@}
}
\lstset{
	backgroundcolor   = \color{white},
	basicstyle        = \ttfamily\scriptsize\color{black},
	breakatwhitespace = true,
	breaklines        = true,
	breakautoindent   = true,
	captionpos        = b,
	commentstyle      = \color{lstcomments},
	escapeinside      = {{*@}{@*}},
	keywordstyle      = \color{lstkeywords},
	language          = Python,
	numbers           = left,
	numberstyle       = \tiny\color{lstlinenumbers},
	showstringspaces  = false,
	stringstyle       = \color{lststrings},
	style             = base,
	tabsize           = 4
}

% Ignore well-known acronyms in write-first-long rule.
% TODO: Replace the usage of \ac{..} with these everywhere!
\newcommand{\eg}{\acrshort{eg}\xspace}
\newcommand{\ie}{\acrshort{ie}\xspace}
\newcommand{\wrt}{\acrshort{wrt}\xspace}

% Other.
\newcommand{\appautorefname}{appendix}



% Glossary styles.
\newcolumntype{L}[1]{>{\raggedright\let\newline\\\arraybackslash\hspace{0pt}}m{#1}}
\newcolumntype{C}[1]{>{\centering\let\newline\\\arraybackslash\hspace{0pt}}m{#1}}
\newcolumntype{R}[1]{>{\raggedleft\let\newline\\\arraybackslash\hspace{0pt}}m{#1}}
\newglossarystyle{iasThesisGeneral}{
	\glossarystyle{super3colheader}%
	\renewenvironment{theglossary}%
		{\begin{longtable}{L{0.15\textwidth}L{0.8\textwidth}R{0\textwidth}}}%
		{\end{longtable}}%
	\renewcommand*{\glossaryheader}{\textbf{\entryname} & \textbf{\descriptionname} & \\}
	\renewcommand*{\glossaryentryfield}[5]{\glsentryitem{##1}\glstarget{##1}{##2} & ##3 \\}
}
\newglossarystyle{iasThesisOperators}{
	\glossarystyle{super3colheader}
	\renewenvironment{theglossary}
		{\begin{longtable}{L{0.15\textwidth}L{0.55\textwidth}L{0.25\textwidth}}}%
		{\end{longtable}}%
	\renewcommand*{\glossaryheader}{\textbf{\entryname} & \textbf{\descriptionname} & \textbf{Operator} \\}
	\renewcommand*{\glossaryentryfield}[5]{\glsentryitem{##1}\glstarget{##1}{##2} & ##3 & ##4 \\}
}

% Acronyms.
\newacronym{iid}{i.i.d.}{independently and identically distributed}
\newacronym{lgds}{LGDS}{Linear Gaussian Dynamical System}
\newacronym{wrt}{w.r.t.}{with respect to}
\newacronym{rts}{RTS}{Rauch-Tung-Striebel}
\newacronym{ie}{i.e.}{that is}
\newacronym{eg}{e.g.}{for example}
%\newacronym[\glslongpluralkey={Markov Decision Processes}]{MDP}{MDP}{Markov Decision Process}
%\newacronym[\glsshortpluralkey={KL}]{KL}{KL}{Kullback-Leibler divergence}

% Operators.
\newglossary[opg]{operator}{opi}{opo}{List of Operators}
\newglossaryentry{expectation}{
	sort        = { expectation value },
	name        = { \ensuremath{\E} },
	symbol      = { \ensuremath{\E}[\cdot] },
	description = { the expectation value },
	type        = operator
}
\newglossaryentry{covariance}{
	sort        = { covariance },
	name        = { \ensuremath{\Cov} },
	symbol      = { \ensuremath{\Cov}[\cdot] },
	description = { the covariance },
	type        = operator
}
\newglossaryentry{ln}{
	sort        = { natural logarithm },
	name        = { \ensuremath{\ln} },
	symbol      = { \ensuremath{\ln(\cdot)} },
	description = { the natural logarithm },
	type        = operator
}
\newglossaryentry{trace}{
	sort        = { linalg trace },
	name        = { \ensuremath{\tr} },
	symbol      = { \ensuremath{\tr(\cdot)} },
	description = { the trace of a matrix },
	type        = operator
}

% TODO: Other:
%  - Gaussian distribution

% Symbols.
\newglossary[syg]{symbol}{sys}{syo}{List of Symbols}
\newglossaryentry{theta}{
  sort        = { parameters theta },
  name        = { \ensuremath{\vec{\theta}} },
  description = { vector of parameters from a probability distribution },
  type        = symbol
}


% Create the gloassaries.
\makeglossaries

% Tikz-diagrams for not cluttering the text.

\newcommand{\tikzHarmonicOscillator}{
	\begin{tikzpicture}
		\node [draw, rectangle] (m) at (0, 3) {\(m\)};
		\draw [thick] (-1.5, 5) -- (1.5, 5);
		\fill [pattern = north east lines] (-1.5, 5) rectangle (1.5, 5.2);
		\draw [decoration = { aspect = 0.3, segment length = 2mm, amplitude = 3mm, coil }, decorate] (0, 5) -- node[left, xshift = -0.4cm]{\(k\)} (m);

		\coordinate (xA) at (1, 4.5);
		\coordinate (xB) at (1, 3.5);
		\draw [->] (xA) -- node[right]{\(x\)} (xB);
	\end{tikzpicture}
}

\newcommand{\tikzSimplePendulum}{
	\begin{tikzpicture}
		\node [draw, circle, fill, minimum width = 0.5cm] (C) at (0, 0) {};
		\node [draw, circle] (mass) at (120:3cm) {\(m\)};
		\coordinate (A) at (90:3cm);
		\draw (C) -- node[left]{\(L\)} (mass);
		\draw [dashed] (C) -- (A);
		\draw pic [draw, "$\varphi$", angle radius = 1.5cm] {angle=A--C--mass};

		\draw [<-] (0.5, 1) -- node[right]{\(g\)} (0.5, 2);
	\end{tikzpicture}
}

\newcommand{\tikzKoopmanOperator}{
	\begin{tikzpicture}[->, state/.style = { draw, circle, minimum width = 1cm, minimum height = 1cm }]
		\node [state]                  (y1) {\raisebox{-5pt}{\(\vec{y}_{\,\mathclap{\,1}}\)}};
		\node [state, right = 1 of y1] (y2) {\raisebox{-5pt}{\(\vec{y}_{\,\mathclap{\,2}}\)}};
		\node [state, right = 1 of y2] (y3) {\raisebox{-5pt}{\(\vec{y}_{\,\mathclap{\,3}}\)}};
		\node [minimum width = 1cm, minimum height = 1cm, right = 1 of y3] (yD) {\(\cdots\)};
		\node [state, right = 1 of yD] (yT) {\raisebox{-5pt}{\(\vec{y}_{\,\mathclap{\,T}}\)}};

		\node [state, below = 1 of y1] (x1) {\raisebox{-5pt}{\(\vec{x}_{\,\mathclap{\,1}}\)}};
		\node [state, below = 1 of y2] (x2) {\raisebox{-5pt}{\(\vec{x}_{\,\mathclap{\,2}}\)}};
		\node [state, below = 1 of y3] (x3) {\raisebox{-5pt}{\(\vec{x}_{\,\mathclap{\,3}}\)}};
		\node [minimum width = 1cm, minimum height = 1cm, below = 1 of yD] (xD) {\(\cdots\)};
		\node [state, below = 1 of yT] (xT) {\raisebox{-5pt}{\(\vec{x}_{\,\mathclap{\,T}}\)}};

		\draw (y1) -- node[above]{\(\mathcal{K}\)} (y2);
		\draw (y2) -- node[above]{\(\mathcal{K}\)} (y3);
		\draw (y3) -- node[above]{\(\mathcal{K}\)} (yD);
		\draw (yD) -- node[above]{\(\mathcal{K}\)} (yT);

		\draw (x1) -- node[below]{\(\vec{F}\)} (x2);
		\draw (x2) -- node[below]{\(\vec{F}\)} (x3);
		\draw (x3) -- node[below]{\(\vec{F}\)} (xD);
		\draw (xD) -- node[below]{\(\vec{F}\)} (xT);

		\draw (y1) to[bend right = 15] node[left]{\(\vec{g}\)} (x1);
		\draw (y2) to[bend right = 15] node[left]{\(\vec{g}\)} (x2);
		\draw (y3) to[bend right = 15] node[left]{\(\vec{g}\)} (x3);
		\draw (yT) to[bend right = 15] node[left]{\(\vec{g}\)} (xT);

		\draw [dashed] (x1) to[bend right = 15] node[right]{\(\vec{g}^{-1}\)} (y1);
		\draw [dashed] (x2) to[bend right = 15] node[right]{\(\vec{g}^{-1}\)} (y2);
		\draw [dashed] (x3) to[bend right = 15] node[right]{\(\vec{g}^{-1}\)} (y3);
		\draw [dashed] (xT) to[bend right = 15] node[right]{\(\vec{g}^{-1}\)} (yT);
	\end{tikzpicture}
}

\newcommand{\tikzHiddenMarkovModel}{
	\begin{tikzpicture}[->, state/.style = { draw, circle, minimum width = 1cm, minimum height = 1cm }]
		\node [state]                  (s1) {\raisebox{-5pt}{\(s_{\,\mathclap{\,1}}\)}};
		\node [state, right = 1 of s1] (s2) {\raisebox{-5pt}{\(s_{\,\mathclap{\,2}}\)}};
		\node [state, right = 1 of s2] (s3) {\raisebox{-5pt}{\(s_{\,\mathclap{\,3}}\)}};
		\node [minimum width = 1cm, minimum height = 1cm, right = 1 of s3] (sD) {\(\cdots\)};
		\node [state, right = 1 of sD] (sT) {\raisebox{-5pt}{\(s_{\,\mathclap{\,T}}\)}};

		\node [state, below = 1 of s1] (y1) {\raisebox{-5pt}{\(\vec{y}_{\,\mathclap{\,1}}\)}};
		\node [state, below = 1 of s2] (y2) {\raisebox{-5pt}{\(\vec{y}_{\,\mathclap{\,2}}\)}};
		\node [state, below = 1 of s3] (y3) {\raisebox{-5pt}{\(\vec{y}_{\,\mathclap{\,3}}\)}};
		\node [state, below = 1 of sT] (yT) {\raisebox{-5pt}{\(\vec{y}_{\,\mathclap{\,T}}\)}};

		\draw (s1) -- (s2);
		\draw (s2) -- (s3);
		\draw (s3) -- (sD);
		\draw (sD) -- (sT);

		\draw (s1) -- (y1);
		\draw (s2) -- (y2);
		\draw (s3) -- (y3);
		\draw (sT) -- (yT);
	\end{tikzpicture}
}

\newcommand{\tikzLinearGaussianDynamicalSystem}{
	\begin{tikzpicture}[->, state/.style = { draw, circle, minimum width = 1cm, minimum height = 1cm }]
		\node [state]                  (x1) {\raisebox{-5pt}{\(\vec{x}_{\,\mathclap{\,1}}\)}};
		\node [state, right = 1 of x1] (x2) {\raisebox{-5pt}{\(\vec{x}_{\,\mathclap{\,2}}\)}};
		\node [state, right = 1 of x2] (x3) {\raisebox{-5pt}{\(\vec{x}_{\,\mathclap{\,3}}\)}};
		\node [minimum width = 1cm, minimum height = 1cm, right = 1 of x3] (xD) {\(\cdots\)};
		\node [state, right = 1 of xD] (xT) {\raisebox{-5pt}{\(\vec{x}_{\,\mathclap{\,T}}\)}};

		\node [state, below = 1 of x1] (y1) {\raisebox{-5pt}{\(\vec{y}_{\,\mathclap{\,1}}\)}};
		\node [state, below = 1 of x2] (y2) {\raisebox{-5pt}{\(\vec{y}_{\,\mathclap{\,2}}\)}};
		\node [state, below = 1 of x3] (y3) {\raisebox{-5pt}{\(\vec{y}_{\,\mathclap{\,3}}\)}};
		\node [state, below = 1 of xT] (yT) {\raisebox{-5pt}{\(\vec{y}_{\,\mathclap{\,T}}\)}};

		\draw (x1) -- node[above]{\(\mat{A}\)} (x2);
		\draw (x2) -- node[above]{\(\mat{A}\)} (x3);
		\draw (x3) -- node[above]{\(\mat{A}\)} (xD);
		\draw (xD) -- node[above]{\(\mat{A}\)} (xT);

		\draw (x1) -- node[left]{\(\mat{C}\)} (y1);
		\draw (x2) -- node[left]{\(\mat{C}\)} (y2);
		\draw (x3) -- node[left]{\(\mat{C}\)} (y3);
		\draw (xT) -- node[left]{\(\mat{C}\)} (yT);
	\end{tikzpicture}
}


\newcommand{\algname}{Nonlinear Gaussian Koopman}



\begin{document}
	% Alph page numbering for preamble.
	\pagenumbering{Alph}

	\Metadata{
		title = Variational Autoencoders for Koopman Dynamical Systems,
		author = Fabian Damken
	}

	\title{Variational Autoencoders for Koopman Dynamical Systems}
	\subtitle{Variational Autoencoder für Koopman-Dynamische Systeme}

	\author{Fabian Damken}
	\birthplace{Frankfurt am Main}
	\reviewer{Jan Peters \and Joe Watson}
	\department{inf}
	\institute{TU Darmstadt}
	\group{Intelligent Autonomous Systems}
	\addTitleBoxLogo*{\includegraphics[width=0.75\linewidth]{img/iasLogo.pdf}}

	\submissiondate{November 20, 2020}
	\examdate{\today}

	\maketitle

	% Include meta-content.
	% TODO: Abstract!

\begin{abstract}[1]
	Here goes a really nice abstract.
\end{abstract}

\selectlanguage{ngerman}
\begin{abstract}[2]
	Hier kommt eine nice Kurzfassung.
\end{abstract}
\selectlanguage{english}

	\chapter*{Acknowledgments}



First of all I would like to thank my supervisor Joe Watson who not only introduced the whole Koopman theory to me and had the idea for this topic, but who always provided strong and invaluable feedback. I would also like to thank him for impressively quick responses to any question I had.

Furthermore I would like to thank Jan Peters, who first aroused my interest in robotics and machine learning.

I want to thank my fellow student Claas for always being there when I had any question and providing feedback from the beginning. I also want to express my appreciation for many helpful discussions in the past.

I want to thank Claas, Heiko, Stefanie and Tim for proofreading this thesis and providing feedback to the last minute. I also want to thank my parents for their support especially throughout my studies and the time of writing this thesis.


	% Sadly, this thesis submission is purely digital…
	\begin{affidavit*}[ngerman]{Erklärung zur Abschlussarbeit gemäß \S{}22~Abs.~7~APB TU~Darmstadt}
		Hiermit versichere ich, Fabian Damken, die vorliegende Bachelorarbeit gemäß \S{}22~Abs.~7~APB der TU Darmstadt ohne Hilfe Dritter und nur mit den angegebenen Quellen und Hilfsmitteln angefertigt zu haben. Alle Stellen, die Quellen entnommen wurden, sind als solche kenntlich gemacht worden. Diese Arbeit hat in gleicher oder ähnlicher Form noch keiner Prüfungsbehörde vorgelegen.

		Mir ist bekannt, dass im Falle eines Plagiats (\S{}38~Abs.~2 ~APB) ein Täuschungsversuch vorliegt, der dazu führt, dass die Arbeit mit 5,0 bewertet und damit ein Prüfungsversuch verbraucht wird. Abschlussarbeiten dürfen nur einmal wiederholt werden.

		Bei einer Thesis des Fachbereichs Architektur entspricht die eingereichte elektronische Fassung dem vorgestellten Modell und den vorgelegten Plänen.
	\end{affidavit*}
	\AffidavitSignature[Darmstadt]

	% Roman page numbering for toc and glossaries.
	\pagenumbering{Roman}

	\tableofcontents

	\chapter*{Figures and Tables}
		\begingroup
		\let\clearpage\relax
		\listoffigures\listoftables
		\endgroup
	% end

	\chapter*{Abbreviations, Symbols and Operators}
		\glsaddall
		\printglossary[type = acronym,  title = List of Abbreviations, style = iasThesisGeneral]
		\printglossary[type = symbol,   style = iasThesisGeneral]
		\printglossary[type = operator, style = iasThesisOperators]
	% end

	% Arabic page numbering for content.
	\cleardoublepage
	\pagenumbering{arabic}

	% Include content.
	\chapter{Introduction}
	\label{c:introduction}
	\IMRADlabel{introduction}

	% People are doing complicated stuff --> back to more basic approaches.

	\todo{Introduction}

	\section{Basics of Dynamical Systems}
		% Probably shortly introduce ODEs, explain why linearization is important.
		% Highlight why local linarization methods (e.g. small angle approximations) fail on a large scale.

		\todo{Intro: Basics of Dynamical Systems}
	% end
% end

	\chapter{Related Work}
	\label{c:relatedWork}

	% See references mindmap.

	\todo{Related Work}
% end

	\chapter{Koopman Theory of Dynamical Systems}
\label{c:koopmanTheoryOfDynamicalSystems}



Out first contribution is based on the need of a linearization technique that generalized globally. We have seen this need in the previous chapter when looking at simple nonlinear systems and small angle approximations. This brings us directly to Koopman theory, original introduced by B.~Koopman in 1931~\cite{koopmanHamiltonianSystemsTransformation1931} in the context of Hamiltonian systems and transformations in Hilbert spaces. Considering a nonlinear discrete-time dynamical system
\begin{equation*}
	\vec{s}_{t + 1} = \vec{F}(\vec{s}_t),\quad \vec{F} : \R^k \to \R^k
\end{equation*}
and observations \( \vec{g} : \R^p \to \R^p \) of this system, \ac{ie} \( \vec{g}_t \coloneqq \vec{g}(\vec{s}_t) \), the infinite-dimensional \emph{Koopman operator} \(\mathcal{K}\) advances all of the measurements forward in time. This relation can also be expressed as the composition
\begin{equation*}
	\mathcal{K} \vec{g} \coloneqq \vec{g} \circ \vec{F} \qquad\iff\qquad \mathcal{K} \vec{g}(\vec{s}_t) = \vec{g}\big( \vec{F}(\vec{s}_t) \big) = \vec{g}(\vec{s}_{t + 1})
\end{equation*}
This relation is true for every possible measurement function \( \vec{g} \) at any point of the system space \( \R^k \)~\cite{bruntonKoopmanInvariantSubspaces2016}. All possible measurement functions \( \vec{g} \) span an infinite-dimensional Hilbert space \( \mathcal{G} \). A finite set of measurements \( \vec{g}_1, \vec{g}_2, \cdots, \vec{g}_p \) that span an invariant subspace \( \mathcal{G}' \subset \mathcal{G} \), \ac{ie} applying the Koopman operator to a linear combination of these functions keeps them in the subspace
\begin{align*}
	\vec{g} &= \alpha_1 \vec{g}_1 + \alpha_2 \vec{g}_2 + \cdots + \alpha_p \vec{g}_p \\
	\mathcal{K} \vec{g} &= \beta_1 \vec{g}_1 + \beta_2 \vec{g}_2 + \cdots + \beta_p \vec{g}_p
\end{align*}
can be considered as a basis of that subspace. If that is possible, we can restrict the Koopman operator onto \( \mathcal{G}' \). Functions that both span the subspace \(\mathcal{G}'\) and do only scale when applying the Koopman operator, \ac{ie} \( \mathcal{K} \vec{\varphi} = \lambda \vec{\varphi} \) are called \emph{Eigenfunctions} of the Koopman operator. Finding these Eigenfunctions is extremely desirable, as it allows us to get a finite Koopman operator \( \mat{K} \) globally linearizing the dynamical system \( \vec{F} \).

For this thesis, we do not focus on finding the actual Eigenfunctions but to just approximate them.

	\chapter{Inference in Dynamical Systems}
	\label{c:inferenceInDynamicalSystems}

	\todo{Inference in Dynamical Systems}

	\section{Hidden Markov Models and LGDS}
		% HMM with cintinuous state --> LGDS

		\todo{Inference: HMM and LGDS}
	% end

	\section{Filtering and Smoothing}
		\subsection{Filtering and Smoothing}

		\subsection{Kalman Filter}
			\todo{Filtering: Kalman}
		% end

		\subsection{Rauch-Tung-Striebel/Kalman Smoother}
			\todo{Filtering: RTS}
		% end
	% end

	\section{Cubature Rules and Filtering}
		\todo{Inference: Cubature}

		\subsection{Square-Root Filtering/Smoothing}
			\todo{Cubature: Sqrt Filtering}
		% end
	% end
% end

	\chapter{The Nonlinear Gaussian Koopman Algorithm}  % TODO: Find better name.
\label{c:nonlinearGaussianKoopman}
\IMRADlabel{methods}



In this chapter we will introduce out contribution and the theoretical background of the \algname algorithm that we will implement as a proof of concept in~\autoref{c:experiments}. We will start by introducing the ideas that lead to the idea, then formulate and also solve the arising (approximate) inference problem.

By looking at the graphical model for an \ac{lgds} and the state transition model for a Koopman dynamical system in~\autoref{fig:lgdsKoopmanRelation}, we can see that there are lots of similarities. The first and most interesting similarity is that both systems assume a latent state that transitions linearly, either with a state dynamics matrix \(\mat{A}\) for the \ac{lgds} or with the Koopman operator\footnote{From now on, we assume a finite-dimensional matrix approximation \(\mat{K}\) of the Koopman operator \(\mathcal{K}\).} \(\mat{K}\). The greatest difference is that measurements in classic \ac{lgds} are taken linearly with an observation matrix \( \mat{C} \) and nonlinear in the Koopman system with the measurement function \( \vec{h}(\cdot) \).

\begin{figure}
	\centering
	\begin{subfigure}[t]{0.5\linewidth}
		\centering
		\resizebox{\linewidth}{!}{\tikzLinearGaussianDynamicalSystem}
		\caption{Graphical model of a linear Gaussian dynamical system with the latent state \(\vec{s}_t\) and the (linear) observations \(\vec{y}_t\). In contrast to the Koopman system, this system is not deterministic and the arrows represent probabilistic dependency rather than hard transitions.}
	\end{subfigure}%
	\begin{subfigure}[t]{0.5\linewidth}
		\centering
		\resizebox{\linewidth}{!}{\tikzKoopmanOperator}
		\caption{State transition model of a Koopman dynamical system with the Koopman operator \( \mathcal{K} \) that can be approximated with a matrix \(\mat{K}\). In contrast to the \ac{lgds}, the state is represented via nonlinear transitions and the observations are the linear dynamics. \\ Adopted from~\cite{bruntonKoopmanInvariantSubspaces2016}.}
	\end{subfigure}
	\caption{Side-by-side comparison of the a \ac{lgds} on the left and a Koopman dynamical system on the right. This side-by-side view highlights our idea of interpreting an Koopman system as a semi-linear dynamical system (\ac{ie} an \ac{lgds} with nonlinear observations).}
	\label{fig:lgdsKoopmanRelation}
\end{figure}

Our idea is to "flip" the Koopman dynamical system and replace the observation matrix \( \mat{C} \) with a nonlinear observation function \( \vec{g}(\cdot) \), that takes the linear states \( \vec{s}_t \) and maps them into a nonlinear observation space \( \vec{y}_t \). In other words, we seek to find the inverse function of \( \vec{h}(\cdot) \) to map out of the linear embedding that Koopman theory guarantees us to exist. Our belief that such an inverse function exists is backed by previous accomplishments in data-driven Koopman analysis~\cite{luschDeepLearningUniversal2018} that also seek and find such an inverse mapping (see~\ref{c:relatedWork} for more information).

Additionally, we contribute a probabilistic view on the Koopman operator, being able to gauche our uncertainty about the embedding and the inverse mapping to the observation space. Speaking of the observation space, this is a good point to clear up chaos of notation and names that builds up when working with two systems where "observation" means contrary concepts. From now on, we will work on the graphical system shown in~\autoref{fig:nonlinearGaussianKoopman} characterized by the dynamics
\begin{align*}
	\vec{s}_{t + 1} &= \eqmakebox[ngkIntro][l]{\( \mat{A} \vec{s}_t + \vec{w}_t,\quad \vec{w}_t \)} \sim \normal(\vec{0}, \mat{Q}) \\
	\vec{y}_t       &= \eqmakebox[ngkIntro][l]{\( \vec{g}(\vec{s}_t) + \vec{v}_t,\quad \vec{v}_t \)} \sim \normal(\vec{0}, \mat{R})
\end{align*}
that can equivalently be formulated as
\begin{align*}
	\vec{s}_{t + 1} &\sim \normal(\mat{A} \vec{s}_t, \mat{Q}) \\
	\vec{y}_t       &\sim \normal\big(\vec{g}(\vec{s}_t), \mat{R}\big)
\end{align*}
We call \( \vec{s}_t \) the \emph{latent variables} or \emph{latents} in the \emph{latent space} with dimensionality \(k\), \( \vec{y}_t \) the \emph{observation variables} or \emph{observations} in the \emph{observation space} with dimensionality \(p\), \( \vec{g} : \R^k \to \R^p \) the \emph{observation function} mapping latents to observations, \( \mat{A} \) the \emph{state dynamics matrix}, \( \mat{Q} \) the state covariance matrix and \( \mat{R} \) the observation covariance matrix.

\begin{figure}
	\centering
	\tikzNonlinearGaussianKoopman
	\caption{The graphical model of the \algname model. Given an observation sequence \( \vec{y}_{1:T} \), we seek the latent dynamics and the corresponding nonlinear mapping \( \vec{g}(\cdot) \) from the latent space to the observations.}
	\label{fig:nonlinearGaussianKoopman}
\end{figure}

%\section{Formulating and Solving the Nonlinear Approximate Inference Problem using an Approximate EM Algorithm}
\section{Formulating and Solving the Inference Problem using an EM Algorithm}
	% Formulate likelihood, expected likelihood.
	% Shortly outline how to do the derivation and reference appendix.
	% Summarize M-step equations.
	% Combine ideas from ch. 2 (cubature and sqrt) and summarize.

	\todo{NGK: Formulating and Solving}
% end

\section{Implementation}
	% Discuss training tricks (max. iterations and stuff).
	% Find reasons why and explain difficulties with learning from multiple observation sequences at once.
	% Highlight more performant QR decomposition on the GPU.

	\todo{NGK: Implementation}

	\subsection{Problems and Solutions}
		\todo{Impl: Problems}
	% end

	\subsection{Notes on Numerical Stability}
		\todo{Impl: Numerical Stability}
	% end
% end

	\chapter{Experiments}
\label{c:experiments}
\IMRADlabel{results}



In this chapter we will present all environments we experimented with as well as \emph{Hyper-Experiments}, \ac{ie} experiments with the hyperparameters and their influence on the algorithm performance (in terms of the prediction error and similar metrics, not the computing time).

\section{Environments and Experiment Setup}
	We will not introduce to you the environments and the setup for each environment that we used. We will not discuss any results here,see~\autoref{c:discussion} for that. Do generate the data, you have to run the file \texttt{src/data.py} with the desired experiment ID as the first argument, \ac{eg} \texttt{python src/data.py pendulum\_damped}. If no argument is given, a regeneration of the data from all experiments is triggered. For each of the following environments we summarize the most important data about the environment at the start of the subsection, including the experiment ID.

	\subsection{Proof of Concept: Classic LGDS}
		\begin{itemize}
			\item Experiment ID: \texttt{lgds}
		\end{itemize}

		As a proof of concept and to empirically verify our proof on the exactness of the cubature rule and hence the similar expected performance for plain linear systems, we benchmark our algorithm against a simple linear system. The linear system has a two-dimensional state \( \vec{s} \coloneqq (x_2, x_2)^T \) with the dynamics
		\begin{equation*}
			\dot{\vec{s}} =
				\begin{bmatrix}
					 0 & 1 \\
					-1 & 0
				\end{bmatrix} \vec{s}
		\end{equation*}
		The initial state \( \vec{s}_1 \) is sampled from a Gaussian with mean \( (0.1,\, 0.2)^T \) and covariance \( \diag\big(10^{-5}, 10^{-5}\big) \). The system is integrated using the implicit Runge-Kutta Radau~IIA~\cite{guglielmiImplementingRadauIIA2001} method with an evaluation interval of \( h = 0.1 \) for \(T 0 240\) time steps where only the first \(T_\train = 120\) steps are used for training and the remaining \(120\) are used for validation. In total \(N = 3\) sequences are generated, each with a different initial value. The raw data is shown in~\autoref{fig:envLgds}.

		\begin{figure}
			\centering
			\includegraphics[width=\linewidth]{figures/experiments/environments/observations-lgds.pdf}
			\caption{Plot of the raw data used for training the proof-of-concept \ac{lgds} environment. The black dots represent the actual data points, all before the red "prediction border" are used for training, the rest for validation. The faint gray line emphasizes the connection between the data points and that they are actually generated from a dynamical system.}
			\label{fig:envLgds}
		\end{figure}
	% end

	\subsection{(Damped) Pendulum}
		\todo{Exp Envs: (Damped) Pendulum}
	% end

	\subsection{Gym Pendulum}
		\todo{Exp Envs: Gym Pendulum}
	% end

	\subsection{Gym Cartpole}
		\todo{Exp Envs: Gym Cartpole}
	% end

	\subsection{Gym Acrobot (Double Pendulum)}
		\todo{Exp Envs: Gym Acrobot}
	% end
% end

\section{Experiments with Hyperparameters}
	% Probably more if I can run more experiments… But all other hyperparameters are not as interesting.

	\todo{Exp: Hyper-experiments}

	\subsection{Influence of the Latent Dimensionality}
		\label{subsec:experimentLatentDim}

		\todo{Exp Hyper: Latent Dim}
	% end
% end

	\chapter{Discussion}
	\label{c:dicussion}
	\IMRADlabel{dicsussion}

	\todo{Experiments}

	\section{Performance}
		\todo{Exp: Performance}
	% end

	\section{Comparison with Related Work}
		% In-Depth Comparison with Lush et al.
		% In-Depth Comparsion with Morton et al.
		% Probably more qualitative comparisons with Sequential VAE or DVBF (Karl et al.).

		\todo{Exp: Comparison}
	% end
% end

	\chapter{Conclusion}
	\label{c:conclusion}

	\todo{Conclusion}

	\section{Summary}
		\todo{Conclusion: Summary}
	% end

	\section{Control}  % Only if there is time left!
		% Introduce approach to control.
		% Show results of LGDS control which is working.
		% Highlight difficulties.

		\todo{Conclusion: Control}
	% end

	\section{Future Work}
		% Control
		% Full Bayesian
		% Automatic Relevance Determination (ARD) on Latent Space, see Beal's Variational Kalman Smoother

		\todo{Conclusion: Future Work}
	% end
% end


	\cleardoublepage
	\bibliography{literature/lit}
	\nocite{*}

	\cleardoublepage
	\appendix
	\chapter{Example Appendix}
	Here goes some lengthy figures or similar\dots
% end

	\chapter{Snippets}
		\section{Basics}



\subsection{Dynamical Systems}
	A \emph{dynamical system}~\cite{birkhoffDynamicalSystems1927} is a (physical) system that evolves over time \(t\) and is completely defined by the values of \(n\) real variables
	\begin{align*}
		x_1, x_2, \,\cdots\!, x_n \quad\longleftrightarrow\quad \vec{x} = \begin{bmatrix} x_1 & x_2 & \cdots & x_m \end{bmatrix}^T
	\end{align*}
	called the \emph{state} and often written in vector form (right). Given the differentiability of these values, we can also study their rate of change (often referred to as the "velocity") and the rate of change of the rate of change (often referred to as the "acceleration"):
	\begin{align*}
		\dot{\vec{x}} = \dv{\vec{x}}{t} \qquad \ddot{\vec{x}} = \dv[2]{\vec{x}}{t}
	\end{align*}
	Describing these systems is possible using differential equations, both ordinary and partial ones. A general \ac{ode} is given by an implicit equation
	\begin{align}
		\vec{0} = \vec{F}\big( \vec{x}, \vec{x}^{(1)}, \vec{x}^{(2)}, \,\cdots\!, \vec{x}^{(k - 1)}, \vec{x}^{(k)}; t \big),\quad \vec{x}^{(l)} \coloneqq \dv[l]{\vec{x}}{t}  \label{eq:ode}
	\end{align}
	which establishes a connection between the state itself and its time derivatives. We call a function \( \vec{x}(t) \) a \emph{solution} of an \ac{ode} if its derivatives fulfill the given \ac{ode}~\eqref{eq:ode}. We will now employ some definitions and terms that we will use throughout the whole thesis.
	\begin{description}[leftmargin = 3cm]
		\item[Order] If \( \vec{x}^{(k)} \) is the derivative of highest order that appears in the \ac{ode}, the \ac{ode} is called to be of order \(k\).
		\item[Linearity] An \ac{ode} is \emph{linear} if \(\vec{F}\) is a linear function in terms of the state and its derivatives, \ac{ie} it is given as a linear combination
	\end{description}
	\begin{align*}
		\vec{F} = \vec{r}(t) + \sum_{i = 1}^{k} c_i(t) \vec{x}^{(i)},\quad \vec{r}(t) : \R \to \R^n,\, c_i(t) : \R \to \R
	\end{align*}
	\begin{description}[leftmargin = 3cm]
		\item[Autonomous] If \(\vec{F}\) explicitly is independent of \(t\) (\ac{ie} \( \pdv{\vec{F}}{t} = \vec{0} \)), the \ac{ode} is called \emph{autonomous}.
		\item[Homogenity] If no term  of \(\vec{F}\) is independent of the state or its derivatives, the \ac{ode} is called \emph{homogeneous}. For any homogeneous \ac{ode} one of its solutions is the trivial solution \( \vec{x} \equiv \vec{0} \).
	\end{description}

	In all of the following, we assume to have explicit, autonomous, first order \acp{ode}. This is valid because we can transform every explicit higher order \ac{ode} into a system of first order \acp{ode} as well as we can introduce another "time state" which makes our \ac{ode} autonomous.

	The solution theory for linear \acp{ode} is highly evolved and solutions exist for nearly every possible \ac{ode}. But for nonlinear \acp{ode}, the world looks different. With the exception of some rare cases, nonlinear \acp{ode} are not tractable. Hence, we often need approximations for the nonlinear case. Some well-known approaches for these approximations are \ac{eg} \emph{small angle approximation} for Sines and Cosines. In small angle approximations, we Taylor-expand \( \sin \)/\( \cos \) at \( \varphi_a = 0 \) and cut all higher order terms:
	\begin{gather*}
		\sin(\varphi) = \varphi - \underbrace{\frac{\varphi^3}{3!} + \frac{\varphi^5}{5!} + \cdots}_\text{higher order terms} \approx \varphi \\
		\cos(\varphi) = 1 - \underbrace{\frac{\varphi^2}{2!} + \frac{\varphi^4}{4!} - \frac{\varphi^6}{6!} + \cdots}_\text{higher order terms} \approx 1
	\end{gather*}
	This approach is illustrated in~\autoref{fig:smallAngleApproximation}.

	\begin{figure}
		\centering
		\begin{subfigure}[t]{0.5\linewidth}
			\centering
			\includegraphics[width = \linewidth]{figures/introduction/generated/small-angle-approximation-sin.pdf}
			\caption{Small angle approximation \( \sin(\varphi) \approx \varphi \) of Sine.}
		\end{subfigure}%
		~
		\begin{subfigure}[t]{0.5\linewidth}
			\centering
			\includegraphics[width = \linewidth]{figures/introduction/generated/small-angle-approximation-cos.pdf}
			\caption{Small angle approximation \( \cos(\varphi) \approx 1 \) of Cosine.}
		\end{subfigure}
		\caption{Visualization of the small angle approximation (given in orange) of the basic trigonometric functions Sine and Cosine (given in blue). It is clear that the approximation is only valid in a small region around zero (\( \varphi \approx 0 \)).}
		\label{fig:smallAngleApproximation}
	\end{figure}

	We now look at two examples of dynamical systems, one of which is linear and one that is not.

	\paragraph{Harmonic Oscillator}
		\label{subsec:harmonicOscillator}

		\begin{figure}
			\centering
			\tikzHarmonicOscillator
			\caption{Illustration of a simple harmonic oscillator with mass \(m\), spring stiffness \(k\) and position \(x\) that is not affected by any external force like gravity. The mass is in equilibrium if \( x = 0 \).}
			\label{fig:simpleHarmonicOscillator}
		\end{figure}

		The \emph{simple harmonic oscillator} describes the dynamical system of a mass \(m\) that is attached to a spring that is following Hooke's Law with stiffness \(k\) (see~\autoref{fig:simpleHarmonicOscillator}). This harmonic oscillator is described by the \ac{ode}
		\begin{align}
			m\ddot{x} = -kx \quad\iff\quad \ddot{x} = -\frac{k}{m} x  \label{eq:harmonicOscillator}
		\end{align}
		where \(x\) and \(\ddot{x}\) are the position and acceleration of the mass, respectively. Note that if \( x = 0 \), the mass is in equilibrium and no force is acting on it.

		By using basic results in the solution theory of linear \acp{ode}, we see that the general solution is given as
		\begin{align*}
			x(t) = A \cos\Big(t \sqrt{k / m} + \varphi\Big)
		\end{align*}
		with the amplitude \(A\) and the phase \(\varphi\) (see~\autoref{app:harmonicOscillatorSolution} for the derivation of the solution). As neither gravity nor damping or other external forces are involved in the dynamical system, the motion continues forever with a non-changing amplitude.
	% end

	\paragraph{Simple Pendulum}
		\label{subsec:simplePendulum}

		\begin{figure}
			\centering
			\tikzSimplePendulum
			\caption{Illustration of an inverse pendulum with mass \(m\) and displacement \(\varphi\) that is only affected by gravity and no other external force. The mass is in equilibrium for both \( \varphi = 0 \) and \( \varphi = \pi \), where the former is an unstable equilibrium.}
			\label{fig:simplePendulum}
		\end{figure}

		The \emph{inverse pendulum} describes the dynamical system of a mass \(m\) that is attached to a rigid pole of length \(L\) which can freely swing around a suspension point (see~\autoref{fig:simplePendulum}). The pendulum stands upright if \( \varphi = 0 \) and its equation of motion is described by the \ac{ode}
		\begin{align*}
			\ddot{\varphi} = \frac{g}{L} \sin(\varphi)
		\end{align*}
		where \(g\), \(L\), \(\varphi\) and \(\ddot{\varphi}\) describe the gravity acceleration, pole length, displacement and acceleration of the mass, respectively. Note that if \( \varphi = 0 \), the mass is in an unstable equilibrium and no force is acting on it.

		In comparison to the harmonic oscillator (\autoref{subsec:harmonicOscillator}), this differential equation is nonlinear. And, even for the case with unit gravity acceleration \( g = 1 \) and unit pole length \( L = 1\), where the \ac{ode} looks really simple
		\begin{align}
			\ddot{\varphi} = \sin(\varphi)  \label{eq:inversePendulum}
		\end{align}
		it is not tractable analytically (\ac{ie} there exists no solution in closed form).

		Still, we can apply the small angle approximation introduced before (in this case, \( \sin(\varphi) \approx \varphi \)) which yields the simple equation
		\begin{align}
			\ddot{\varphi} \approx \varphi  \label{eq:linearizedInversePendulum}
		\end{align}
		which is solved by
		\begin{align*}
			\varphi(t) = \frac{1}{2} e^{-t} \big(\varphi_0 + e^{2t} \varphi_0 - \dot{\varphi}_0 + e^{2t} \dot{\varphi}_0\big)
		\end{align*}
		where \(\varphi_0\) and \(\dot{\varphi}_0\) are the initial displacement and velocity, respectively.

		However, this small angle approximation can only forecast small displacements \( \varphi \ll \pi/2 \). And, as the equilibrium at \( \varphi = 0 \) is unstable, the approximation becomes worse as time goes by because the pendulum falls down. This behavior is shown in~\autoref{fig:inversePendulumApprox}.

		\begin{figure}
			\centering
			\begin{subfigure}[t]{0.5\linewidth}
				\centering
				\includegraphics[width = \linewidth]{figures/introduction/generated/pendulum-motion-solutions}
				\caption{Trajectories of two solution strategies to the inverse pendulum, where the blue is a numerical solution of the actual motion of equation (solved using the Radau~IIA method~\cite[72]{hairerSolvingOrdinaryDifferential1996}) and the orange one is the analytically computed solution linearized \ac{ode}. The latter is linearized using small angle approximation. The dashed gray vertical line shows when the distance tolerance of \( \varepsilon = 10^{-3} \) is exceeded.}
			\end{subfigure}%
			~
			\begin{subfigure}[t]{0.5\linewidth}
				\centering
				\includegraphics[width = \linewidth]{figures/introduction/generated/pendulum-motion-difference}
				\caption{Differences between the small angle approximation and the numerical solution of the \ac{ode}. The dashed gray vertical line shows when the distance tolerance of \( \varepsilon = 10^{-3} \) is exceeded.}
			\end{subfigure}
			\caption{Comparison of a numerical solution to the \ac{ode} of the inverse pendulum given in\eqref{eq:inversePendulum} and the analytical solution of the linearized \ac{ode} given in\eqref{eq:linearizedInversePendulum}. A tolerance value of \( \varepsilon = 10^{-3} \) is used to show when both solutions diverge from each other.}
			\label{fig:inversePendulumApprox}
		\end{figure}
	% end

	\paragraph{Discrete-Time Dynamical Systems}
		In comparison to continuous-time dynamical systems described by \acp{ode}, discrete-time systems are described by a \emph{dynamics function} \( \vec{F} : \R^n \to \R^n \) advancing all states forward in time:
		\begin{align*}
			\vec{x}_{t + 1} = \vec{F}(\vec{x}_t)
		\end{align*}
		But we should note that, while seeming more restrictive, discrete-time dynamical systems are more general than continuous-time systems as we can discretize every continuous-time system
		\begin{align*}
			\dot{\vec{x}} = \vec{f}(\vec{x})
		\end{align*}
		as a discrete-time dynamical system
		\begin{align*}
			\vec{x}_{t + 1} = \vec{F}(\vec{x}_t)
		\end{align*}
		using the state dynamics function
		\begin{align*}
			\vec{F}\big(\vec{x}(t_0)\big) = \vec{x}(t_0 + \Delta_t) = \vec{x}(t_0) + \int_{t_0}^{t_0 + \Delta_t} \! \vec{f}\big(\vec{x}(\tau)\big) \dd{\tau}
		\end{align*}
		where \( \Delta_t \) is called the \emph{discretization interval} and \( \vec{x}_k = \vec{x}(k \Delta_t) \). Note that, from the Nyquist-Shannon sampling theorem, we know for band-limited functions that the sampling rate \( f_s = 1/\Delta_t \) has to be greater than two times the maximum frequency of the original function~\cite{shannonCommunicationPresenceNoise1949}, \ac{ie} \( f_s = 1/\Delta_t > 2 f \).
	% end

	\subsubsection{Koopman Dynamical Systems and the Koopman Operator}
		As we have seen, classical linearization approaches like the small angle approximation are only valid in a narrow section around the linearization point. While this works well for simple control tasks such as swinging up and balancing an inverted pendulum~\cite{bugejaNonlinearSwingupStabilizing2003}, it does not work out for predicting future trajectories of the system. % TODO: Citation needed.

		Consider a first-order, autonomous dynamical system
		\begin{align*}
			\dot{\vec{x}} = \vec{f}(\vec{x}),\quad \vec{f} : \R^n \to \R^n
		\end{align*}
		and observables ("measurements") \( \vec{g} : \R^n \to \R^m \). The Koopman operator \( \mathcal{K} \), an infinite-dimensional linear operator, acts on them as follows:
		\begin{align*}
			\mathcal{K} \vec{g} = \vec{g} \circ \vec{F} \quad\iff\quad \mathcal{K} \vec{g}(\vec{x}_t) = \vec{g}\big(\vec{F}(\vec{x}_t)\big) = \vec{g}(\vec{x}_{t + 1})
		\end{align*}
		In other words, the Koopman operator advances our observables \(\vec{g}\) linearly forward in time. This behavior is illustrated in~\autoref{fig:koopmanOperatorBrunton}.

		The Koopman operator can also be formulated for time-continuous systems~\cite{abrahamManifoldsTensorAnalysis2012}:
		\begin{align*}
			\dv{t} \vec{g} = \mathcal{K} \vec{g}
		\end{align*}

		% TODO: Further explain Koopman theory after reading the corresponding sections in the Abraham book.

		\begin{figure}
			\centering
			\tikzKoopmanOperator
			\caption{Illustration of a Koopman dynamical system where time flows from left to right. The top row shows the linear embedding with the Koopman operator \( \mathcal{K} \) to transition from \(\vec{y}_t\) to \(\vec{y}_{t + 1}\). The bottom row shows the nonlinear dynamics produced by the dynamics function \(\vec{F}\). The measurement function \(\vec{g}\) allows transitioning between the two representations of the system state where \(\vec{g}^{-1}\) describes an "inverse" measurement function to recover the original nonlinear state from the linear embedding. We would like to find such a function. Adopted from~\cite{bruntonKoopmanInvariantSubspaces2016}.}
			\label{fig:koopmanOperatorBrunton}
		\end{figure}
	% end
% end

\subsection{Hidden Markov Models and Linear Gaussian Dynamical Systems}
	\subsubsection{Hidden Markov Models}
		\label{subsec:hiddenMarkovModel}

		\acp{hmm} are simple Bayesian networks described by a non-observable (\emph{hidden}) Markov chain. This non-observable discrete chain can be indirectly observed with measurements that are emitted by every state. One of the key features of a Markov chain is that a state does only depend on the previous state, but not on the second previous, third previous, and so on, states. That is, knowing only the previous state is sufficient and no more information can be gathered by knowing every other state before. This is described by the state transition distribution
		\begin{align*}
			s_{k + 1} \sim p(s_{k + 1} \given s_k)
		\end{align*}
		being only dependent on the previous state \(s_k\). Analogous, a measurement \(\vec{y}_k\) of a state \(s_k\) is only dependent on that specific state:
		\begin{align*}
			\vec{y}_k \sim p(\vec{y}_k \given s_k)
		\end{align*}
		These assumptions are called the \emph{Markov property} and a system fulfilling this property is called \emph{Markovian}. These conditional distributions can be written in a graphical model as shown in~\autoref{fig:hiddenMarkovModel}. Note that, in general, no assumption has to be made on the "type" of state/observation (\ac{ie} whether it is a scalar, a vector or something completely different). Also, no assumption is made on the specific transition distributions, \acp{hmm} can also be used to model deterministic transitions using a Dirac delta distribution.

		\begin{figure}
			\centering
			\tikzHiddenMarkovModel
			\caption{Illustration of a completely general Hidden Markov Model with states \(s_k\) and emissions/observations \(\vec{y}_k\).}
			\label{fig:hiddenMarkovModel}
		\end{figure}
	% end

	\subsubsection{Linear Gaussian Dynamical Systems}
		A general time-discrete linear system is described by a state transition
		\begin{align*}
			\vec{x}_{t + 1} = \mat{A} \vec{x}_t
		\end{align*}
		with a dynamics matrix \(\mat{A}\). Using purely additive Gaussian zero-mean noise \( \vec{\epsilon} \sim \normal(\vec{0}, \mat{Q}) \) with covariance matrix \( \mat{Q} \), the state transition becomes probabilistic:
		\begin{align*}
			\vec{x}_{t + 1} = \mat{A} \vec{x}_t + \vec{\epsilon} \quad\iff\quad \vec{x}_{t + 1} \sim \normal(\mat{A} \vec{x}_t, \mat{Q})
		\end{align*}
		With analogously defined measurements,
		\begin{align*}
			\vec{y}_t \sim \normal(\mat{C} \vec{x}_t, \mat{R})
		\end{align*}
		the dynamical system becomes a \ac{lgds}, which is very similar to the \ac{hmm} described in~\autoref{subsec:hiddenMarkovModel}. It can also be represented using a graphical model as shown in~\autoref{fig:lgds}. In fact, a \ac{lgds} is kind of a \ac{hmm}, except the states are not discrete but continuous.

		\begin{figure}
			\centering
			\tikzLinearGaussianDynamicalSystem
			\caption{Illustration of a Linear Gaussian Dynamical System with states \(\vec{x}_t\) and observations \(\vec{y}_t\). Solid arrows represent probabilistic dependency, where the matrix \(\mat{A}\) is the dynamics matrix indicating that the mean transitions linearly, so do the observations with the observation matrix \(\mat{C}\).}
			\label{fig:lgds}
		\end{figure}
	% end

	\subsubsection{Inference}
		% TODO: Stopped here.
	% end
% end

\subsection{The Expectation-Maximization Algorithm}
	The \ac{em} algorithm, first introduced by Ceppellini et al. in 1955~\cite{ceppelliniEstimationGeneFrequencies1955} and popularized by Dempster et al. in 1977~\cite{dempsterMaximumLikelihoodIncomplete1977}, can be used for tackling the following optimization problem: Assuming some model with latent (hidden) states \(\vec{x}\), observations \(\vec{y}\) and model parameters \(\vec{\theta}\), we want to maximize the likelihood \( p(\vec{y} \given \vec{\theta}) \) \ac{wrt} the latent states \(\vec{x}\) and the parameters \(\vec{\theta}\). However, the marginal distribution
	\begin{align*}
		p(\vec{y} \given \vec{\theta}) = \int\! p(\vec{x}, \vec{y} \given \vec{\theta}) \dd{\vec{x}}
	\end{align*}
	is generally intractable. As usual on maximum likelihood approaches, it is useful to not maximize the likelihood directly, but to maximize the log-likelihood
	\begin{align*}
		\mathcal{L}(\vec{\theta}) \coloneqq \log p(\vec{y} \given \vec{\theta}) = \log \int\! p(\vec{x}, \vec{y} \given \vec{\theta}) \dd{\vec{x}}
	\end{align*}
	instead. This yields the same maximum as the logarithm is strictly increasing. By introducing an auxiliary probability distribution \( q(\vec{x} \given \vec{y}) \) over the latent variables, we can rewrite the marginal and find a lower bound on \(\mathcal{L}\) by using Jensen's inequality~\cite{jensenFonctionsConvexesInegalites1906}:
	\begin{align}
		\mathcal{L}(\vec{\theta})
			&= \log \int\! p(\vec{x}, \vec{y} \given \vec{\theta}) \dd{\vec{x}}  \nonumber \\
			&= \log \int\! q(\vec{x} \given \vec{y}) \frac{p(\vec{x}, \vec{y} \given \vec{\theta})}{q(\vec{x} \given \vec{y})} \dd{\vec{x}}  \nonumber \\
			&\geq \int\! q(\vec{x} \given \vec{y}) \log \frac{p(\vec{x}, \vec{y} \given \vec{\theta})}{q(\vec{x} \given \vec{y})} \dd{\vec{x}}  \nonumber \\
			&= \int\! q(\vec{x} \given \vec{y}) \log p(\vec{x}, \vec{y} \given \vec{\theta}) \dd{\vec{x}} - \int\! q(\vec{x} \given \vec{y}) \log q(\vec{x} \given \vec{y}) \dd{\vec{x}} \eqqcolon \mathcal{L}_\mathrm{EM}[q, \vec{\theta}]  \label{eq:emLowerBound}
	\end{align}
	This draws a lower bound \( \mathcal{L}_\mathrm{EM}[q, \vec{\theta}] \) on the log-likelihood \( \mathcal{L}(\vec{\theta}) \) we can maximize instead, maximizing the log-likelihood simultaneously. Note that this lower bound is in fact a functional of the distribution \( q(\vec{x} \given \vec{y}) \).

	The \ac{em} algorithm now iteratively maximizes the lower bound and thus indirectly maximizes the original objective, the likelihood \( p(\vec{y} \given \vec{\theta}) \). The E and M step are as follows:
	\begin{description}
		\item[E-Step] Infers the auxiliary distribution \( q(\vec{x} \given \vec{y}) \) using the current estimations of the parameters \( \vec{\theta} \) by maximizing the lower bound \ac{wrt} the auxiliary distribution.
		\item[M-Step] Infers the parameters \(\vec{\theta}\) using the current auxiliary distribution by maximizing the lower bound \ac{wrt} the parameters.
	\end{description}
	Expressed in equations with the index \( \cdot^{(n)} \) to denote the values of the \(n\)-th iteration, we get the procedure which will be repeated until convergence:
	\begin{description}
		\item[E-Step] \eqparbox[r]{emSteps}{\(\displaystyle q^{(n + 1)} \)} \(\displaystyle = \arg\max_{q}\, \mathcal{L}_\mathrm{EM}\big[ q, \vec{\theta}^{(n)} \big] \)
		\item[M-Step] \eqparbox[r]{emSteps}{\(\displaystyle \vec{\theta}^{(n + 1)} \)} \(\displaystyle = \arg\max_{\vec{\theta}}\, \mathcal{L}_\mathrm{EM}\big[ q^{(n + 1)}, \vec{\theta} \big] \)
	\end{description}
	Additionally, the E-step has the constraint that \(q(\vec{x} \given \vec{y})\) really is a probability distribution, so it must integrate to one:
	\begin{align*}
		\int\! q(\vec{x} \given \vec{y}) \dd{\vec{x}} = 1
	\end{align*}
	We can incorporate this into the maximization, \ac{eg} using Lagrange multipliers. Using a bit of variational calculus, it can be shown~\cite{bealVariationalAlgorithmsApproximate2003} that the maximization is gained by choosing the auxiliary distribution \(q(\vec{x} \given \vec{y})\) to be the same as the distribution \( p(\vec{x} \given \vec{y}, \vec{\theta}) \). That is, we set
	\begin{align*}
		q^{(n + 1)}(\vec{x} \given \vec{y}) = p\big(\vec{x} \biggiven \vec{y}, \vec{\theta}^{(n)}\big)
	\end{align*}
	while holding the parameters \(\vec{\theta}\) fixed. This maximization turns the lower bound into an equality with the actual likelihood~\cite{bealVariationalAlgorithmsApproximate2003}.

	For the M-step, we keep the auxiliary distribution fixed and maximize the lower bound \ac{wrt} the parameters \(\vec{\theta}\). As this involves taking the derivative of \(\mathcal{L}_\mathrm{EM}\) \ac{wrt} \(\vec{\theta}\), we can safely omit the right hand side of the lower bound as it does not depend on \(\vec{\theta}\). Hence, we get the M-step as:
	\begin{align*}
		\vec{\theta}^{(n + 1)}
			&= \arg\max_{\vec{\theta}}\, \mathcal{L}_\mathrm{EM}\big[ q^{(n)}, \vec{\theta} \big] \\
			&= \arg\max_{\vec{\theta}} \int\! q^{(n)}(\vec{x} \given \vec{y}) \log p(\vec{x}, \vec{y} \given \vec{\theta}) \dd{\vec{x}} \\
			&= \arg\max_{\vec{\theta}} \int\! p\big(\vec{x} \biggiven \vec{y}, \vec{\theta}^{(n)}\big) \log p(\vec{x}, \vec{y} \given \vec{\theta}) \dd{\vec{x}} \\
			&= \arg\max_{\vec{\theta}}\, \E_{p\big(\vec{x} \given \vec{y}, \vec{\theta}^{(n)}\big)}[\log p(\vec{x}, \vec{y} \given \vec{\theta})]
	\end{align*}
	The quantity to optimize, \( Q(\vec{\theta}) \coloneqq \E[\log p(\vec{x}, \vec{y} \given \vec{\theta})] \), is also called the \emph{expected complete log-likelihood} as it involves both the observables and the latent variables.

	As shown in~\cite{bealVariationalAlgorithmsApproximate2003}, the lower bound turns into an equality after the E-step. Hence we are guaranteed to always rise the log-likelihood after each \ac{em} iteration if we do not already have the optimal auxiliary distribution and parameters. If this would be the case, both the E- and the M-step would not change anything, so our log-likelihood is monotonically increasing. Also we might not want to calculate the whole distribution in the E-step, but we might want to restrict our computations to sufficient statistics that cover our whole distribution. This procedure may include constraining our auxiliary distribution to some kind of parametric family, \ac{eg} a Gaussian. The procedure for the \ac{em} algorithm is analogous, but the lower bound does not turn into an equality after the E-step, see~\cite[49-51]{bealVariationalAlgorithmsApproximate2003} for more details.

	For the rest of this thesis, we know our distribution \( p(\vec{x} \given \vec{y}, \vec{\theta}) \) is Gaussian, so we can assume the auxiliary distribution to be Gaussian too, given that we set them equal. Hence, we only need to calculate the mean and correlation or covariance of each variable to cover the whole distribution and to be able to proceed. This will turn out to be really useful later on.
% end

\subsection{Variational Autoencoders and the Evidence Lower Bound}
	\acp{vae} have been first introduced in the context of the \ac{aevb} algorithm by Kingma and Welling in 2014~\cite{kingmaAutoEncodingVariationalBayes2014}. They tackle inference and learning in probabilistic models with latent variables, similar to the \ac{em} algorithm. To keep the notation analogous to the derivation of the \ac{em} algorithm, we deviate from~\cite{kingmaAutoEncodingVariationalBayes2014} in terms that we keep \(\vec{x}\) as our latent variables and \(\vec{y}\) as the observables.

	To perform inference, we want to maximize the (log-) likelihood
	\begin{align*}
		\mathcal{L}(\vec{\theta}) \coloneqq \log p(\vec{y} \given \vec{\theta}) = \int\! p(\vec{x}, \vec{y} \given \vec{\theta}) \dd{\vec{x}}
	\end{align*}
	by maximizing the \ac{elbo} \( \mathcal{L}_\mathrm{ELBO} \) which we can find by using Jensen's inequality~\cite{jensenFonctionsConvexesInegalites1906}:
	\begin{align}
		\mathcal{L}(\vec{\theta})
			&= \log \int\! p(\vec{x}, \vec{y} \given \vec{\theta}) \dd{\vec{x}}  \nonumber \\
			&= \log \int\! q(\vec{x} \given \vec{y}, \vec{\phi}) \frac{p(\vec{x}, \vec{y} \given \vec{\theta})}{q(\vec{x} \given \vec{y}, \vec{\phi})} \dd{\vec{x}}  \nonumber \\
			&\geq \int\! q(\vec{x} \given \vec{y}, \vec{\phi}) \log \frac{p(\vec{x}, \vec{y} \given \vec{\theta})}{q(\vec{x} \given \vec{y}, \vec{\phi})} \dd{\vec{x}}  \nonumber \\
			&= \int\! q(\vec{x} \given \vec{y}, \vec{\phi}) \log \frac{p(\vec{y} \given \vec{x}, \vec{\theta}) p(\vec{x} \given \vec{\theta})}{q(\vec{x} \given \vec{y}, \vec{\phi})} \dd{\vec{x}}  \nonumber \\
			&= \int\! q(\vec{x} \given \vec{y}, \vec{\phi}) \frac{p(\vec{y} \given \vec{x}, \vec{\theta})}{q(\vec{x} \given \vec{y}, \vec{\phi})} \dd{\vec{x}} + \int\! q(\vec{x} \given \vec{y}, \vec{\phi}) p(\vec{y} \given \vec{x}, \vec{\theta}) \dd{\vec{x}}  \nonumber \\
			&= -\KL{q(\vec{x} \given \vec{y}, \vec{\phi})}{p(\vec{y} \given \vec{x}, \vec{\theta})} + \E_{q(\vec{x} \given \vec{y}, \vec{\phi})}[p(\vec{y} \given \vec{x}, \vec{\theta})] \eqqcolon \mathcal{L}_\mathrm{ELBO}(\vec{\theta}, \vec{\psi})  \label{eq:elbo}
	\end{align}
	Note that we have introduced an auxiliary parametric distribution \( q(\vec{x} \given \vec{y}, \vec{\phi}) \) which originally leads to the first integral to be an expectation and makes the application of Jensen's inequality possible. We now seek to maximize the \ac{elbo}~\eqref{eq:elbo} \ac{wrt} the variational and generative parameters \(\vec{\phi}\) and \(\vec{\theta}\), respectively. But the gradient \ac{wrt} \(\vec{\phi}\) is problematic, as the usual naive Monte Carlo estimator exhibits very high variance~\cite{kingmaAutoEncodingVariationalBayes2014,paisleyVariationalBayesianInference2012a}.

	They propose a solution to this problem called the \emph{reparametrization trick}. Instead of sampling the latent state \( \tilde{\vec{x}} \sim q(\vec{x} \given \vec{y}, \vec{\phi}) \) directly, a (differentiable) transformation \( \vec{g}_{\varphi}(\vec{\epsilon}, \vec{y}) \) with an auxiliary noise variable \(\vec{\epsilon} \sim p(\vec{\epsilon})\) such that \( \tilde{\vec{x}} = \vec{g}_{\vec{\phi}}(\vec{\epsilon}, \vec{y}) \). By sampling the noise \(\vec{\epsilon}\) and applying the transformation \(\vec{g}_{\vec{\phi}}\), the expectation of some function \( \vec{f}(\vec{x}) \) can be easily calculated with Monte Carlo estimates~\cite{kingmaAutoEncodingVariationalBayes2014} which allows the gradients to flow through the expectation:
	\begin{align*}
		\E_{q_{\vec{\phi}}(\vec{x} \subgiven \vec{y})} [\vec{f}(\vec{x})]
			= \E_{p(\vec{\epsilon})}\big[ \vec{f}(\vec{g}_{\vec{\phi}}(\vec{\epsilon}, \vec{x})) \big]
			\approx \frac{1}{L} \sum_{l = 1}^{L} \vec{f}\big(\vec{g}_{\vec{\phi}}(\vec{\epsilon}^{(l)}, \vec{x})\big),\quad \vec{\epsilon}^{(l)} \sim p(\vec{\epsilon})
	\end{align*}
	This transformation is really simple for a multivariate Gaussian distribution \( \vec{x} \sim \normal(\vec{\mu}, \vec{\sigma}^2 \mat{I}) \) with diagonal covariance, where the reparametrization is just \( \vec{g}_{\vec{\phi}}(\vec{\epsilon}, \vec{y}) = \vec{\mu} + \vec{\sigma} \odot \vec{\epsilon} \) with \( \vec{\epsilon} \sim \normal(\vec{0}, \mat{I}) \) and the element-wise product \( \odot \), also known as the Hadamard product. See~\cite[5]{kingmaAutoEncodingVariationalBayes2014}.

	Shifting to Variational Auto-Encoders, we now represent the auxiliary distribution \( q(\vec{x} \given \vec{y}, \vec{\phi}) \) using a neural network with some prior \( p(\vec{x} \given \vec{\theta}) \), e.g. a standard Gaussian \( p(\vec{x} \given \vec{\theta}) = \normal(\vec{0}, \mat{I}) \) to keep the latents "close to the center". This neural network, called the \emph{amortization network}, produces the mean and the diagonal covariance of \( q(\vec{x} \given \vec{y}, \vec{\phi}) \) and computing the expectation is done using the Monte Carlo sampling method combined with the reparametrization trick. A second neural network is used for decoding the latent dimension, representing the decoding distribution \( p(\vec{y} \given \vec{x}, \vec{\theta}) \). If we assume a Gaussian decoding distribution with constant variance, the right side of the \ac{elbo}~\eqref{eq:elbo} just becomes the squared error between the decoding mean and the input \(\vec{y}\). Typically, we assume a smaller dimension of \(\vec{x}\) (the \emph{latent dimension}) than \( \vec{y} \) (the \emph{observation dimension}) to enforce an encoding of the observations and possibly find smaller representations. Such a network architecture is illustrated in~\autoref{fig:vae}.

	\begin{figure}
		\centering
		\tikzVariationalAutoEncoder
		\caption{Illustration of a Variational Auto-Encoder with the amortization network on the left and the decoder network on the right. Notice that the green-ish neurons in the middle are not "real", deterministic, neurons, but represent the sampling section of the \ac{vae} where the reparametrization takes place. The input and output size (on the left and right, respectively) are the same as we want to reconstruct our original data from the smaller latent state in the middle.}
		\label{fig:vae}
	\end{figure}

	\paragraph{Connection between EM and VAEs}
		As we have seen, the log-likelihood
		\begin{align*}
			\mathcal{L}(\vec{\theta}) = \log \int\! p(\vec{x}, \vec{y} \given \vec{\theta}) \dd{\vec{x}}
		\end{align*}
		gives rise to a lower bound \( \mathcal{L}_\mathrm{EM} \)~\eqref{eq:emLowerBound} and the \ac{elbo} \( \mathcal{L}_\mathrm{ELBO} \)~\eqref{eq:elbo}:
		\begin{align*}
			\mathcal{L}_\mathrm{EM}
				&= \int\! q(\vec{x} \given \vec{y}) \log p(\vec{x}, \vec{y} \given \vec{\theta}) \dd{\vec{x}} - \int\! q(\vec{x} \given \vec{y}) \log q(\vec{x} \given \vec{y}) \dd{\vec{x}}
		\end{align*}
		\begin{align*}
			\mathcal{L}_\mathrm{ELBO}
				&= -\KL{q(\vec{x} \given \vec{y}, \vec{\phi})}{p(\vec{y} \given \vec{x}, \vec{\theta})} + \E_{q(\vec{x} \given \vec{y}, \vec{\phi})}[p(\vec{y} \given \vec{x}, \vec{\theta})] \eqqcolon \mathcal{L}_\mathrm{ELBO}(\vec{\theta}, \vec{\psi})
		\end{align*}
		The lower bounds look quite different, but they are in fact equivalent,
		\begin{align*}
			\mathcal{L}_\mathrm{EM} = \mathcal{L}_\mathrm{ELBO}
		\end{align*}
		as the whole difference is just that the \ac{elbo} uses a factorization \( p(\vec{x}, \vec{y} \given \vec{\theta}) = p(\vec{y} \given \vec{x}, \vec{\theta}) p(\vec{x} \given \vec{\theta}) \) and the \ac{em} lower bound does not.

		The maximization procedures differ in the following way:
		\begin{itemize}
			\item In the \ac{em} algorithm, we separately maximize the components of the lower bound, firstly finding the next auxiliary distribution \(q\) and then maximizing the lower bound \ac{wrt} the parameters.
			\item In \acp{vae}, we use an amortization network to model the auxiliary distribution \(q\) in a parameterized way. Then we maximize the lower bound \ac{wrt} to both the variational and the generative parameters at once.
		\end{itemize}
		An obvious advantage of the \ac{em} algorithm is that we are guaranteed to always rise our lower bound and we will never get worse. But the great catch of the \ac{em} algorithm comes when evaluating the expected complete log-likelihood
		\begin{align*}
			Q(\vec{\theta}) = \int\! p\big(\vec{x} \given \vec{y}, \vec{\theta}^{(n)}\big) \log p(\vec{x}, \vec{y} \given \vec{\theta}) \dd{\vec{\theta}}
		\end{align*}
		to maximize it \ac{wrt} \(\vec{\theta}\): This expectation may not be tractable and thus we may not find update equations for \(\vec{\theta}\). \acp{vae} circumvent this issue by approximating the posterior and performing Monte Carlo sampling of the approximate posterior~\cite{kingmaAutoEncodingVariationalBayes2014}.

		% TODO: Highlight disadvantages of VAEs. If they are so great, why not use them instead of EM? Also, is "always rise out lower bound" really an advantage? Doesn't a regular SGD do the same thing?

		In the next section we will take a look at alternative methods for approximating Gaussian expectations like \(Q\) if the analytic integral is not tractable. That way we can still use an \ac{em} algorithm instead of a \ac{vae} for approximate inference.
	% end
% end

		\section{Cubature Rules}

\subsection{Spherical-Radial Cubature Rule}
	Using the spherical-radial cubature rules~\cite{solinCubatureIntegrationMethods2010}, any Gaussian-like integral
	\begin{equation*}
		\int\! \vec{f}(\vec{x}) \,\normal(\vec{x} \given \vec{\mu}, \mat{\Sigma})
	\end{equation*}
	can be approximated using the cubature points \( \vec{\xi}_i = \sqrt{n} [\vec{1}]_i \):
	\begin{equation}
		\int\! \vec{f}(\vec{x}) \,\normal(\vec{x} \given \vec{\mu}, \mat{\Sigma}) \approx \frac{1}{2n} \sum_{i = 1}^{2n} \vec{f}\big( \sqrt{\mat{\Sigma}} \vec{\xi}_i + \vec{\mu} \big)  \label{eq:sphericalRadialGaussianCubatureRule}
	\end{equation}
	Here, \( \sqrt{\mat{\Sigma}} \) is a matrix such that \( \mat{\Sigma} = \sqrt{\mat{\Sigma}} \sqrt{\mat{\Sigma}} \), \(n\) is the dimension of \(\vec{x}\) and \( [\vec{1}]_i \) are "intersections between the Cartesian axes and the \(n\)-dimensional unit hypersphere."~\cite{solinCubatureIntegrationMethods2010}.
% end

		\section{Notation of Linear Gaussian Dynamical Systems}



\begin{table}[ht]
	\centering
	\begin{tabular}{l|cc}
		\textbf{Element / Definition} & \textbf{Ghahramani}~\cite{ghahramaniParameterEstimationLinear1996} & \textbf{Minka}~\cite{minkaHiddenMarkovModels1999} \\ \hline
		Time                         & \( t \)                         & \( t \)                 \\
		Last Timestep                & \( T \)                         & \( T \)                 \\
		State                        & \( \vec{x}_t \)                 & \( \vec{s}_t \)         \\
		Observations                 & \( \vec{y}_t \)                 & \( \vec{x}_t \)         \\
		State Dynamics Matrix        & \( \mat{A} \)                   & \( \mat{A} \)           \\
		State Noise Covariance       & \( \mat{Q} \)                   & \( \mat{\Gamma} \)      \\
		Observation Matrix           & \( \mat{C} \)                   & \( \mat{C} \)           \\
		Observation Noise Covariance & \( \mat{R} \)                   & \( \mat{\Sigma} \)      \\
		Full Sequence \( \big( \vec{x}_1, \vec{x}_2, \cdots, \vec{x}_T \big) \)
		                             & \( \{ \vec{x} \} \)             & \( \vec{x}_{1:T} \)     \\
		Subsequence \( \big( \vec{x}_{t_0}, \vec{x}_{t_0 + 1}, \cdots, \vec{x}_{t_1} \big) \)
		                             & \( \{ \vec{x} \}_{t_0}^{t_1} \) & \( \vec{x}_{t_0:t_1} \) \\
		Expected Log Likelihood      & \( Q \)                         & \( Q \)                 \\
		Expected State \( \E{\vec{x}_t \biggiven \{ \vec{y} \}} \)
		                             & \( \hat{\vec{x}}_t \)           & \( \mat{\vec{m}}_t \)   \\
		Self-Correlation \( \E{\vec{x}_t \vec{x}_t^T \biggiven \{ \vec{y} \}} \)
		                             & \( \mat{P}_t \)                 & -                       \\
		Cross-Correlation \( \E{\vec{x}_t \vec{x}_{t-1}^T \biggiven \{ \vec{y} \}} \)
		                             & \( \mat{P}_{t, t - 1} \)        & -
	\end{tabular}
	\caption{Notations used for linear Gaussian dynamical systems.}
\end{table}

		\section{Linear Dynamical Systems with Nonlinear Measurements}



Our idea is to replace the linear measurements \( \vec{y}_t \) in an \ac{lgds}
\begin{align*}
	\vec{s}_{t + 1} & = \mat{A} \vec{s}_t + \vec{w}_t \\
	\vec{y}_t       & = \mat{C} \vec{s}_t + \vec{v}_t
\end{align*}
reported in~\cite{ghahramaniParameterEstimationLinear1996} with an arbitrary, possible nonlinear but differentiable function \( \vec{g}_{\vec{\theta}} : \R^k \to \R^p \) with characterizing parameters \( \vec{\theta} \). In all of the following, we keep the parameters implicit for brevity.

The vectors \( \vec{w}_t \) and \( \vec{v}_t \) represent the purely additive Gaussian noise of the system (with zero mean and covariance \( \mat{Q} \) and \( \mat{R} \), respectively). Then the states \( \vec{s}_t \), \( \vec{s}_{t - 1} \) and emissions \( \vec{y}_t \) are jointly Gaussian~\cite{minkaHiddenMarkovModels1999}:
\begin{align*}
	p(\vec{s}_t \given \vec{s}_{t - 1}) & \sim \normal(\mat{A} \vec{s}_{t - 1}, \mat{Q})    \\
	p(\vec{y}_t \given \vec{s}_t)       & \sim \normal\big(\vec{g}(\vec{s}_t), \mat{R}\big)
\end{align*}
This model can also be written as a set of both linear and nonlinear equations:
\begin{align*}
	\vec{s}_{t + 1} & = \mat{A} \vec{s}_t + \vec{w}_t  \\
	\vec{w}_t       & \sim \normal(\vec{0}, \mat{Q})   \\
	\vec{y}_t       & = \vec{g}(\vec{s}_t) + \vec{v}_t \\
	\vec{v}_t       & \sim \normal(\vec{0}, \mat{R})
\end{align*}

In all of the following, we assume to have \(N\) \ac{iid} observation sequences. Let \( \vec{y}_{1:T}^{(n)} \) be the \(n\)-th observation sequence and \( \vec{s}_{1:T}^{(n)} \) the corresponding state sequence. All sequences share a single state dynamics matrix, the same noise covariances and observation function. Let \( \vec{y}_{1:T} \coloneqq \big(\vec{y}_{1:T}^{(1)}, \vec{y}_{1:T}^{(2)}, \cdots, \vec{y}_{1:T}^{(N)}\big) \) and \( \vec{s}_{1:T} \coloneqq \big(\vec{s}_{1:T}^{(1)}, \vec{s}_{1:T}^{(2)}, \cdots, \vec{s}_{1:T}^{(n)}\big) \) be the sequences of all observation and state sequences, respectively. The same goes for \( \vec{y}_t \) and \( \vec{s}_t \).

For a single observation sequence, the complete log-likelihood \( \ln p\Big(\vec{s}_{1:T}^{(n)}, \vec{y}_{1:T}^{(n)}\Big) \) has the form
\begin{align*}
	\ln p\Big(\vec{s}_{1:T}^{(n)}, \vec{y}_{1:T}^{(n)}\Big)
		&= \ln p\Big(\vec{s}_1^{(n)}\Big) \prod_{t = 2}^{T} p\Big(\vec{s}_t^{(n)} \given \vec{s}_{t - 1}^{(n)}\Big) \prod_{t = 1}^{T} p\Big(\vec{y}_t^{(n)} \given \vec{s}_t^{(n)}\Big) \\
		&= \ln p\Big(\vec{s}_1^{(n)}\Big) + \sum_{t = 2}^{T} \ln p\Big(\vec{s}_t^{(n)} \given \vec{s}_{t - 1}^{(n)}\Big) + \sum_{t = 1}^{T} \ln p\Big(\vec{y}_t^{(n)} \given \vec{s}_t^{(n)}\Big) \\
		&= \logGaussianMulti{\vec{s}_1^{(n)}}{\vec{m}_0}{\mat{V}_0}{k} \\
			&\qquad\qquad + \sum_{t = 2}^{T} \bigg( \logGaussianMulti{\vec{s}_t^{(n)}}{\mat{A} \vec{s}_{t - 1}^{(n)}}{\mat{Q}}{k} \bigg) \\
			&\qquad\qquad + \sum_{t = 1}^{T} \bigg( \logGaussianMulti{\vec{y}_t^{(n)}}{\vec{g}\Big(\vec{s}_t^{(n)}\Big)}{\mat{R}}{p} \bigg) \\
		&= -\frac{T(k + p)}{2} \ln(2\pi) - \frac{1}{2} \ln \lvert \mat{V}_0 \rvert - \frac{T - 1}{2} \ln \lvert \mat{Q} \rvert - \frac{T}{2} \ln \lvert \mat{R} \rvert \\
			&\qquad\qquad -\frac{1}{2} \Big( \vec{s}_1^{(n)} - \vec{m}_0 \Big)^T \mat{V}_0^{-1} \Big( \vec{s}_1^{(n)} - \vec{m}_0 \Big) \\
			&\qquad\qquad -\frac{1}{2} \sum_{t = 2}^{T} \Big( \vec{s}_t^{(n)} - \mat{A} \vec{s}_{t - 1}^{(n)} \Big)^T \mat{Q}^{-1} \Big( \vec{s}_t^{(n)} - \mat{A} \vec{s}_{t - 1}^{(n)} \Big) \\
			&\qquad\qquad -\frac{1}{2} \sum_{t = 1}^{T} \Big( \vec{y}_t^{(n)} - \vec{g}\Big(\vec{s}_t^{(n)}\Big) \Big)^T \mat{R}^{-1} \Big( \vec{y}_t^{(n)} - \vec{g}\Big(\vec{s}_t^{(n)}\Big) \Big)
\end{align*}
where \( \vec{m}_0 \) and \( \mat{V}_0 \) describe the initial state mean and covariance. As the observation sequences are independent, we can formulate the joint complete log-likelihood \( \ln p(\vec{s}_{1:T}, \vec{y}_{1:T}) \) as the sum of all subsequent log-likelihoods:
\begin{align*}
	\ln p(\vec{s}_{1:T}, \vec{y}_{1:T})
		&= \ln \prod_{n = 1}^{N} p\Big(\vec{s}_{1:T}^{(n)}, \vec{y}_{1:T}^{(n)}\Big) = \sum_{n = 1}^{N} \ln p\Big(\vec{s}_{1:T}^{(n)}, \vec{y}_{1:T}^{(n)}\Big) \\
		&= -\frac{NT(k + p)}{2} \ln(2\pi) - \frac{N}{2} \ln \lvert \mat{V}_0 \rvert - \frac{N(T - 1)}{2} \ln \lvert \mat{Q} \rvert - \frac{NT}{2} \ln \lvert \mat{R} \rvert \\
			&\qquad\qquad -\frac{1}{2} \sum_{n = 1}^{N} \Big( \vec{s}_1^{(n)} - \vec{m}_0 \Big)^T \mat{V}_0^{-1} \Big( \vec{s}_1^{(n)} - \vec{m}_0 \Big) \\
			&\qquad\qquad -\frac{1}{2} \sum_{n = 1}^{N} \sum_{t = 2}^{T} \Big( \vec{s}_t^{(n)} - \mat{A} \vec{s}_{t - 1}^{(n)} \Big)^T \mat{Q}^{-1} \Big( \vec{s}_t^{(n)} - \mat{A} \vec{s}_{t - 1}^{(n)} \Big) \\
			&\qquad\qquad -\frac{1}{2} \sum_{n = 1}^{N} \sum_{t = 1}^{T} \Big( \vec{y}_t^{(n)} - \vec{g}\Big(\vec{s}_t^{(n)}\Big) \Big)^T \mat{R}^{-1} \Big( \vec{y}_t^{(n)} - \vec{g}\Big(\vec{s}_t^{(n)}\Big) \Big)
\end{align*}

To derive the M-step formulas, we need to maximize the \emph{expected} complete log-likelihood. Therefore, we will now derive the expected log-likelihood
\begin{equation*}
	Q = \E\big[ p(\vec{s}_{1:T}, \vec{y}_{1:T}) \given \vec{y}_{1:T} \big]
\end{equation*}
This expectation depends on three expectations
\begin{align*}
	\hat{\vec{s}}_t^{(n)}    & \coloneqq \E\Big[\vec{s}_t^{(n)} \Biggiven \vec{y}_{1:T}\Big]                                      & \hat{\vec{s}}_t    & \coloneqq \frac{1}{N} \sum_{n = 1}^{N} \hat{\vec{s}}_t^{(n)}          \\
	\mat{P}_t^{(n)}          & \coloneqq \E\bigg[\vec{s}_t^{(n)} \Big(\vec{s}_t^{(n)}\Big)^T \bigggiven \vec{y}_{1:T}\bigg]       & \mat{P}_t          & \coloneqq \frac{1}{N} \sum_{n = 1}^{N} \hat{\mat{P}}_t^{(n)}          \\
	\mat{P}_{t, t - 1}^{(n)} & \coloneqq \E\bigg[\vec{s}_t^{(n)} \Big(\vec{s}_{t - 1}^{(n)}\Big)^T \bigggiven \vec{y}_{1:T}\bigg] & \mat{P}_{t, t - 1} & \coloneqq \frac{1}{N} \sum_{n = 1}^{N} \hat{\mat{P}}_{t, t - 1}^{(n)}
\end{align*}
which form the expected state, self-correlation and cross-correlation, respectively. Additionally we define
\begin{align}
	\hat{\vec{g}}_t^{(n)} & \coloneqq \E\Big[\vec{g}\Big(\vec{s}_t^{(n)}\Big) \Biggiven \vec{y}_{1:T}\Big]    \label{eqn:expectedMeasurement}                                              \\
	\mat{G}_t^{(n)}       & \coloneqq \E\bigg[\vec{g}\Big(\vec{s}_t^{(n)}\Big) \Big(\vec{g}\Big(\vec{s}_t\Big)\Big)^T \bigggiven \vec{y}_{1:T}\bigg]    \label{eqn:expectedMeasurementMat}
\end{align}
to be the expectations of the measurement function \( \vec{g}(\vec{s}_t) \). We will see that evaluating this expectation is not possible in a closed form and has to be approximated, but for now we will be deriving the expected complete log-likelihood dependent on the defined expectations.

For simplicity, let
\begin{align*}
	q_1 &\coloneqq -\frac{NT(k + p)}{2} \ln(2\pi) - \frac{N}{2} \ln \lvert \mat{V}_0 \rvert - \frac{N(T - 1)}{2} \ln \lvert \mat{Q} \rvert - \frac{NT}{2} \ln \lvert \mat{R} \rvert \\
	q_2 &\coloneqq -\frac{1}{2} \sum_{n = 1}^{N} \Big( \vec{s}_1^{(n)} - \vec{m}_0 \Big)^T \mat{V}_0^{-1} \Big( \vec{s}_1^{(n)} - \vec{m}_0 \Big) \\
	q_3 &\coloneqq -\frac{1}{2} \sum_{n = 1}^{N} \sum_{t = 2}^{T} \Big( \vec{s}_t^{(n)} - \mat{A} \vec{s}_{t - 1}^{(n)} \Big)^T \mat{Q}^{-1} \Big( \vec{s}_t^{(n)} - \mat{A} \vec{s}_{t - 1}^{(n)} \Big) \\
	q_4 &\coloneqq -\frac{1}{2} \sum_{n = 1}^{N} \sum_{t = 1}^{T} \Big( \vec{y}_t^{(n)} - \vec{g}\Big(\vec{s}_t^{(n)}\Big) \Big)^T \mat{R}^{-1} \Big( \vec{y}_t^{(n)} - \vec{g}\Big(\vec{s}_t^{(n)}\Big) \Big)
\end{align*}
such that \( \ln p(\vec{s}_{1:T}, \vec{y}_{1:T}) = q_1 + q_2 + q_3 + q_4 \) and, with \(Q_1\), \(Q_2\), \(Q_3\) and \(Q_4\) being the corresponding expectations, \( Q = Q_1 + Q_2 + Q_3 + Q_4 \). Also let \( \vec{y}_t \coloneqq \sum_{n = 1}^{N} \vec{y}_t^{(n)} \).

We start with \(Q_1\):
\begin{equation*}
	Q_1 = \E[q_1 \given \vec{y}_{1:T}] = -\frac{NT(k + p)}{2} \ln(2\pi) - \frac{N}{2} \ln \lvert \mat{V}_0 \rvert - \frac{N(T - 1)}{2} \ln \lvert \mat{Q} \rvert - \frac{NT}{2} \ln \lvert \mat{R} \rvert
\end{equation*}
Then following \(Q_1\):
\begin{align*}
	Q_2
		&= \E[q_2 \given \vec{y}_{1:T}] \\
		&= \E\Bigg[ -\frac{1}{2} \sum_{n = 1}^{N} \Big( \vec{s}_1^{(n)} - \vec{m}_0 \Big)^T \mat{V}_0^{-1} \Big( \vec{s}_1^{(n)} - \vec{m}_0 \Big) \bigggiven \vec{y}_{1:T} \Bigg] \\
		&= -\frac{1}{2} \sum_{n = 1}^{N} \E\Bigg[ \Big( \vec{s}_1^{(n)} - \vec{m}_0 \Big)^T \mat{V}_0^{-1} \Big( \vec{s}_1^{(n)} - \vec{m}_0 \Big) \bigggiven \vec{y}_{1:T} \Bigg] \\
		&= -\frac{1}{2} \sum_{n = 1}^{N} \E\Bigg[ \tr\!\bigg(\!\Big( \vec{s}_1^{(n)} - \vec{m}_0 \Big)^T \mat{V}_0^{-1} \Big( \vec{s}_1^{(n)} - \vec{m}_0 \Big)\!\bigg) \bigggiven \vec{y}_{1:T} \Bigg] \\
		&= -\frac{1}{2} \sum_{n = 1}^{N} \E\Bigg[ \tr\!\bigg(\!\Big( \vec{s}_1^{(n)} - \vec{m}_0 \Big) \Big( \vec{s}_1^{(n)} - \vec{m}_0 \Big)^T \mat{V}_0^{-1} \!\bigg) \bigggiven \vec{y}_{1:T} \Bigg] \\
		&= -\frac{1}{2} \sum_{n = 1}^{N} \E\Bigg[ \tr\!\bigg(\!\Big( \vec{s}_1^{(n)} - \vec{m}_0 \Big) \Big( \Big(\vec{s}_1^{(n)}\Big)^T - \vec{m}_0^T \Big) \mat{V}_0^{-1} \!\bigg) \bigggiven \vec{y}_{1:T} \Bigg] \\
		&= -\frac{1}{2} \sum_{n = 1}^{N} \E\Bigg[ \tr\!\bigg(\! \vec{s}_1^{(n)} \Big(\vec{s}_1^{(n)}\Big)^T - \vec{s}_1^{(n)} \vec{m}_0^T - \vec{m}\Big(\vec{s}_1^{(n)}\Big)^T + \vec{m}_0 \vec{m}_0^T \!\bigg) \mat{V}_0^{-1} \bigggiven \vec{y}_{1:T} \Bigg] \\
		&= -\frac{1}{2} \sum_{n = 1}^{N} \tr\!\bigg(\! \mat{P}_1^{(n)} \mat{V}_0^{-1} - \hat{\vec{s}}_1^{(n)} \vec{m}_0^T \mat{V}_0^{-1} - \vec{m}\Big(\hat{\vec{s}}_1^{(n)}\Big)^T \mat{V}_0^{-1} + \vec{m}_0 \vec{m}_0^T \mat{V}_0^{-1} \!\bigg) \\
		&= -\frac{1}{2} \sum_{n = 1}^{N} \tr\!\Big( \mat{P}_1^{(n)} \mat{V}_0^{-1} \Big) - \tr\!\Big( \hat{\vec{s}}_1^{(n)} \vec{m}_0^T \mat{V}_0^{-1} \Big) - \tr\!\bigg(\! \vec{m}_0 \Big(\hat{\vec{s}}_1^{(n)}\Big)^T \mat{V}_0^{-1} \!\bigg) + \tr\!\Big(\vec{m}_0 \vec{m}_0^T \mat{V}_0^{-1} \Big) \\
		&=  -\frac{N}{2} \tr\!\Big( \mat{P}_1 \mat{V}_0^{-1} \Big) + \frac{N}{2} \tr\!\Big( \hat{\vec{s}}_1 \vec{m}_0^T \mat{V}_0^{-1} \Big) + \frac{N}{2} \tr\!\Big( \vec{m}_0 \hat{\vec{s}}_1^T \mat{V}_0^{-1} \Big) - \frac{N}{2} \tr\!\Big(\vec{m}_0 \vec{m}_0^T \mat{V}_0^{-1} \Big)
\end{align*}
And \(Q_3\):
\begin{align*}
	Q_3
		&= \E[q_3 \given \vec{y}_{1:T}] \\
		&= \E\Bigg[ -\frac{1}{2} \sum_{n = 1}^{N} \sum_{t = 2}^{T} \Big( \vec{s}_t^{(n)} - \mat{A} \vec{s}_{t - 1}^{(n)} \Big)^T \mat{Q}^{-1} \Big( \vec{s}_t^{(n)} - \mat{A} \vec{s}_{t - 1}^{(n)} \Big) \bigggiven \vec{y}_{1:T} \Bigg] \\
		&= -\frac{1}{2} \sum_{n = 1}^{N} \sum_{t = 2}^{T} \E\Bigg[ \Big( \vec{s}_t^{(n)} - \mat{A} \vec{s}_{t - 1}^{(n)} \Big)^T \mat{Q}^{-1} \Big( \vec{s}_t^{(n)} - \mat{A} \vec{s}_{t - 1}^{(n)} \Big) \bigggiven \vec{y}_{1:T} \Bigg] \\
		&= -\frac{1}{2} \sum_{n = 1}^{N} \sum_{t = 2}^{T} \E\Bigg[ \tr\!\bigg(\!\Big( \vec{s}_t^{(n)} - \mat{A} \vec{s}_{t - 1}^{(n)} \Big)^T \mat{Q}^{-1} \Big( \vec{s}_t^{(n)} - \mat{A} \vec{s}_{t - 1}^{(n)} \Big)\!\bigg) \bigggiven \vec{y}_{1:T} \Bigg] \\
		&= -\frac{1}{2} \sum_{n = 1}^{N} \sum_{t = 2}^{T} \E\Bigg[ \tr\!\bigg(\!\Big( \vec{s}_t^{(n)} - \mat{A} \vec{s}_{t - 1}^{(n)} \Big) \Big( \vec{s}_t^{(n)} - \mat{A} \vec{s}_{t - 1}^{(n)} \Big)^T \mat{Q}^{-1} \!\bigg) \bigggiven \vec{y}_{1:T} \Bigg] \\
		&= -\frac{1}{2} \sum_{n = 1}^{N} \sum_{t = 2}^{T} \E\Bigg[ \tr\!\bigg(\!\Big( \vec{s}_t^{(n)} - \mat{A} \vec{s}_{t - 1}^{(n)} \Big) \Big(\big(\vec{s}_t^{(n)}\big)^T - \big(\vec{s}_{t - 1}^{(n)}\big)^T \mat{A}^T \Big) \mat{Q}^{-1} \!\bigg) \bigggiven \vec{y}_{1:T} \Bigg] \\
		&= -\frac{1}{2} \sum_{n = 1}^{N} \sum_{t = 2}^{T} \E\Bigg[ \tr\!\bigg(\!\Big( \vec{s}_t^{(n)} \Big(\vec{s}_t^{(n)}\Big)^T - \vec{s}_t^{(n)} \Big(\vec{s}_{t - 1}^{(n)}\Big)^T \mat{A}^T - \mat{A} \vec{s}_{t - 1}^{(n)} \Big(\vec{s}_t^{(n)}\Big)^T - \mat{A} \vec{s}_{t - 1}^{(n)} \Big(\vec{s}_{t - 1}^{(n)}\Big)^T \mat{A}^T \Big) \mat{Q}^{-1} \!\bigg) \bigggiven \vec{y}_{1:T} \Bigg] \\
		&= -\frac{1}{2} \sum_{n = 1}^{N} \sum_{t = 2}^{T} \tr\!\Big( \mat{P}_t^{(n)} \mat{Q}^{-1} - \mat{P}_{t, t - 1}^{(n)} \mat{A}^T \mat{Q}^{-1} - \mat{A} \underbrace{\mat{P}_{t - 1, t}^{(n)}}_{=\, \mat{P}_{t, t - 1}^{(n)}} \mat{Q}^{-1} - \mat{A} \mat{P}_{t - 1}^{(n)} \mat{A}^T \mat{Q}^{-1} \Big) \\
		&= -\frac{1}{2} \sum_{n = 1}^{N} \sum_{t = 2}^{T} \tr\!\Big( \mat{P}_t^{(n)} \mat{Q}^{-1} \Big) - \tr\!\Big( \mat{P}_{t, t - 1}^{(n)} \mat{A}^T \mat{Q}^{-1} \Big) - \tr\!\Big( \mat{A} \mat{P}_{t, t - 1}^{(n)} \mat{Q}^{-1} \Big) - \tr\!\Big( \mat{A} \mat{P}_{t - 1}^{(n)} \mat{A}^T \mat{Q}^{-1} \Big) \\
		&= -\frac{N}{2} \sum_{t = 2}^{T} \tr\!\Big( \mat{P}_t \mat{Q}^{-1} \Big) - \tr\!\Big( \mat{P}_{t, t - 1} \mat{A}^T \mat{Q}^{-1} \Big) - \tr\!\Big( \mat{A} \mat{P}_{t, t - 1} \mat{Q}^{-1} \Big) - \tr\!\Big( \mat{A} \mat{P}_{t - 1} \mat{A}^T \mat{Q}^{-1} \Big)
\end{align*}
Finally, we drive \(Q_4\). This is the one that depends on the non-closed-form expectations~\ref{eqn:expectedMeasurement} and~\ref{eqn:expectedMeasurementMat}:
\begin{align*}
	Q_4
		&= \E[q_4 \given \vec{y}_{1:T}] \\
		&= \E\Bigg[ -\frac{1}{2} \sum_{n = 1}^{N} \sum_{t = 1}^{T} \Big( \vec{y}_t^{(n)} - \vec{g}\Big(\vec{s}_t^{(n)}\Big) \Big)^T \mat{R}^{-1} \Big( \vec{y}_t^{(n)} - \vec{g}\Big(\vec{s}_t^{(n)}\Big) \Big) \bigggiven \vec{y}_{1:T} \Bigg] \\
		&= -\frac{1}{2} \sum_{n = 1}^{N} \sum_{t = 1}^{T} \E\Bigg[ \Big( \vec{y}_t^{(n)} - \vec{g}\Big(\vec{s}_t^{(n)}\Big) \Big)^T \mat{R}^{-1} \Big( \vec{y}_t^{(n)} - \vec{g}\Big(\vec{s}_t^{(n)}\Big) \Big) \bigggiven \vec{y}_{1:T} \Bigg] \\
		&= -\frac{1}{2} \sum_{n = 1}^{N} \sum_{t = 1}^{T} \E\Bigg[ \tr\!\bigg(\!\Big( \vec{y}_t^{(n)} - \vec{g}\Big(\vec{s}_t^{(n)}\Big) \Big)^T \mat{R}^{-1} \Big( \vec{y}_t^{(n)} - \vec{g}\Big(\vec{s}_t^{(n)}\Big) \Big)\!\bigg) \bigggiven \vec{y}_{1:T} \Bigg] \\
		&= -\frac{1}{2} \sum_{n = 1}^{N} \sum_{t = 1}^{T} \E\Bigg[ \tr\!\bigg(\!\Big( \vec{y}_t^{(n)} - \vec{g}\Big(\vec{s}_t^{(n)}\Big) \Big) \Big( \vec{y}_t^{(n)} - \vec{g}\Big(\vec{s}_t^{(n)}\Big) \Big)^T \mat{R}^{-1} \!\bigg) \bigggiven \vec{y}_{1:T} \Bigg] \\
		&= -\frac{1}{2} \sum_{n = 1}^{N} \sum_{t = 1}^{T} \E\Bigg[ \tr\!\bigg(\!\Big( \vec{y}_t^{(n)} - \vec{g}\Big(\vec{s}_t^{(n)}\Big) \Big) \bigg(\!\Big(\vec{y}_t^{(n)}\Big)^T - \Big(\vec{g}\Big(\vec{s}_t^{(n)}\Big)\Big)^T \bigg) \mat{R}^{-1} \!\bigg) \bigggiven \vec{y}_{1:T} \Bigg] \\
		&= -\frac{1}{2} \sum_{n = 1}^{N} \sum_{t = 1}^{T} \E\Bigg[ \tr\!\bigg(\!\bigg(\! \vec{y}_t^{(n)} \Big(\vec{y}_t^{(n)}\Big)^T - \vec{y}_t^{(n)} \Big(\vec{g}\Big(\vec{s}_t^{(n)}\Big)\Big)^T - \vec{g}\Big(\vec{s}_t^{(n)}\Big) \Big(\vec{y}_t^{(n)}\Big)^T + \vec{g}\Big(\vec{s}_t^{(n)}\Big) \Big(\vec{g}\Big(\vec{s}_t^{(n)}\Big)\Big)^T \bigg) \mat{R}^{-1} \!\bigg) \bigggiven \vec{y}_{1:T} \Bigg] \\
		&= -\frac{1}{2} \sum_{n = 1}^{N} \sum_{t = 1}^{T} \tr\!\bigg(\! \vec{y}_t^{(n)} \Big(\vec{y}_t^{(n)}\Big)^T \mat{R}^{-1} - \vec{y}_t^{(n)} \Big(\hat{\vec{g}}_t^{(n)}\Big)^T \mat{R}^{-1} - \hat{\vec{g}}_t^{(n)} \Big(\vec{y}_t^{(n)}\Big)^T \mat{R}^{-1} + \mat{G}_t^{(n)} \mat{R}^{-1} \!\bigg) \\
		&= -\frac{1}{2} \sum_{n = 1}^{N} \sum_{t = 1}^{T} \tr\!\bigg(\! \vec{y}_t^{(n)} \Big(\vec{y}_t^{(n)}\Big)^T \mat{R}^{-1} \!\bigg) - \tr\!\bigg(\! \vec{y}_t^{(n)} \Big(\hat{\vec{g}}_t^{(n)}\Big)^T \mat{R}^{-1} \!\bigg) - \tr\!\bigg(\! \hat{\vec{g}}_t^{(n)} \Big(\vec{y}_t^{(n)}\Big)^T \mat{R}^{-1} \!\bigg) + \tr\!\Big( \mat{G}_t^{(n)} \mat{R}^{-1} \Big) \\
\end{align*}

Putting it all together, the expected complete log-likelihood across all observation sequences is given as
\begin{align*}
	Q
		&= Q_1 + Q_2 + Q_3 + Q_4 \\
		&= -\frac{NT(k + p)}{2} \ln(2\pi) - \frac{N}{2} \ln \lvert \mat{V}_0 \rvert - \frac{N(T - 1)}{2} \ln \lvert \mat{Q} \rvert - \frac{NT}{2} \ln \lvert \mat{R} \rvert \\
			&\qquad\qquad -\frac{N}{2} \tr\!\Big( \mat{P}_1 \mat{V}_0^{-1} \Big) + \frac{N}{2} \tr\!\Big( \hat{\vec{s}}_1 \vec{m}_0^T \mat{V}_0^{-1} \Big) + \frac{N}{2} \tr\!\Big( \vec{m}_0 \hat{\vec{s}}_1^T \mat{V}_0^{-1} \Big) - \frac{N}{2} \tr\!\Big(\vec{m}_0 \vec{m}_0^T \mat{V}_0^{-1} \Big) \\
			&\qquad\qquad -\frac{N}{2} \sum_{t = 2}^{T} \tr\!\Big( \mat{P}_t \mat{Q}^{-1} \Big) - \tr\!\Big( \mat{P}_{t, t - 1} \mat{A}^T \mat{Q}^{-1} \Big) - \tr\!\Big( \mat{A} \mat{P}_{t, t - 1} \mat{Q}^{-1} \Big) - \tr\!\Big( \mat{A} \mat{P}_{t - 1} \mat{A}^T \mat{Q}^{-1} \Big) \\
			&\qquad\qquad -\frac{1}{2} \sum_{n = 1}^{N} \sum_{t = 1}^{T} \tr\!\bigg(\! \vec{y}_t^{(n)} \Big(\vec{y}_t^{(n)}\Big)^T \mat{R}^{-1} \!\bigg) - \tr\!\bigg(\! \vec{y}_t^{(n)} \Big(\hat{\vec{g}}_t^{(n)}\Big)^T \mat{R}^{-1} \!\bigg) - \tr\!\bigg(\! \hat{\vec{g}}_t^{(n)} \Big(\vec{y}_t^{(n)}\Big)^T \mat{R}^{-1} \!\bigg) + \tr\!\Big( \mat{G}_t^{(n)} \mat{R}^{-1} \Big)
\end{align*}
where \( Q_1 \), \( Q_2 \), \( Q_3 \) and \( Q_4 \) are functions of the parameters \( \mat{A} \), \( \mat{Q} \), \( \vec{\theta} \), \( \mat{R} \), \( \vec{m}_0 \) and \( \mat{V}_0 \):
\begin{align*}
	Q_1 & = Q_1(\mat{V}_0, \mat{Q}, \mat{R}) \\
	Q_2 & = Q_2(\vec{m}_0, \mat{V}_0)        \\
	Q_3 & = Q_3(\mat{A}, \mat{Q})            \\
	Q_4 & = Q_4(\vec{\theta}, \mat{R})
\end{align*}

% TODO: Evaluate \hat{g} and G.

Now we have everything together to derive the M-step equations by maximizing \(Q\).


\subsection{M-Step}
	To maximize \(Q\) \ac{wrt} all parameters, that is
	\begin{itemize}
		\item state dynamics matrix \(\mat{A}\),
		\item state noise covariance \(\mat{Q}\),
		\item measurement function parameters \(\vec{\theta}\),
		\item measurement noise covariance \(\mat{R}\),
		\item initial state mean \(\vec{m}_0\) and
		\item initial state covariance \(\mat{V}_0\),
	\end{itemize}
	we have to take the derivatives \ac{wrt} to all the above parameters and set them to zero.
	
	To maximize \(Q\) \ac{wrt} all parameters, we have to take the derivatives \ac{wrt} the parameters and set them to zero.
	\begin{itemize}
		\item State dynamics matrix \(\mat{A}\):
	\end{itemize}
	\begin{align}
		&&\pdv{Q}{\mat{A}}
			&= \pdv{Q_1}{\mat{A}} + \pdv{Q_2}{\mat{A}} + \pdv{Q_3}{\mat{A}} + \pdv{Q_4}{\mat{A}} = \pdv{Q_3}{\mat{A}} & \nonumber \\
		&&	&= -\frac{N}{2} \sum_{t = 2}^{T} -\mat{Q}^{-1} \mat{P}_{t, t - 1} - \mat{Q}^{-T} \mat{P}_{t, t - 1}^T + \mat{Q}^{-T} \mat{A} \mat{P}_{t - 1}^T + \mat{Q}^{-1} \mat{A} \mat{P}_{t - 1} & \nonumber \\
		&&	&\oversetfootnotemark{=} -N \sum_{t = 2}^{T} -\mat{Q}^{-1} \mat{P}_{t, t - 1} + \mat{Q}^{-1} \mat{A} \mat{P}_{t - 1} \overset{!}{=} \mat{0} & \nonumber \\
		\implies && \mat{A}^\new &= \Bigg(\! \sum_{t = 2}^{T} \mat{P}_{t, t - 1} \!\Bigg) \Bigg(\! \sum_{t = 2}^{T} \mat{P}_{t - 1} \!\Bigg)^{-1} & \label{eqn:stateDynamicsMatrix}
	\end{align}
	\footnotetext{Covariance and correlation matrices (\( \mat{Q} \), \( \mat{R} \), \( \mat{P}_t \), \( \mat{P}_{t, t - 1} \)) are symmetric by definition.}
	
	\begin{itemize}
		\item State noise covariance \(\mat{Q}\): \\ Instead of maximizing \ac{wrt} \(\mat{Q}\), we can also minimize \ac{wrt} \(\mat{Q}^{-1}\) which has the same effect.
	\end{itemize}
	\begin{align}
		&&\pdv{Q}{\mat{Q}^{-1}}
			&= \pdv{Q_1}{\mat{Q}^{-1}} + \pdv{Q_2}{\mat{Q}^{-1}} + \pdv{Q_3}{\mat{Q}^{-1}} + \pdv{Q_4}{\mat{Q}^{-1}} = \pdv{Q_1}{\mat{Q}^{-1}} + \pdv{Q_3}{\mat{Q}^{-1}} & \nonumber \\
		&&	&\oversetfootnotemark{=} \frac{N(T - 1)}{2} \mat{Q} - \frac{1}{2} \sum_{t = 2}^{T} \mat{P}_t^T - \mat{A} \mat{P}_{t, t - 1}^T - \mat{P}_{t, t - 1}^T \mat{A}^T + \mat{A} \mat{P}_{t - 1}^T \mat{A}^T & \nonumber \\
		&&	&= \frac{N(T - 1)}{2} \mat{Q} - \frac{N}{2} \sum_{t = 2}^{T} \mat{P}_t - \mat{A} \mat{P}_{t, t - 1} - \mat{P}_{t, t - 1} \mat{A}^T + \mat{A} \mat{P}_{t - 1} \mat{A}^T & \nonumber \\
		&&	&= \frac{N(T - 1)}{2} \mat{Q} - \frac{N}{2} \sum_{t = 2}^{T} \mat{P}_t + \frac{N}{2} \sum_{t = 2}^{T} \mat{A} \mat{P}_{t, t - 1} + \frac{N}{2} \sum_{t = 2}^{T} \mat{P}_{t, t - 1} \mat{A}^T - \frac{N}{2} \sum_{t = 2}^{T} \mat{A} \mat{P}_{t - 1} \mat{A}^T & \nonumber \\
		&&	&= \frac{N(T - 1)}{2} \mat{Q} - \frac{N}{2} \sum_{t = 2}^{T} \mat{P}_t + \frac{N}{2} \mat{A}^\new \sum_{t = 2}^{T} \mat{P}_{t, t - 1} + \frac{N}{2} \Bigg(\! \sum_{t = 2}^{T} \mat{P}_{t, t - 1} \!\Bigg) \big(\mat{A}^\new\big)^T & \nonumber \\
			&&	&\qquad\qquad - \frac{N}{2} \mat{A}^\new \Bigg(\! \sum_{t = 2}^{T} \mat{P}_{t - 1} \!\Bigg) \big(\mat{A}^\new\big)^T & \nonumber \\
		&&	&= \frac{N(T - 1)}{2} \mat{Q} - \frac{N}{2} \sum_{t = 2}^{T} \mat{P}_t + \frac{N}{2} \mat{A}^\new \sum_{t = 2}^{T} \mat{P}_{t, t - 1} + \frac{N}{2} \Bigg(\! \sum_{t = 2}^{T} \mat{P}_{t, t - 1} \!\Bigg) \Bigg(\! \sum_{t = 2}^{T} \mat{P}_{t - 1} \!\Bigg)^{-T} \Bigg(\! \sum_{t = 2}^{T} \mat{P}_{t, t - 1} \!\Bigg)^T & \nonumber \\
			&&	&\qquad\qquad - \frac{N}{2} \Bigg(\! \sum_{t = 2}^{T} \mat{P}_{t, t - 1} \!\Bigg) \Bigg(\! \sum_{t = 2}^{T} \mat{P}_{t - 1} \!\Bigg)^{-1} \Bigg(\! \sum_{t = 2}^{T} \mat{P}_{t - 1} \!\Bigg) \Bigg(\! \sum_{t = 2}^{T} \mat{P}_{t - 1} \!\Bigg)^{-T} \Bigg(\! \sum_{t = 2}^{T} \mat{P}_{t, t - 1} \!\Bigg)^T & \nonumber \\
		&&	&= \frac{N(T - 1)}{2} \mat{Q} - \frac{N}{2} \sum_{t = 2}^{T} \mat{P}_t + \frac{N}{2} \mat{A}^\new \sum_{t = 2}^{T} \mat{P}_{t, t - 1} + \frac{N}{2} \Bigg(\! \sum_{t = 2}^{T} \mat{P}_{t, t - 1} \!\Bigg) \Bigg(\! \sum_{t = 2}^{T} \mat{P}_{t - 1} \!\Bigg)^{-1} \Bigg(\! \sum_{t = 2}^{T} \mat{P}_{t, t - 1} \!\Bigg) & \nonumber \\
			&&	&\qquad\qquad - \frac{N}{2} \Bigg(\! \sum_{t = 2}^{T} \mat{P}_{t, t - 1} \!\Bigg) \Bigg(\! \sum_{t = 2}^{T} \mat{P}_{t - 1} \!\Bigg)^{-1} \cancel{\Bigg(\! \sum_{t = 2}^{T} \mat{P}_{t - 1} \!\Bigg) \Bigg(\! \sum_{t = 2}^{T} \mat{P}_{t - 1} \!\Bigg)^{-1}} \Bigg(\! \sum_{t = 2}^{T} \mat{P}_{t, t - 1} \!\Bigg) & \nonumber \\
		&&	&= \frac{N(T - 1)}{2} \mat{Q} - \frac{N}{2} \Bigg(\! \sum_{t = 2}^{T} \mat{P}_t - \mat{A}^\new \sum_{t = 2}^{T} \mat{P}_{t, t - 1} \!\Bigg) \overset{!}{=} \mat{0} & \nonumber \\
		\implies && \mat{Q}^\new &= \frac{1}{T - 1} \Bigg(\! \sum_{t = 2}^{T} \mat{P}_t - \mat{A}^\new \sum_{t = 2}^{T} \mat{P}_{t, t - 1} \!\Bigg) & \label{eqn:stateNoiseCovariance}
	\end{align}
	\footnotetext{Note that \( \pdv{\mat{Q}^{-1}} \ln \det \mat{Q} = \pdv{\mat{Q}^{-1}} \ln \det \big(\mat{Q}^{-1}\big)^{-1} = \bigg( \pdv{\det (\mat{Q}^{-1})^{-1}} \ln \det \big(\mat{Q}^{-1}\big)^{-1} \bigg) \bigg( \pdv{\mat{Q}^{-1}} \det \big(\mat{Q}^{-1}\big)^{-1} \bigg) \overset{\text{cov. matrix}}{=} -\mat{Q} \)}

	\begin{itemize}
		\item Initial state mean \(\vec{m}_0\):
	\end{itemize}
	\begin{align}
		&&\pdv{Q}{\vec{m}_0}
			&= \pdv{Q_1}{\vec{m}_0} + \pdv{Q_2}{\vec{m}_0} + \pdv{Q_3}{\vec{m}_0} + \pdv{Q_4}{\vec{m}_0} = \pdv{Q_2}{\vec{m}_0} & \nonumber \\
		&&	&= \frac{N}{2} \mat{V}_0^{-1} \hat{\vec{s}}_1 + \frac{N}{2} \mat{V}_0^{-T} \hat{\vec{s}}_1 - \frac{N}{2} \big( \mat{V}_0^{-1} \vec{m}_0 + \mat{V}_0^{-T} \vec{m}_0 \big) & \nonumber \\
		&&	&= N \mat{V}_0^{-1} \hat{\vec{s}}_1 - N \mat{V}_0^{-1} \vec{m}_0 & \nonumber \\
		&&	&= N \mat{V}_0^{-1} \big( \hat{\vec{s}}_1 - \vec{m}_0 \big) \overset{!}{=} \vec{0} & \nonumber \\
		\implies && \vec{m}_0^\new &= \hat{\vec{s}}_1 = \frac{1}{N} \sum_{n = 0}^{N} \hat{\vec{s}}_1^{(n)} & \label{eqn:initialStateMean}
	\end{align}
	
	\begin{itemize}
		\item Initial state covariance \(\mat{V}_0\):
	\end{itemize}
	\begin{align}
		&&\pdv{Q}{\mat{V}_0^{-1}}
			&= \pdv{Q_1}{\mat{V}_0^{-1}} + \pdv{Q_2}{\mat{V}_0^{-1}} + \pdv{Q_3}{\mat{V}_0^{-1}} + \pdv{Q_4}{\mat{V}_0^{-1}} = \pdv{Q_1}{\mat{V}_0^{-1}} + \pdv{Q_2}{\mat{V}_0^{-1}} & \nonumber \\
		&&	&= \frac{N}{2} \mat{V}_0 - \frac{N}{2} \mat{P}_1^T + \frac{N}{2} \vec{m}_0 \hat{\vec{s}}_1^T + \frac{N}{2} \hat{\vec{s}}_1 \vec{m}_0^T - \frac{N}{2} \vec{m}_0 \vec{m}_0^T & \nonumber \\
		&&	&= \frac{N}{2} \mat{V}_0 - \frac{N}{2} \mat{P}_1^T + \frac{N}{2} \hat{\vec{s}}_1 \hat{\vec{s}}_1^T + \frac{N}{2} \hat{\vec{s}}_1 \hat{\vec{s}}_1^T - \frac{N}{2} \hat{\vec{s}}_1 \hat{\vec{s}}_1^T & \nonumber \\
		&&	&= \frac{N}{2} \mat{V}_0 - \frac{N}{2} \mat{P}_1^T + \frac{N}{2} \hat{\vec{s}}_1 \hat{\vec{s}}_1^T \overset{!}{=} \mat{0} & \nonumber \\
		\implies && \mat{V}_0^\new &= \mat{P}_1^T - \hat{\vec{s}}_1 \hat{\vec{s}}_1^T & \label{eqn:initialStateCovariance}
	\end{align}
	
	\begin{itemize}
		\item Measurement noise covariance \(\mat{R}\):
	\end{itemize}
	\begin{align}
		&&\pdv{Q}{\mat{R}^{-1}}
			&= \pdv{Q_1}{\mat{R}^{-1}} + \pdv{Q_2}{\mat{R}^{-1}} + \pdv{Q_3}{\mat{R}^{-1}} + \pdv{Q_4}{\mat{R}^{-1}} = \pdv{Q_1}{\mat{R}^{-1}} + \pdv{Q_4}{\mat{R}^{-1}} & \nonumber \\
		&&	&= \frac{NT}{2} \mat{R} - \frac{1}{2} \sum_{n = 1}^{N} \sum_{t = 1}^{T} \vec{y}_t^{(n)} \Big(\vec{y}_t^{(n)}\Big)^T - \hat{\vec{g}}_t^{(n)} \Big(\vec{y}_t^{(n)}\Big)^T - \vec{y}_t^{(n)} \Big(\hat{\vec{g}}_t^{(n)}\Big)^T + \Big(\mat{G}_t^{(n)}\Big)^T \overset{!}{=} \mat{0} & \nonumber \\
		\implies && \mat{R} &= \frac{1}{NT} \sum_{n = 1}^{N} \sum_{t = 1}^{T} \vec{y}_t^{(n)} \Big(\vec{y}_t^{(n)}\Big)^T - \hat{\vec{g}}_t^{(n)} \Big(\vec{y}_t^{(n)}\Big)^T - \vec{y}_t^{(n)} \Big(\hat{\vec{g}}_t^{(n)}\Big)^T + \Big(\mat{G}_t^{(n)}\Big)^T & \label{eqn:measurementNoiseCovariance}
	\end{align}
	
	The equations for the state dynamics matrix~\ref{eqn:stateDynamicsMatrix}, the state noise covariance~\ref{eqn:stateNoiseCovariance}, the initial state mean~\ref{eqn:initialStateMean} and the initial state covariance~\ref{eqn:initialStateCovariance} match exactly the equations given in~\cite{ghahramaniParameterEstimationLinear1996}. The equation for the measurement noise covariance~\ref{eqn:measurementNoiseCovariance} differs from the one given in~\cite{ghahramaniParameterEstimationLinear1996} as we are having nonlinear measurements and thus we cannot write the covariance in such a compact form.
	
	As said before, the expectations~\ref{eqn:expectedMeasurement} and~\ref{eqn:expectedMeasurementMat} are problematic as we cannot evaluate the integrals
	\begin{align*}
		\hat{\vec{g}}_t^{(n)} &= \int\! \vec{g}\Big(\vec{s}_t^{(n)}\Big) p\Big(\vec{s}_t^{(n)} \Biggiven \vec{y}_{1:T}\Big) \dd{\vec{s}_{1:T}^{(n)}} \\
		\mat{G}_t^{(n)}       &= \int\! \vec{g}\Big(\vec{s}_t^{(n)}\Big) \Big(\vec{g}\Big(\vec{s}_t^{(n)}\Big)\Big)^T p\Big(\vec{s}_t^{(n)} \Biggiven \vec{y}_{1:T} \Big) \dd{\vec{s}_{1:T}^{(n)}}
	\end{align*}
	in closed form (the function \( \vec{g}(\cdot) \) is nonlinear). Note that the posterior distribution is Gaussian
	\begin{equation*} % TODO: Is the posterior covariance dependent on the observation sequence?
		p\Big(\vec{s}_t^{(n)} \Biggiven \vec{y}_{1:T}\Big) \sim \normal\Big(\hat{\vec{m}}_t^{(n)}, \hat{\mat{V}}_t\Big)
	\end{equation*}
	where \( \hat{\vec{m}}_t^{(n)} \) and \( \hat{\mat{V}}_t \) form the posterior mean and covariance, respectively. These are calculated in the E-step (formulas~\ref{eqn:posteriorMean} and~\ref{eqn:posteriorCov} \todo{references}). Due to this Gaussian behavior, we can approximate the integrals using the spherical-radial cubature rule~\cite{solinCubatureIntegrationMethods2010} given in equation~\ref{eqn:sphericalRadialGaussianCubatureRule} (with \( \vec{\xi}_i = \sqrt{k} [\vec{1}]_i \)):
	\begin{align}
		\int\! \vec{g}\Big(\vec{s}_t^{(n)}\Big) p\Big(\vec{s}_t^{(n)} \Biggiven \vec{y}_{1:T}\Big) \dd{\vec{s}_{1:T}^{(n)}}
			&\approx \sum_{i = 1}^{2k} \vec{g}\bigg(\sqrt{\hat{\mat{V}}_t} \vec{\xi}_i + \hat{\vec{m}}_t^{(n)}\bigg)  \label{eqn:lgsCubatureG} \\
		\int\! \vec{g}\Big(\vec{s}_t^{(n)}\Big) \Big(\vec{g}\Big(\vec{s}_t^{(n)}\Big)\Big)^T p\Big(\vec{s}_t^{(n)} \Biggiven \vec{y}_{1:T} \Big)
			&\approx \sum_{i = 1}^{2k} \vec{g}\bigg(\sqrt{\hat{\mat{V}}_t} \vec{\xi}_i + \hat{\vec{m}}_t^{(n)}\bigg) \bigg( \vec{g}\bigg(\sqrt{\hat{\mat{V}}_t} \vec{\xi}_i + \hat{\vec{m}}_t^{(n)}\bigg) \bigg)^T  \label{eqn:lgsCubatureGG}
	\end{align}
	These approximations can be inserted into the closed-form calculation of \(\mat{R}\) given in equation~\ref{eqn:measurementNoiseCovariance} and can be easily differentiated \ac{wrt} \(\vec{\theta}\), \ac{eg} by using a neural network for approximating \(\vec{g}(\cdot)\) and an autograd engine like PyTorch~\cite{paszkePyTorchImperativeStyle2019}. Then we can use a numerical optimizer (\ac{eg} Adam~\cite{kingmaAdamMethodStochastic2017} or Adagrad~\cite{duchiAdaptiveSubgradientMethods2011}) and take one optimization step in each invocation of the M-step. That way we do not maximize \(Q\) in every M-step, but increase \(Q\) such that the convergence properties will still hold~\cite{moonExpectationmaximizationAlgorithm1996}.
	
	We are now able to perform the M-step with the closed-form maximizations given in equations~\ref{eqn:stateDynamicsMatrix},~\ref{eqn:stateNoiseCovariance},~\ref{eqn:initialStateMean},~\ref{eqn:initialStateCovariance} and~\ref{eqn:measurementNoiseCovariance} and using a numerical approach for the measurement parameters \(\vec{\theta}\) by applying cubature approximations given in~\ref{eqn:lgsCubatureG} and~\ref{eqn:lgsCubatureGG}. For completeness, we summarize all of them here:
	\begin{align*}
		\mat{A}^\new &= \Bigg(\! \sum_{t = 2}^{T} \mat{P}_{t, t - 1} \!\Bigg) \Bigg(\! \sum_{t = 2}^{T} \mat{P}_{t - 1} \!\Bigg)^{-1} \\
		\mat{Q}^\new &= \frac{1}{T - 1} \Bigg(\! \sum_{t = 2}^{T} \mat{P}_t - \mat{A}^\new \sum_{t = 2}^{T} \mat{P}_{t, t - 1} \!\Bigg) \\
		\vec{m}_0^\new &= \hat{\vec{s}}_1 = \frac{1}{N} \sum_{n = 0}^{N} \hat{\vec{s}}_1^{(n)} \\
		\mat{V}_0^\new &= \mat{P}_1^T - \hat{\vec{s}}_1 \hat{\vec{s}}_1^T \\
		\mat{R} &= \frac{1}{NT} \sum_{n = 1}^{N} \sum_{t = 1}^{T} \vec{y}_t^{(n)} \Big(\vec{y}_t^{(n)}\Big)^T - \hat{\vec{g}}_t^{(n)} \Big(\vec{y}_t^{(n)}\Big)^T - \vec{y}_t^{(n)} \Big(\hat{\vec{g}}_t^{(n)}\Big)^T + \Big(\mat{G}_t^{(n)}\Big)^T \\
		\hat{\vec{g}}_t^{(n)} &\approx \sum_{i = 1}^{2k} \vec{g}\bigg(\sqrt{\hat{\mat{V}}_t} \vec{\xi}_i + \hat{\vec{m}}_t^{(n)}\bigg) \\
		\mat{G}_t^{(n)} &\approx \sum_{i = 1}^{2k} \vec{g}\bigg(\sqrt{\hat{\mat{V}}_t} \vec{\xi}_i + \hat{\vec{m}}_t^{(n)}\bigg) \bigg( \vec{g}\bigg(\sqrt{\hat{\mat{V}}_t} \vec{\xi}_i + \hat{\vec{m}}_t^{(n)}\bigg) \bigg)^T
	\end{align*}
% end





















	% end
\end{document}
