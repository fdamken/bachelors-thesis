\makeatletter

% Make LOF a section.
\renewcommand\listoffigures{
	\section*{\listfigurename}
	\@mkboth{\MakeUppercase\listfigurename}{\MakeUppercase\listfigurename}
	\@starttoc{lof}
}
% Make LOT a section.
\renewcommand\listoftables{
	\section*{\listtablename}
	\@mkboth{\MakeUppercase\listtablename}{\MakeUppercase\listtablename}
	\@starttoc{lot}
}

\makeatother



% Style definitions.

% Description-list styling.
\SetLabelAlign{parright}{\parbox[t]{\labelwidth}{\raggedleft#1}}
\setlist[description]{style = multiline, leftmargin = 4cm, align = parright}

\MakeOuterQuote{"}

% Captions should be centered.
\captionsetup{justification=centering}

% Matrix/Vector notation.
\renewcommand{\vec}[1]{\boldsymbol{\mathrm{#1}}}
\newcommand{\mat}[1]{\boldsymbol{\mathrm{#1}}}
% Shorthands.
\renewcommand{\C}{\mathbb{C}}
\newcommand{\R}{\mathbb{R}}
\newcommand{\E}{\mathbb{E}}
\newcommand{\normal}{\mathcal{N}}
\newcommand{\gaussianMulti}[4]{\frac{1}{\left(2\pi\right)^{#4/2} \cdot \lvert #3 \rvert^{1/2}} \exp \bigg\{\! -\frac{1}{2} \left(#1 - #2\right)^T #3^{-1} \left(#1 - #2\right) \bigg\}}
\newcommand{\logGaussianMulti}[4]{-\frac{1}{2} \ln \lvert #3 \rvert - \frac{#4}{2} \ln\left(2\pi\right) - \frac{1}{2} \left(#1 - #2\right)^T #3^{-1} \left(#1 - #2\right)}
\newcommand{\subgiven}{\vert}
\newcommand{\given}{\,\vert\,}
\newcommand{\biggiven}{\,\big\vert\,}
\newcommand{\Biggiven}{\,\Big\vert\,}
\newcommand{\bigggiven}{\,\bigg\vert\,}
\newcommand{\Bigggiven}{\,\Bigg\vert\,}
\newcommand{\new}{\mathrm{new}}
\newcommand{\KL}[2]{D_\mathrm{KL}\big( #1 \,\Vert\, #2 \big)}
\newcommand{\SRC}{\mathit{SRC}}
\newcommand{\rangedots}{\,\cdots\!}
% Math operators.
\DeclareMathOperator{\const}{const}
\DeclareMathOperator{\Cov}{Cov}
\DeclareMathOperator{\diag}{diag}

\newcommand{\oversetfootnotemark}[1]{\stepcounter{footnote} \overset{\mathclap{(\thefootnote)}}{#1}}
\let\realfootnote\footnote
\let\realfootnotetext\footnotetext
\renewcommand{\footnote}[1]{\realfootnote{\, #1}}
\renewcommand{\footnotetext}[1]{\realfootnotetext{\, #1}}
\newcommand{\doublefootnotetext}[2]{\addtocounter{footnote}{-1} \footnotetext{#1} \stepcounter{footnote} \footnotetext{#2}}
\newcommand{\triplefootnotetext}[3]{\addtocounter{footnote}{-2} \footnotetext{#1} \stepcounter{footnote} \footnotetext{#2} \stepcounter{footnote} \footnotetext{#3}}

\renewcommand{\arraystretch}{1.5}
\parindent0pt

% Definitions, Theorems and Lemmata.
\newtheorem{theorem}{Theorem}[chapter]
\newtheorem{definition}[theorem]{Definition}
\newtheorem{lemma}[theorem]{Lemma}

% Do not include subsections and lower in TOC.
\setcounter{tocdepth}{1}

% BibTeX.
\renewcommand{\bibname}{References}
%\bibliographystyle{ieeetr}
\bibliographystyle{alpha}

% Penalties.

% TikZ-related stuff.
\tikzset{> = { Latex[length = 2mm] }}

% Listings
\colorlet{changedpurple}{TUDa-11a}
\colorlet{lerrorred}{TUDa-9b}
\colorlet{lstcomments}{TUDa-4c}
\colorlet{lstkeywords}{TUDa-9d}
\colorlet{lstlinenumbers}{TUDa-0c}
\colorlet{lststrings}{TUDa-2c}
\lstdefinestyle{base}{
	moredelim = **[is][\color{errorred}]{@@!}{@@@},
	moredelim = **[is][\color{changedpurple}]{@@?}{@@@}
}
\lstset{
	backgroundcolor   = \color{white},
	basicstyle        = \ttfamily\scriptsize\color{black},
	breakatwhitespace = true,
	breaklines        = true,
	breakautoindent   = true,
	commentstyle      = \color{lstcomments},
	escapeinside      = {{*@}{@*}},
	keywordstyle      = \color{lstkeywords},
	language          = Python,
	numbers           = left,
	numberstyle       = \tiny\color{lstlinenumbers},
	showstringspaces  = false,
	stringstyle       = \color{lststrings},
	style             = base,
	tabsize           = 4
}



% Glossary styles.
\newcolumntype{L}[1]{>{\raggedright\let\newline\\\arraybackslash\hspace{0pt}}m{#1}}
\newcolumntype{C}[1]{>{\centering\let\newline\\\arraybackslash\hspace{0pt}}m{#1}}
\newcolumntype{R}[1]{>{\raggedleft\let\newline\\\arraybackslash\hspace{0pt}}m{#1}}
\newglossarystyle{iasThesisGeneral}{
	\glossarystyle{super3colheader}%
	\renewenvironment{theglossary}%
		{\begin{longtable}{L{0.15\textwidth}L{0.8\textwidth}R{0\textwidth}}}%
		{\end{longtable}}%
	\renewcommand*{\glossaryheader}{\textbf{\entryname} & \textbf{\descriptionname} & \\}
	\renewcommand*{\glossaryentryfield}[5]{\glsentryitem{##1}\glstarget{##1}{##2} & ##3 \\}
}
\newglossarystyle{iasThesisOperators}{
	\glossarystyle{super3colheader}
	\renewenvironment{theglossary}
		{\begin{longtable}{L{0.15\textwidth}L{0.55\textwidth}L{0.25\textwidth}}}%
		{\end{longtable}}%
	\renewcommand*{\glossaryheader}{\textbf{\entryname} & \textbf{\descriptionname} & \textbf{Operator} \\}
	\renewcommand*{\glossaryentryfield}[5]{\glsentryitem{##1}\glstarget{##1}{##2} & ##3 & ##4 \\}
}
