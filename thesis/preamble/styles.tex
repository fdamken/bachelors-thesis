\makeatletter

% Make LOF a section.
\renewcommand\listoffigures{
	\section*{\listfigurename}
	\@mkboth{\MakeUppercase\listfigurename}{\MakeUppercase\listfigurename}
	\@starttoc{lof}
}
% Make LOT a section.
\renewcommand\listoftables{
	\section*{\listtablename}
	\@mkboth{\MakeUppercase\listtablename}{\MakeUppercase\listtablename}
	\@starttoc{lot}
}

\makeatother



% Style definitions.

\MakeOuterQuote{"}

% Captions should be centered.
\captionsetup{justification=centering}

% Matrix/Vector notation.
\renewcommand{\vec}[1]{\boldsymbol{\mathrm{#1}}}
\newcommand{\mat}[1]{\boldsymbol{\mathrm{#1}}}
% Shorthands.
\newcommand{\R}{\mathbb{R}}
\newcommand{\E}{\mathbb{E}}
\newcommand{\normal}{\mathcal{N}}
\newcommand{\gaussianMulti}[4]{\frac{1}{\left(2\pi\right)^{#4/2} \cdot \lvert #3 \rvert^{1/2}} \exp \bigg\{\! -\frac{1}{2} \left(#1 - #2\right)^T #3^{-1} \left(#1 - #2\right) \bigg\}}
\newcommand{\logGaussianMulti}[4]{-\frac{1}{2} \ln \lvert #3 \rvert - \frac{#4}{2} \ln\left(2\pi\right) - \frac{1}{2} \left(#1 - #2\right)^T #3^{-1} \left(#1 - #2\right)}
\newcommand{\subgiven}{\vert}
\newcommand{\given}{\,\vert\,}
\newcommand{\biggiven}{\,\big\vert\,}
\newcommand{\Biggiven}{\,\Big\vert\,}
\newcommand{\bigggiven}{\,\bigg\vert\,}
\newcommand{\Bigggiven}{\,\Bigg\vert\,}
\newcommand{\new}{\mathrm{new}}
% Math operators.
\DeclareMathOperator{\Cov}{Cov}

\newcommand{\oversetfootnotemark}[1]{\stepcounter{footnote} \overset{\mathclap{(\thefootnote)}}{#1}}
\let\realfootnote\footnote
\let\realfootnotetext\footnotetext
\renewcommand{\footnote}[1]{\realfootnote{\, #1}}
\renewcommand{\footnotetext}[1]{\realfootnotetext{\, #1}}
\newcommand{\doublefootnotetext}[2]{\addtocounter{footnote}{-1} \footnotetext{#1} \stepcounter{footnote} \footnotetext{#2}}
\newcommand{\triplefootnotetext}[3]{\addtocounter{footnote}{-2} \footnotetext{#1} \stepcounter{footnote} \footnotetext{#2} \stepcounter{footnote} \footnotetext{#3}}

\renewcommand{\arraystretch}{1.5}
\parindent0pt

% Allow page breaks inside of equations.
%\allowdisplaybreaks

% Do not include subsections and lower in TOC.
\setcounter{tocdepth}{1}

% BibTeX.
\renewcommand{\bibname}{References}
%\bibliographystyle{ieeetr}
\bibliographystyle{alpha}

% Penalties.

% TikZ-related stuff.



% Glossary styles.
\newcolumntype{L}[1]{>{\raggedright\let\newline\\\arraybackslash\hspace{0pt}}m{#1}}
\newcolumntype{C}[1]{>{\centering\let\newline\\\arraybackslash\hspace{0pt}}m{#1}}
\newcolumntype{R}[1]{>{\raggedleft\let\newline\\\arraybackslash\hspace{0pt}}m{#1}}
\newglossarystyle{iasThesisGeneral}{
	\glossarystyle{super3colheader}%
	\renewenvironment{theglossary}%
		{\begin{longtable}{L{0.15\textwidth}L{0.8\textwidth}R{0\textwidth}}}%
		{\end{longtable}}%
	\renewcommand*{\glossaryheader}{\bf{\entryname} & \bf{\descriptionname} & \\}
	\renewcommand*{\glossaryentryfield}[5]{\glsentryitem{##1}\glstarget{##1}{##2} & ##3 \\}
}
\newglossarystyle{iasThesisOperators}{
	\glossarystyle{super3colheader}
	\renewenvironment{theglossary}
		{\begin{longtable}{L{0.15\textwidth}L{0.55\textwidth}L{0.25\textwidth}}}%
		{\end{longtable}}%
	\renewcommand*{\glossaryheader}{\bf{\entryname} & \bf{\descriptionname} & \bf{Operator} \\}
	\renewcommand*{\glossaryentryfield}[5]{\glsentryitem{##1}\glstarget{##1}{##2} & ##3 & ##4 \\}
}
