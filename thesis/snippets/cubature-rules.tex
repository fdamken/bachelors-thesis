\section{Cubature Rules}

\subsection{Spherical-Radial Cubature Rule}
	Using the spherical-radial cubature rules~\cite{solinCubatureIntegrationMethods2010}, any Gaussian-like integral
	\begin{equation*}
		\int\! \vec{f}(\vec{x}) \,\normal(\vec{x} \given \vec{\mu}, \mat{\Sigma})
	\end{equation*}
	can be approximated using the cubature points \( \vec{\xi}_i = \sqrt{n} [\vec{1}]_i \):
	\begin{equation}
		\int\! \vec{f}(\vec{x}) \,\normal(\vec{x} \given \vec{\mu}, \mat{\Sigma}) \approx \frac{1}{2n} \sum_{i = 1}^{2n} \vec{f}\big( \sqrt{\mat{\Sigma}} \vec{\xi}_i + \vec{\mu} \big)  \label{eq:sphericalRadialGaussianCubatureRule}
	\end{equation}
	Here, \( \sqrt{\mat{\Sigma}} \) is a matrix such that \( \mat{\Sigma} = \sqrt{\mat{\Sigma}} \sqrt{\mat{\Sigma}} \), \(n\) is the dimension of \(\vec{x}\) and \( [\vec{1}]_i \) are "intersections between the Cartesian axes and the \(n\)-dimensional unit hypersphere."~\cite{solinCubatureIntegrationMethods2010}.
% end
