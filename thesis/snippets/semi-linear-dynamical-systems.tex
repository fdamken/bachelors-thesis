\section{Linear Dynamical Systems with Nonlinear Measurements}

Our idea is to replace the linear measurements \( \vec{y}_t \) in an \ac{lgds}
\begin{align*}
	\vec{s}_{t + 1} & = \mat{A} \vec{s}_t + \vec{w}_t \\
	\vec{y}_t       & = \mat{C} \vec{s}_t + \vec{v}_t
\end{align*}
reported in~\cite{ghahramaniParameterEstimationLinear1996} with an arbitrary, possible nonlinear but differentiable function \( \vec{g}_{\vec{\theta}} : \R^k \to \R^p \) with characterizing parameters \( \vec{\theta} \). In all of the following, we keep the parameters implicit for brevity.

The vectors \( \vec{w}_t \) and \( \vec{v}_t \) represent the purely additive Gaussian noise of the system (with zero mean and covariance \( \mat{Q} \) and \( \mat{R} \), respectively). Then the states \( \vec{s}_t \), \( \vec{s}_{t - 1} \) and emissions \( \vec{y}_t \) are jointly Gaussian~\cite{minkaHiddenMarkovModels1999}:
\begin{align*}
	p(\vec{s}_t \given \vec{s}_{t - 1}) & \sim \normal(\mat{A} \vec{s}_{t - 1}, \mat{Q})    \\
	p(\vec{y}_t \given \vec{s}_t)       & \sim \normal\big(\vec{g}(\vec{s}_t), \mat{R}\big)
\end{align*}
This model can also be written as a set of both linear and nonlinear equations:
\begin{align*}
	\vec{s}_{t + 1} & = \mat{A} \vec{s}_t + \vec{w}_t  \\
	\vec{w}_t       & \sim \normal(\vec{0}, \mat{Q})   \\
	\vec{y}_t       & = \vec{g}(\vec{s}_t) + \vec{v}_t \\
	\vec{v}_t       & \sim \normal(\vec{0}, \mat{R})
\end{align*}
In all of the following, we assume to have \(N\) \ac{iid} observation sequences. Let \( \vec{y}_{1:T}^{(n)} \) be the \(n\)-th observation sequence and \( \vec{s}_{1:T}^{(n)} \) the corresponding state sequence. All sequences share a single state dynamics matrix, the same noise covariances and observation function. Let \( \vec{y}_{1:T} \coloneqq \big(\vec{y}_{1:T}^{(1)}, \vec{y}_{1:T}^{(2)}, \cdots, \vec{y}_{1:T}^{(N)}\big) \) and \( \vec{s}_{1:T} \coloneqq \big(\vec{s}_{1:T}^{(1)}, \vec{s}_{1:T}^{(2)}, \cdots, \vec{s}_{1:T}^{(n)}\big) \) be the sequences of all observation and state sequences, respectively. The same goes for \( \vec{y}_t \) and \( \vec{s}_t \).


\subsection{M-Step}
	For a single observation sequence, the complete log-likelihood \( \ln p\Big(\vec{s}_{1:T}^{(n)}, \vec{y}_{1:T}^{(n)}\Big) \) has the form
	\begin{align*}
		\ln p\Big(\vec{s}_{1:T}^{(n)}, \vec{y}_{1:T}^{(n)}\Big)
			&= \ln \Bigg(\! p\Big(\vec{s}_1^{(n)}\Big) \prod_{t = 2}^{T} p\Big(\vec{s}_t^{(n)} \given \vec{s}_{t - 1}^{(n)}\Big) \prod_{t = 1}^{T} p\Big(\vec{y}_t^{(n)} \given \vec{s}_t^{(n)}\Big) \!\Bigg) \\
			&= \ln p\Big(\vec{s}_1^{(n)}\Big) + \sum_{t = 2}^{T} \ln p\Big(\vec{s}_t^{(n)} \given \vec{s}_{t - 1}^{(n)}\Big) + \sum_{t = 1}^{T} \ln p\Big(\vec{y}_t^{(n)} \given \vec{s}_t^{(n)}\Big) \\
			&= \logGaussianMulti{\vec{s}_1^{(n)}}{\vec{m}_0}{\mat{V}_0}{k} \\
				&\qquad\qquad + \sum_{t = 2}^{T} \bigg( \logGaussianMulti{\vec{s}_t^{(n)}}{\mat{A} \vec{s}_{t - 1}^{(n)}}{\mat{Q}}{k} \bigg) \\
				&\qquad\qquad + \sum_{t = 1}^{T} \bigg( \logGaussianMulti{\vec{y}_t^{(n)}}{\vec{g}\Big(\vec{s}_t^{(n)}\Big)}{\mat{R}}{p} \bigg) \\
			&= -\frac{T(k + p)}{2} \ln(2\pi) - \frac{1}{2} \ln \lvert \mat{V}_0 \rvert - \frac{T - 1}{2} \ln \lvert \mat{Q} \rvert - \frac{T}{2} \ln \lvert \mat{R} \rvert \\
				&\qquad\qquad -\frac{1}{2} \Big( \vec{s}_1^{(n)} - \vec{m}_0 \Big)^T \mat{V}_0^{-1} \Big( \vec{s}_1^{(n)} - \vec{m}_0 \Big) \\
				&\qquad\qquad -\frac{1}{2} \sum_{t = 2}^{T} \Big( \vec{s}_t^{(n)} - \mat{A} \vec{s}_{t - 1}^{(n)} \Big)^T \mat{Q}^{-1} \Big( \vec{s}_t^{(n)} - \mat{A} \vec{s}_{t - 1}^{(n)} \Big) \\
				&\qquad\qquad -\frac{1}{2} \sum_{t = 1}^{T} \Big( \vec{y}_t^{(n)} - \vec{g}\Big(\vec{s}_t^{(n)}\Big) \Big)^T \mat{R}^{-1} \Big( \vec{y}_t^{(n)} - \vec{g}\Big(\vec{s}_t^{(n)}\Big) \Big)
	\end{align*}
	where \( \vec{m}_0 \) and \( \mat{V}_0 \) are the initial state mean and covariance. As the observation sequences are independent, we can formulate the joint complete log-likelihood \( \ln p(\vec{s}_{1:T}, \vec{y}_{1:T}) \) as the sum of all subsequent log-likelihoods:
	\begin{align*}
		\ln p(\vec{s}_{1:T}, \vec{y}_{1:T})
			&= \ln \prod_{n = 1}^{N} p\Big(\vec{s}_{1:T}^{(n)}, \vec{y}_{1:T}^{(n)}\Big) = \sum_{n = 1}^{N} \ln p\Big(\vec{s}_{1:T}^{(n)}, \vec{y}_{1:T}^{(n)}\Big) \\
			&= -\frac{NT(k + p)}{2} \ln(2\pi) - \frac{N}{2} \ln \lvert \mat{V}_0 \rvert - \frac{N(T - 1)}{2} \ln \lvert \mat{Q} \rvert - \frac{NT}{2} \ln \lvert \mat{R} \rvert \\
				&\qquad\qquad -\frac{1}{2} \sum_{n = 1}^{N} \Big( \vec{s}_1^{(n)} - \vec{m}_0 \Big)^T \mat{V}_0^{-1} \Big( \vec{s}_1^{(n)} - \vec{m}_0 \Big) \\
				&\qquad\qquad -\frac{1}{2} \sum_{n = 1}^{N} \sum_{t = 2}^{T} \Big( \vec{s}_t^{(n)} - \mat{A} \vec{s}_{t - 1}^{(n)} \Big)^T \mat{Q}^{-1} \Big( \vec{s}_t^{(n)} - \mat{A} \vec{s}_{t - 1}^{(n)} \Big) \\
				&\qquad\qquad -\frac{1}{2} \sum_{n = 1}^{N} \sum_{t = 1}^{T} \Big( \vec{y}_t^{(n)} - \vec{g}\Big(\vec{s}_t^{(n)}\Big) \Big)^T \mat{R}^{-1} \Big( \vec{y}_t^{(n)} - \vec{g}\Big(\vec{s}_t^{(n)}\Big) \Big)
	\end{align*}
	
	To derive the M-step formulas, we need to maximize the \emph{expected} complete log-likelihood. Therefore, we will now derive the expected log-likelihood
	\begin{equation*}
		Q = \E\big[ p(\vec{s}_{1:T}, \vec{y}_{1:T}) \given \vec{y}_{1:T} \big]
	\end{equation*}
	This quantity depends on three expectations
	\begin{align*}
		\hat{\vec{s}}_{t \subgiven t_0}^{(n)}  & \coloneqq \E\Big[\vec{s}_t^{(n)} \Biggiven \vec{y}_{1:t_0}\Big]                                      & \hat{\vec{s}}_{t, t_0}           & \coloneqq \frac{1}{N} \sum_{n = 1}^{N} \hat{\vec{s}}_{t \subgiven t_0}^{(n)}  \\
		\mat{P}_{t \subgiven t_0}^{(n)}        & \coloneqq \E\bigg[\vec{s}_t^{(n)} \Big(\vec{s}_t^{(n)}\Big)^T \bigggiven \vec{y}_{1:t_0}\bigg]       & \mat{P}_{t \subgiven t_0}        & \coloneqq \frac{1}{N} \sum_{n = 1}^{N} \mat{P}_{t \subgiven t_0}^{(n)}        \\
		\mat{P}_{t, t - 1 \subgiven t_0}^{(n)} & \coloneqq \E\bigg[\vec{s}_t^{(n)} \Big(\vec{s}_{t - 1}^{(n)}\Big)^T \bigggiven \vec{y}_{1:t_0}\bigg] & \mat{P}_{t, t - 1 \subgiven t_0} & \coloneqq \frac{1}{N} \sum_{n = 1}^{N} \mat{P}_{t, t - 1 \subgiven t_0}^{(n)}
	\end{align*}
	which form the expected state, self-correlation and cross-correlation, respectively. We call \( \hat{\vec{s}}_{t \subgiven t - 1} \) the prior, \( \hat{\vec{s}}_{t \subgiven t} \) the posterior and \( \hat{\vec{s}}_{t \subgiven T} \) the smoothed states (same for the self- and cross-correlation). For brevity, we also write \( \hat{\vec{s}}_t \) for the smoothed states \( \hat{\vec{s}}_{t \subgiven T} \) (same for the self- and cross-correlation). Additionally we define
	\begin{align}
		\hat{\vec{g}}_t^{(n)} & \coloneqq \E\Big[\vec{g}\Big(\vec{s}_t^{(n)}\Big) \Biggiven \vec{y}_{1:T}\Big]    \label{eq:expectedMeasurement}                                              \\
		\mat{G}_t^{(n)}       & \coloneqq \E\bigg[\vec{g}\Big(\vec{s}_t^{(n)}\Big) \Big(\vec{g}\Big(\vec{s}_t\Big)\Big)^T \bigggiven \vec{y}_{1:T}\bigg]    \label{eq:expectedMeasurementMat}
	\end{align}
	to be the expectations of the measurement function \( \vec{g}(\vec{s}_t) \). We will see that evaluating this expectation is not possible in a closed form and has to be approximated, but for now we will be deriving the expected complete log-likelihood dependent on the defined expectations.
	
	For simplicity, let
	\begin{align*}
		q_1 &\coloneqq -\frac{NT(k + p)}{2} \ln(2\pi) - \frac{N}{2} \ln \lvert \mat{V}_0 \rvert - \frac{N(T - 1)}{2} \ln \lvert \mat{Q} \rvert - \frac{NT}{2} \ln \lvert \mat{R} \rvert \\
		q_2 &\coloneqq -\frac{1}{2} \sum_{n = 1}^{N} \Big( \vec{s}_1^{(n)} - \vec{m}_0 \Big)^T \mat{V}_0^{-1} \Big( \vec{s}_1^{(n)} - \vec{m}_0 \Big) \\
		q_3 &\coloneqq -\frac{1}{2} \sum_{n = 1}^{N} \sum_{t = 2}^{T} \Big( \vec{s}_t^{(n)} - \mat{A} \vec{s}_{t - 1}^{(n)} \Big)^T \mat{Q}^{-1} \Big( \vec{s}_t^{(n)} - \mat{A} \vec{s}_{t - 1}^{(n)} \Big) \\
		q_4 &\coloneqq -\frac{1}{2} \sum_{n = 1}^{N} \sum_{t = 1}^{T} \Big( \vec{y}_t^{(n)} - \vec{g}\Big(\vec{s}_t^{(n)}\Big) \Big)^T \mat{R}^{-1} \Big( \vec{y}_t^{(n)} - \vec{g}\Big(\vec{s}_t^{(n)}\Big) \Big)
	\end{align*}
	such that \( \ln p(\vec{s}_{1:T}, \vec{y}_{1:T}) = q_1 + q_2 + q_3 + q_4 \) and, with \(Q_1\), \(Q_2\), \(Q_3\) and \(Q_4\) being the corresponding expectations, \( Q = Q_1 + Q_2 + Q_3 + Q_4 \).
	
	We start with \(Q_1\):
	\begin{equation*}
		Q_1 = \E[q_1 \given \vec{y}_{1:T}] = -\frac{NT(k + p)}{2} \ln(2\pi) - \frac{N}{2} \ln \lvert \mat{V}_0 \rvert - \frac{N(T - 1)}{2} \ln \lvert \mat{Q} \rvert - \frac{NT}{2} \ln \lvert \mat{R} \rvert
	\end{equation*}
	Then following \(Q_2\):
	\begin{align*}
		Q_2
			&= \E[q_2 \given \vec{y}_{1:T}] \\
			&= \E\Bigg[ -\frac{1}{2} \sum_{n = 1}^{N} \Big( \vec{s}_1^{(n)} - \vec{m}_0 \Big)^T \mat{V}_0^{-1} \Big( \vec{s}_1^{(n)} - \vec{m}_0 \Big) \bigggiven \vec{y}_{1:T} \Bigg] \\
			&= -\frac{1}{2} \sum_{n = 1}^{N} \E\Bigg[ \Big( \vec{s}_1^{(n)} - \vec{m}_0 \Big)^T \mat{V}_0^{-1} \Big( \vec{s}_1^{(n)} - \vec{m}_0 \Big) \bigggiven \vec{y}_{1:T} \Bigg] \\
			&= -\frac{1}{2} \sum_{n = 1}^{N} \E\Bigg[ \tr\!\bigg(\!\Big( \vec{s}_1^{(n)} - \vec{m}_0 \Big)^T \mat{V}_0^{-1} \Big( \vec{s}_1^{(n)} - \vec{m}_0 \Big)\!\bigg) \bigggiven \vec{y}_{1:T} \Bigg] \\
			&= -\frac{1}{2} \sum_{n = 1}^{N} \E\Bigg[ \tr\!\bigg(\!\Big( \vec{s}_1^{(n)} - \vec{m}_0 \Big) \Big( \vec{s}_1^{(n)} - \vec{m}_0 \Big)^T \mat{V}_0^{-1} \!\bigg) \bigggiven \vec{y}_{1:T} \Bigg] \\
			&= -\frac{1}{2} \sum_{n = 1}^{N} \E\Bigg[ \tr\!\bigg(\!\Big( \vec{s}_1^{(n)} - \vec{m}_0 \Big) \Big( \Big(\vec{s}_1^{(n)}\Big)^T - \vec{m}_0^T \Big) \mat{V}_0^{-1} \!\bigg) \bigggiven \vec{y}_{1:T} \Bigg] \\
			&= -\frac{1}{2} \sum_{n = 1}^{N} \E\Bigg[ \tr\!\bigg(\! \vec{s}_1^{(n)} \Big(\vec{s}_1^{(n)}\Big)^T - \vec{s}_1^{(n)} \vec{m}_0^T - \vec{m}\Big(\vec{s}_1^{(n)}\Big)^T + \vec{m}_0 \vec{m}_0^T \!\bigg) \mat{V}_0^{-1} \bigggiven \vec{y}_{1:T} \Bigg] \\
			&= -\frac{1}{2} \sum_{n = 1}^{N} \tr\!\bigg(\! \mat{P}_1^{(n)} \mat{V}_0^{-1} - \hat{\vec{s}}_1^{(n)} \vec{m}_0^T \mat{V}_0^{-1} - \vec{m}\Big(\hat{\vec{s}}_1^{(n)}\Big)^T \mat{V}_0^{-1} + \vec{m}_0 \vec{m}_0^T \mat{V}_0^{-1} \!\bigg) \\
			&= -\frac{1}{2} \sum_{n = 1}^{N} \tr\!\Big( \mat{P}_1^{(n)} \mat{V}_0^{-1} \Big) - \tr\!\Big( \hat{\vec{s}}_1^{(n)} \vec{m}_0^T \mat{V}_0^{-1} \Big) - \tr\!\bigg(\! \vec{m}_0 \Big(\hat{\vec{s}}_1^{(n)}\Big)^T \mat{V}_0^{-1} \!\bigg) + \tr\!\Big(\vec{m}_0 \vec{m}_0^T \mat{V}_0^{-1} \Big) \\
			&=  -\frac{N}{2} \tr\!\Big( \mat{P}_1 \mat{V}_0^{-1} \Big) + \frac{N}{2} \tr\!\Big( \hat{\vec{s}}_1 \vec{m}_0^T \mat{V}_0^{-1} \Big) + \frac{N}{2} \tr\!\Big( \vec{m}_0 \hat{\vec{s}}_1^T \mat{V}_0^{-1} \Big) - \frac{N}{2} \tr\!\Big(\vec{m}_0 \vec{m}_0^T \mat{V}_0^{-1} \Big)
	\end{align*}
	And \(Q_3\):
	\begin{align*}
		Q_3
			&= \E[q_3 \given \vec{y}_{1:T}] \\
			&= \E\Bigg[ -\frac{1}{2} \sum_{n = 1}^{N} \sum_{t = 2}^{T} \Big( \vec{s}_t^{(n)} - \mat{A} \vec{s}_{t - 1}^{(n)} \Big)^T \mat{Q}^{-1} \Big( \vec{s}_t^{(n)} - \mat{A} \vec{s}_{t - 1}^{(n)} \Big) \bigggiven \vec{y}_{1:T} \Bigg] \\
			&= -\frac{1}{2} \sum_{n = 1}^{N} \sum_{t = 2}^{T} \E\Bigg[ \Big( \vec{s}_t^{(n)} - \mat{A} \vec{s}_{t - 1}^{(n)} \Big)^T \mat{Q}^{-1} \Big( \vec{s}_t^{(n)} - \mat{A} \vec{s}_{t - 1}^{(n)} \Big) \bigggiven \vec{y}_{1:T} \Bigg] \\
			&= -\frac{1}{2} \sum_{n = 1}^{N} \sum_{t = 2}^{T} \E\Bigg[ \tr\!\bigg(\!\Big( \vec{s}_t^{(n)} - \mat{A} \vec{s}_{t - 1}^{(n)} \Big)^T \mat{Q}^{-1} \Big( \vec{s}_t^{(n)} - \mat{A} \vec{s}_{t - 1}^{(n)} \Big)\!\bigg) \bigggiven \vec{y}_{1:T} \Bigg] \\
			&= -\frac{1}{2} \sum_{n = 1}^{N} \sum_{t = 2}^{T} \E\Bigg[ \tr\!\bigg(\!\Big( \vec{s}_t^{(n)} - \mat{A} \vec{s}_{t - 1}^{(n)} \Big) \Big( \vec{s}_t^{(n)} - \mat{A} \vec{s}_{t - 1}^{(n)} \Big)^T \mat{Q}^{-1} \!\bigg) \bigggiven \vec{y}_{1:T} \Bigg] \\
			&= -\frac{1}{2} \sum_{n = 1}^{N} \sum_{t = 2}^{T} \E\Bigg[ \tr\!\bigg(\!\Big( \vec{s}_t^{(n)} - \mat{A} \vec{s}_{t - 1}^{(n)} \Big) \Big(\big(\vec{s}_t^{(n)}\big)^T - \big(\vec{s}_{t - 1}^{(n)}\big)^T \mat{A}^T \Big) \mat{Q}^{-1} \!\bigg) \bigggiven \vec{y}_{1:T} \Bigg] \\
			&= -\frac{1}{2} \sum_{n = 1}^{N} \sum_{t = 2}^{T} \E\Bigg[ \tr\!\bigg(\!\Big( \vec{s}_t^{(n)} \Big(\vec{s}_t^{(n)}\Big)^T - \vec{s}_t^{(n)} \Big(\vec{s}_{t - 1}^{(n)}\Big)^T \mat{A}^T - \mat{A} \vec{s}_{t - 1}^{(n)} \Big(\vec{s}_t^{(n)}\Big)^T - \mat{A} \vec{s}_{t - 1}^{(n)} \Big(\vec{s}_{t - 1}^{(n)}\Big)^T \mat{A}^T \Big) \mat{Q}^{-1} \!\bigg) \bigggiven \vec{y}_{1:T} \Bigg] \\
			&= -\frac{1}{2} \sum_{n = 1}^{N} \sum_{t = 2}^{T} \tr\!\Big( \mat{P}_t^{(n)} \mat{Q}^{-1} - \mat{P}_{t, t - 1}^{(n)} \mat{A}^T \mat{Q}^{-1} - \mat{A} \underbrace{\mat{P}_{t - 1, t}^{(n)}}_{=\, \mat{P}_{t, t - 1}^{(n)}} \mat{Q}^{-1} - \mat{A} \mat{P}_{t - 1}^{(n)} \mat{A}^T \mat{Q}^{-1} \Big) \\
			&= -\frac{1}{2} \sum_{n = 1}^{N} \sum_{t = 2}^{T} \tr\!\Big( \mat{P}_t^{(n)} \mat{Q}^{-1} \Big) - \tr\!\Big( \mat{P}_{t, t - 1}^{(n)} \mat{A}^T \mat{Q}^{-1} \Big) - \tr\!\Big( \mat{A} \mat{P}_{t, t - 1}^{(n)} \mat{Q}^{-1} \Big) - \tr\!\Big( \mat{A} \mat{P}_{t - 1}^{(n)} \mat{A}^T \mat{Q}^{-1} \Big) \\
			&= -\frac{N}{2} \sum_{t = 2}^{T} \tr\!\Big( \mat{P}_t \mat{Q}^{-1} \Big) - \tr\!\Big( \mat{P}_{t, t - 1} \mat{A}^T \mat{Q}^{-1} \Big) - \tr\!\Big( \mat{A} \mat{P}_{t, t - 1} \mat{Q}^{-1} \Big) - \tr\!\Big( \mat{A} \mat{P}_{t - 1} \mat{A}^T \mat{Q}^{-1} \Big)
	\end{align*}
	Finally, we derive \(Q_4\). This is the one that depends on the non-closed-form expectations~\eqref{eq:expectedMeasurement} and~\eqref{eq:expectedMeasurementMat}:
	\begin{align*}
		Q_4
			&= \E[q_4 \given \vec{y}_{1:T}] \\
			&= \E\Bigg[ -\frac{1}{2} \sum_{n = 1}^{N} \sum_{t = 1}^{T} \Big( \vec{y}_t^{(n)} - \vec{g}\Big(\vec{s}_t^{(n)}\Big) \Big)^T \mat{R}^{-1} \Big( \vec{y}_t^{(n)} - \vec{g}\Big(\vec{s}_t^{(n)}\Big) \Big) \bigggiven \vec{y}_{1:T} \Bigg] \\
			&= -\frac{1}{2} \sum_{n = 1}^{N} \sum_{t = 1}^{T} \E\Bigg[ \Big( \vec{y}_t^{(n)} - \vec{g}\Big(\vec{s}_t^{(n)}\Big) \Big)^T \mat{R}^{-1} \Big( \vec{y}_t^{(n)} - \vec{g}\Big(\vec{s}_t^{(n)}\Big) \Big) \bigggiven \vec{y}_{1:T} \Bigg] \\
			&= -\frac{1}{2} \sum_{n = 1}^{N} \sum_{t = 1}^{T} \E\Bigg[ \tr\!\bigg(\!\Big( \vec{y}_t^{(n)} - \vec{g}\Big(\vec{s}_t^{(n)}\Big) \Big)^T \mat{R}^{-1} \Big( \vec{y}_t^{(n)} - \vec{g}\Big(\vec{s}_t^{(n)}\Big) \Big)\!\bigg) \bigggiven \vec{y}_{1:T} \Bigg] \\
			&= -\frac{1}{2} \sum_{n = 1}^{N} \sum_{t = 1}^{T} \E\Bigg[ \tr\!\bigg(\!\Big( \vec{y}_t^{(n)} - \vec{g}\Big(\vec{s}_t^{(n)}\Big) \Big) \Big( \vec{y}_t^{(n)} - \vec{g}\Big(\vec{s}_t^{(n)}\Big) \Big)^T \mat{R}^{-1} \!\bigg) \bigggiven \vec{y}_{1:T} \Bigg] \\
			&= -\frac{1}{2} \sum_{n = 1}^{N} \sum_{t = 1}^{T} \E\Bigg[ \tr\!\bigg(\!\Big( \vec{y}_t^{(n)} - \vec{g}\Big(\vec{s}_t^{(n)}\Big) \Big) \bigg(\!\Big(\vec{y}_t^{(n)}\Big)^T - \Big(\vec{g}\Big(\vec{s}_t^{(n)}\Big)\Big)^T \bigg) \mat{R}^{-1} \!\bigg) \bigggiven \vec{y}_{1:T} \Bigg] \\
			&= -\frac{1}{2} \sum_{n = 1}^{N} \sum_{t = 1}^{T} \E\Bigg[ \tr\!\bigg(\!\bigg(\! \vec{y}_t^{(n)} \Big(\vec{y}_t^{(n)}\Big)^T - \vec{y}_t^{(n)} \Big(\vec{g}\Big(\vec{s}_t^{(n)}\Big)\Big)^T - \vec{g}\Big(\vec{s}_t^{(n)}\Big) \Big(\vec{y}_t^{(n)}\Big)^T + \vec{g}\Big(\vec{s}_t^{(n)}\Big) \Big(\vec{g}\Big(\vec{s}_t^{(n)}\Big)\Big)^T \bigg) \mat{R}^{-1} \!\bigg) \bigggiven \vec{y}_{1:T} \Bigg] \\
			&= -\frac{1}{2} \sum_{n = 1}^{N} \sum_{t = 1}^{T} \tr\!\bigg(\! \vec{y}_t^{(n)} \Big(\vec{y}_t^{(n)}\Big)^T \mat{R}^{-1} - \vec{y}_t^{(n)} \Big(\hat{\vec{g}}_t^{(n)}\Big)^T \mat{R}^{-1} - \hat{\vec{g}}_t^{(n)} \Big(\vec{y}_t^{(n)}\Big)^T \mat{R}^{-1} + \mat{G}_t^{(n)} \mat{R}^{-1} \!\bigg) \\
			&= -\frac{1}{2} \sum_{n = 1}^{N} \sum_{t = 1}^{T} \tr\!\bigg(\! \vec{y}_t^{(n)} \Big(\vec{y}_t^{(n)}\Big)^T \mat{R}^{-1} \!\bigg) - \tr\!\bigg(\! \vec{y}_t^{(n)} \Big(\hat{\vec{g}}_t^{(n)}\Big)^T \mat{R}^{-1} \!\bigg) - \tr\!\bigg(\! \hat{\vec{g}}_t^{(n)} \Big(\vec{y}_t^{(n)}\Big)^T \mat{R}^{-1} \!\bigg) + \tr\!\Big( \mat{G}_t^{(n)} \mat{R}^{-1} \Big) \\
	\end{align*}
	
	Putting it all together, the expected complete log-likelihood across all observation sequences is given as
	\begin{align*}
		Q
			&= Q_1 + Q_2 + Q_3 + Q_4 \\
			&= -\frac{NT(k + p)}{2} \ln(2\pi) - \frac{N}{2} \ln \lvert \mat{V}_0 \rvert - \frac{N(T - 1)}{2} \ln \lvert \mat{Q} \rvert - \frac{NT}{2} \ln \lvert \mat{R} \rvert \\
				&\qquad\qquad -\frac{N}{2} \tr\!\Big( \mat{P}_1 \mat{V}_0^{-1} \Big) + \frac{N}{2} \tr\!\Big( \hat{\vec{s}}_1 \vec{m}_0^T \mat{V}_0^{-1} \Big) + \frac{N}{2} \tr\!\Big( \vec{m}_0 \hat{\vec{s}}_1^T \mat{V}_0^{-1} \Big) - \frac{N}{2} \tr\!\Big(\vec{m}_0 \vec{m}_0^T \mat{V}_0^{-1} \Big) \\
				&\qquad\qquad -\frac{N}{2} \sum_{t = 2}^{T} \tr\!\Big( \mat{P}_t \mat{Q}^{-1} \Big) - \tr\!\Big( \mat{P}_{t, t - 1} \mat{A}^T \mat{Q}^{-1} \Big) - \tr\!\Big( \mat{A} \mat{P}_{t, t - 1} \mat{Q}^{-1} \Big) - \tr\!\Big( \mat{A} \mat{P}_{t - 1} \mat{A}^T \mat{Q}^{-1} \Big) \\
				&\qquad\qquad -\frac{1}{2} \sum_{n = 1}^{N} \sum_{t = 1}^{T} \tr\!\bigg(\! \vec{y}_t^{(n)} \Big(\vec{y}_t^{(n)}\Big)^T \mat{R}^{-1} \!\bigg) - \tr\!\bigg(\! \vec{y}_t^{(n)} \Big(\hat{\vec{g}}_t^{(n)}\Big)^T \mat{R}^{-1} \!\bigg) - \tr\!\bigg(\! \hat{\vec{g}}_t^{(n)} \Big(\vec{y}_t^{(n)}\Big)^T \mat{R}^{-1} \!\bigg) + \tr\!\Big( \mat{G}_t^{(n)} \mat{R}^{-1} \Big)
	\end{align*}
	where \( Q_1 \), \( Q_2 \), \( Q_3 \) and \( Q_4 \) are functions of the parameters \( \mat{A} \), \( \mat{Q} \), \( \vec{\theta} \), \( \mat{R} \), \( \vec{m}_0 \) and \( \mat{V}_0 \):
	\begin{align*}
		Q_1 & = Q_1(\mat{V}_0, \mat{Q}, \mat{R}) \\
		Q_2 & = Q_2(\vec{m}_0, \mat{V}_0)        \\
		Q_3 & = Q_3(\mat{A}, \mat{Q})            \\
		Q_4 & = Q_4(\vec{\theta}, \mat{R})
	\end{align*}

	Now we have everything together to derive the M-step equations by maximizing \(Q\). To maximize \(Q\) \ac{wrt} all parameters, that is
	\begin{itemize}
		\item state dynamics matrix \(\mat{A}\),
		\item state noise covariance \(\mat{Q}\),
		\item measurement function parameters \(\vec{\theta}\),
		\item measurement noise covariance \(\mat{R}\),
		\item initial state mean \(\vec{m}_0\) and
		\item initial state covariance \(\mat{V}_0\),
	\end{itemize}
	we have to take the derivatives \ac{wrt} to all the above parameters and set them to zero.
	
	To maximize \(Q\) \ac{wrt} all parameters, we have to take the derivatives \ac{wrt} the parameters and set them to zero.
	\begin{itemize}
		\item State dynamics matrix \(\mat{A}\):
	\end{itemize}
	\begin{align}
		&&\pdv{Q}{\mat{A}}
			&= \pdv{Q_1}{\mat{A}} + \pdv{Q_2}{\mat{A}} + \pdv{Q_3}{\mat{A}} + \pdv{Q_4}{\mat{A}} = \pdv{Q_3}{\mat{A}} & \nonumber \\
		&&	&= -\frac{N}{2} \sum_{t = 2}^{T} -\mat{Q}^{-1} \mat{P}_{t, t - 1} - \mat{Q}^{-T} \mat{P}_{t, t - 1}^T + \mat{Q}^{-T} \mat{A} \mat{P}_{t - 1}^T + \mat{Q}^{-1} \mat{A} \mat{P}_{t - 1} & \nonumber \\
		&&	&\oversetfootnotemark{=} -N \sum_{t = 2}^{T} -\mat{Q}^{-1} \mat{P}_{t, t - 1} + \mat{Q}^{-1} \mat{A} \mat{P}_{t - 1} \overset{!}{=} \mat{0} & \nonumber \\
		\implies && \mat{A}^\new &= \Bigg(\! \sum_{t = 2}^{T} \mat{P}_{t, t - 1} \!\Bigg) \Bigg(\! \sum_{t = 2}^{T} \mat{P}_{t - 1} \!\Bigg)^{-1} & \label{eq:stateDynamicsMatrix}
	\end{align}
	\footnotetext{Covariance and correlation matrices (\( \mat{Q} \), \( \mat{R} \), \( \mat{P}_t \), \( \mat{P}_{t, t - 1} \)) are symmetric by definition.}
	
	\begin{itemize}
		\item State noise covariance \(\mat{Q}\): \\ Instead of maximizing \ac{wrt} \(\mat{Q}\), we can also minimize \ac{wrt} \(\mat{Q}^{-1}\) which has the same effect.
	\end{itemize}
	\begin{align}
		&&\pdv{Q}{\mat{Q}^{-1}}
			&= \pdv{Q_1}{\mat{Q}^{-1}} + \pdv{Q_2}{\mat{Q}^{-1}} + \pdv{Q_3}{\mat{Q}^{-1}} + \pdv{Q_4}{\mat{Q}^{-1}} = \pdv{Q_1}{\mat{Q}^{-1}} + \pdv{Q_3}{\mat{Q}^{-1}} & \nonumber \\
		&&	&\oversetfootnotemark{=} \frac{N(T - 1)}{2} \mat{Q} - \frac{N}{2} \sum_{t = 2}^{T} \mat{P}_t^T - \mat{A} \mat{P}_{t, t - 1}^T - \mat{P}_{t, t - 1}^T \mat{A}^T + \mat{A} \mat{P}_{t - 1}^T \mat{A}^T & \nonumber \\
		&&	&= \frac{N(T - 1)}{2} \mat{Q} - \frac{N}{2} \sum_{t = 2}^{T} \mat{P}_t - \mat{A} \mat{P}_{t, t - 1} - \mat{P}_{t, t - 1} \mat{A}^T + \mat{A} \mat{P}_{t - 1} \mat{A}^T & \nonumber \\
		&&	&= \frac{N(T - 1)}{2} \mat{Q} - \frac{N}{2} \sum_{t = 2}^{T} \mat{P}_t + \frac{N}{2} \sum_{t = 2}^{T} \mat{A} \mat{P}_{t, t - 1} + \frac{N}{2} \sum_{t = 2}^{T} \mat{P}_{t, t - 1} \mat{A}^T - \frac{N}{2} \sum_{t = 2}^{T} \mat{A} \mat{P}_{t - 1} \mat{A}^T & \nonumber \\
		&&	&= \frac{N(T - 1)}{2} \mat{Q} - \frac{N}{2} \sum_{t = 2}^{T} \mat{P}_t + \frac{N}{2} \mat{A}^\new \sum_{t = 2}^{T} \mat{P}_{t, t - 1} + \frac{N}{2} \Bigg(\! \sum_{t = 2}^{T} \mat{P}_{t, t - 1} \!\Bigg) \big(\mat{A}^\new\big)^T & \nonumber \\
			&&	&\qquad\qquad - \frac{N}{2} \mat{A}^\new \Bigg(\! \sum_{t = 2}^{T} \mat{P}_{t - 1} \!\Bigg) \big(\mat{A}^\new\big)^T & \nonumber \\
		&&	&= \frac{N(T - 1)}{2} \mat{Q} - \frac{N}{2} \sum_{t = 2}^{T} \mat{P}_t + \frac{N}{2} \mat{A}^\new \sum_{t = 2}^{T} \mat{P}_{t, t - 1} + \frac{N}{2} \Bigg(\! \sum_{t = 2}^{T} \mat{P}_{t, t - 1} \!\Bigg) \Bigg(\! \sum_{t = 2}^{T} \mat{P}_{t - 1} \!\Bigg)^{-T} \Bigg(\! \sum_{t = 2}^{T} \mat{P}_{t, t - 1} \!\Bigg)^T & \nonumber \\
			&&	&\qquad\qquad - \frac{N}{2} \Bigg(\! \sum_{t = 2}^{T} \mat{P}_{t, t - 1} \!\Bigg) \Bigg(\! \sum_{t = 2}^{T} \mat{P}_{t - 1} \!\Bigg)^{-1} \Bigg(\! \sum_{t = 2}^{T} \mat{P}_{t - 1} \!\Bigg) \Bigg(\! \sum_{t = 2}^{T} \mat{P}_{t - 1} \!\Bigg)^{-T} \Bigg(\! \sum_{t = 2}^{T} \mat{P}_{t, t - 1} \!\Bigg)^T & \nonumber \\
		&&	&= \frac{N(T - 1)}{2} \mat{Q} - \frac{N}{2} \sum_{t = 2}^{T} \mat{P}_t + \frac{N}{2} \mat{A}^\new \sum_{t = 2}^{T} \mat{P}_{t, t - 1} + \frac{N}{2} \Bigg(\! \sum_{t = 2}^{T} \mat{P}_{t, t - 1} \!\Bigg) \Bigg(\! \sum_{t = 2}^{T} \mat{P}_{t - 1} \!\Bigg)^{-1} \Bigg(\! \sum_{t = 2}^{T} \mat{P}_{t, t - 1} \!\Bigg) & \nonumber \\
			&&	&\qquad\qquad - \frac{N}{2} \Bigg(\! \sum_{t = 2}^{T} \mat{P}_{t, t - 1} \!\Bigg) \Bigg(\! \sum_{t = 2}^{T} \mat{P}_{t - 1} \!\Bigg)^{-1} \cancel{\Bigg(\! \sum_{t = 2}^{T} \mat{P}_{t - 1} \!\Bigg) \Bigg(\! \sum_{t = 2}^{T} \mat{P}_{t - 1} \!\Bigg)^{-1}} \Bigg(\! \sum_{t = 2}^{T} \mat{P}_{t, t - 1} \!\Bigg) & \nonumber \\
		&&	&= \frac{N(T - 1)}{2} \mat{Q} - \frac{N}{2} \Bigg(\! \sum_{t = 2}^{T} \mat{P}_t - \mat{A}^\new \sum_{t = 2}^{T} \mat{P}_{t, t - 1} \!\Bigg) \overset{!}{=} \mat{0} & \nonumber \\
		\implies && \mat{Q}^\new &= \frac{1}{T - 1} \Bigg(\! \sum_{t = 2}^{T} \mat{P}_t - \mat{A}^\new \sum_{t = 2}^{T} \mat{P}_{t, t - 1} \!\Bigg) & \label{eq:stateNoiseCovariance}
	\end{align}
	\footnotetext{Note that \( \pdv{\mat{Q}^{-1}} \ln \det \mat{Q} = \pdv{\mat{Q}^{-1}} \ln \det \big(\mat{Q}^{-1}\big)^{-1} = \bigg( \pdv{\det (\mat{Q}^{-1})^{-1}} \ln \det \big(\mat{Q}^{-1}\big)^{-1} \bigg) \bigg( \pdv{\mat{Q}^{-1}} \det \big(\mat{Q}^{-1}\big)^{-1} \bigg) \overset{\text{cov. matrix}}{=} -\mat{Q} \)}

	\begin{itemize}
		\item Initial state mean \(\vec{m}_0\):
	\end{itemize}
	\begin{align}
		&&\pdv{Q}{\vec{m}_0}
			&= \pdv{Q_1}{\vec{m}_0} + \pdv{Q_2}{\vec{m}_0} + \pdv{Q_3}{\vec{m}_0} + \pdv{Q_4}{\vec{m}_0} = \pdv{Q_2}{\vec{m}_0} & \nonumber \\
		&&	&= \frac{N}{2} \mat{V}_0^{-1} \hat{\vec{s}}_1 + \frac{N}{2} \mat{V}_0^{-T} \hat{\vec{s}}_1 - \frac{N}{2} \big( \mat{V}_0^{-1} \vec{m}_0 + \mat{V}_0^{-T} \vec{m}_0 \big) & \nonumber \\
		&&	&= N \mat{V}_0^{-1} \hat{\vec{s}}_1 - N \mat{V}_0^{-1} \vec{m}_0 & \nonumber \\
		&&	&= N \mat{V}_0^{-1} \big( \hat{\vec{s}}_1 - \vec{m}_0 \big) \overset{!}{=} \vec{0} & \nonumber \\
		\implies && \vec{m}_0^\new &= \hat{\vec{s}}_1 = \frac{1}{N} \sum_{n = 0}^{N} \hat{\vec{s}}_1^{(n)} & \label{eq:initialStateMean}
	\end{align}
	
	\begin{itemize}
		\item Initial state covariance \(\mat{V}_0\):
	\end{itemize}
	\begin{align}
		&&\pdv{Q}{\mat{V}_0^{-1}}
			&= \pdv{Q_1}{\mat{V}_0^{-1}} + \pdv{Q_2}{\mat{V}_0^{-1}} + \pdv{Q_3}{\mat{V}_0^{-1}} + \pdv{Q_4}{\mat{V}_0^{-1}} = \pdv{Q_1}{\mat{V}_0^{-1}} + \pdv{Q_2}{\mat{V}_0^{-1}} & \nonumber \\
		&&	&= \frac{N}{2} \mat{V}_0 - \frac{N}{2} \mat{P}_1^T + \frac{N}{2} \vec{m}_0 \hat{\vec{s}}_1^T + \frac{N}{2} \hat{\vec{s}}_1 \vec{m}_0^T - \frac{N}{2} \vec{m}_0 \vec{m}_0^T & \nonumber \\
		&&	&= \frac{N}{2} \mat{V}_0 - \frac{N}{2} \mat{P}_1^T + \frac{N}{2} \hat{\vec{s}}_1 \hat{\vec{s}}_1^T + \frac{N}{2} \hat{\vec{s}}_1 \hat{\vec{s}}_1^T - \frac{N}{2} \hat{\vec{s}}_1 \hat{\vec{s}}_1^T & \nonumber \\
		&&	&= \frac{N}{2} \mat{V}_0 - \frac{N}{2} \mat{P}_1^T + \frac{N}{2} \hat{\vec{s}}_1 \hat{\vec{s}}_1^T \overset{!}{=} \mat{0} & \nonumber \\
		\implies && \mat{V}_0^\new &= \mat{P}_1^T - \hat{\vec{s}}_1 \hat{\vec{s}}_1^T & \label{eq:initialStateCovariance}
	\end{align}
	
	\begin{itemize}
		\item Measurement noise covariance \(\mat{R}\):
	\end{itemize}
	\begin{align}
		&&\pdv{Q}{\mat{R}^{-1}}
			&= \pdv{Q_1}{\mat{R}^{-1}} + \pdv{Q_2}{\mat{R}^{-1}} + \pdv{Q_3}{\mat{R}^{-1}} + \pdv{Q_4}{\mat{R}^{-1}} = \pdv{Q_1}{\mat{R}^{-1}} + \pdv{Q_4}{\mat{R}^{-1}} & \nonumber \\
		&&	&= \frac{NT}{2} \mat{R} - \frac{1}{2} \sum_{n = 1}^{N} \sum_{t = 1}^{T} \vec{y}_t^{(n)} \Big(\vec{y}_t^{(n)}\Big)^T - \hat{\vec{g}}_t^{(n)} \Big(\vec{y}_t^{(n)}\Big)^T - \vec{y}_t^{(n)} \Big(\hat{\vec{g}}_t^{(n)}\Big)^T + \Big(\mat{G}_t^{(n)}\Big)^T \overset{!}{=} \mat{0} & \nonumber \\
		\implies && \mat{R}^\new &= \frac{1}{NT} \sum_{n = 1}^{N} \sum_{t = 1}^{T} \vec{y}_t^{(n)} \Big(\vec{y}_t^{(n)}\Big)^T - \hat{\vec{g}}_t^{(n)} \Big(\vec{y}_t^{(n)}\Big)^T - \vec{y}_t^{(n)} \Big(\hat{\vec{g}}_t^{(n)}\Big)^T + \Big(\mat{G}_t^{(n)}\Big)^T & \label{eq:measurementNoiseCovariance}
	\end{align}
	
	The equations for the state dynamics matrix~\eqref{eq:stateDynamicsMatrix}, the state noise covariance~\eqref{eq:stateNoiseCovariance}, the initial state mean~\eqref{eq:initialStateMean} and the initial state covariance~\eqref{eq:initialStateCovariance} match exactly the equations given in~\cite{ghahramaniParameterEstimationLinear1996}. The equation for the measurement noise covariance~\eqref{eq:measurementNoiseCovariance} differs from the one given in~\cite{ghahramaniParameterEstimationLinear1996} as we are having nonlinear measurements and thus we cannot write the covariance in such a compact form.
	
	As said before, the expectations~\eqref{eq:expectedMeasurement} and~\eqref{eq:expectedMeasurementMat} are problematic as we cannot evaluate the integrals
	\begin{align*}
		\hat{\vec{g}}_t^{(n)} &= \int\! \vec{g}\Big(\vec{s}_t^{(n)}\Big) p\Big(\vec{s}_t^{(n)} \Biggiven \vec{y}_{1:T}\Big) \dd{\vec{s}_{1:T}^{(n)}} \\
		\mat{G}_t^{(n)}       &= \int\! \vec{g}\Big(\vec{s}_t^{(n)}\Big) \Big(\vec{g}\Big(\vec{s}_t^{(n)}\Big)\Big)^T p\Big(\vec{s}_t^{(n)} \Biggiven \vec{y}_{1:T} \Big) \dd{\vec{s}_{1:T}^{(n)}}
	\end{align*}
	in closed form (the function \( \vec{g}(\cdot) \) is nonlinear). Note that the posterior distribution is Gaussian
	\begin{equation*}
		p\Big(\vec{s}_t^{(n)} \Biggiven \vec{y}_{1:T}\Big) \sim \normal\Big(\hat{\vec{s}}_{t}^{(n)}, \mat{V}_{t}\Big)
	\end{equation*}
	where \( \hat{\vec{s}}_{t}^{(n)} \) and \( \mat{V}_{t} \) form the posterior mean and covariance, respectively. These are calculated in the E-step (formulas~\eqref{eq:posteriorMean} and~\eqref{eq:posteriorCov}). Due to this Gaussian behavior, we can approximate the integrals using the spherical-radial cubature rule~\cite{solinCubatureIntegrationMethods2010} given in equation~\eqref{eq:sphericalRadialGaussianCubatureRule} (with \( \vec{\xi}_i = \sqrt{k} [\vec{1}]_i \)):
	\begin{align}
		\int\! \vec{g}\Big(\vec{s}_t^{(n)}\Big) p\Big(\vec{s}_t^{(n)} \Biggiven \vec{y}_{1:T}\Big) \dd{\vec{s}_t^{(n)}}
			&\approx \frac{1}{2k} \sum_{i = 1}^{2k} \vec{g}\bigg(\!\sqrt{\mat{V}_t} \vec{\xi}_i + \hat{\vec{s}}_t^{(n)}\!\bigg)  \label{eq:lgsCubatureG} \\
		\int\! \vec{g}\Big(\vec{s}_t^{(n)}\Big) \Big(\vec{g}\Big(\vec{s}_t^{(n)}\Big)\Big)^T p\Big(\vec{s}_t^{(n)} \Biggiven \vec{y}_{1:T} \Big) \dd{\vec{s}_t^{(n)}}
			&\approx \frac{1}{2k} \sum_{i = 1}^{2k} \vec{g}\bigg(\!\sqrt{\mat{V}_t} \vec{\xi}_i + \hat{\vec{s}}_t^{(n)}\!\bigg) \bigg(\!\vec{g}\bigg(\!\sqrt{\mat{V}_t} \vec{\xi}_i + \hat{\vec{s}}_t^{(n)}\!\bigg)\!\bigg)^T  \label{eq:lgsCubatureGG}
	\end{align}
	These approximations can be inserted into the closed-form calculation of \(\mat{R}\) given in equation~\eqref{eq:measurementNoiseCovariance} and can be easily differentiated \ac{wrt} \(\vec{\theta}\), \ac{eg} by using a neural network for approximating \(\vec{g}(\cdot)\) and an autograd engine like PyTorch~\cite{paszkePyTorchImperativeStyle2019}. Then we can use a numerical optimizer (\ac{eg} Adam~\cite{kingmaAdamMethodStochastic2017} or Adagrad~\cite{duchiAdaptiveSubgradientMethods2011}) and take one optimization step in each invocation of the M-step. That way we do not maximize \(Q\) in every M-step, but increase \(Q\) such that the convergence properties will still hold~\cite{moonExpectationmaximizationAlgorithm1996}.
	
	We are now able to perform the M-step with the closed-form maximizations given in equations~\eqref{eq:stateDynamicsMatrix},~\eqref{eq:stateNoiseCovariance},~\eqref{eq:initialStateMean},~\eqref{eq:initialStateCovariance} and~\eqref{eq:measurementNoiseCovariance} and using a numerical approach for the measurement parameters \(\vec{\theta}\) by applying cubature approximations given in~\eqref{eq:lgsCubatureG} and~\eqref{eq:lgsCubatureGG}. For completeness, we summarize all of them here:
	\begin{align*}
		\mat{A}^\new &= \Bigg(\! \sum_{t = 2}^{T} \mat{P}_{t, t - 1} \!\Bigg) \Bigg(\! \sum_{t = 2}^{T} \mat{P}_{t - 1} \!\Bigg)^{-1} \\
		\mat{Q}^\new &= \frac{1}{T - 1} \Bigg(\! \sum_{t = 2}^{T} \mat{P}_t - \mat{A}^\new \sum_{t = 2}^{T} \mat{P}_{t, t - 1} \!\Bigg) \\
		\vec{m}_0^\new &= \hat{\vec{s}}_1 = \frac{1}{N} \sum_{n = 0}^{N} \hat{\vec{s}}_1^{(n)} \\
		\mat{V}_0^\new &= \mat{P}_1^T - \hat{\vec{s}}_1 \hat{\vec{s}}_1^T \\
		\mat{R}^\new &= \frac{1}{NT} \sum_{n = 1}^{N} \sum_{t = 1}^{T} \vec{y}_t^{(n)} \Big(\vec{y}_t^{(n)}\Big)^T - \hat{\vec{g}}_t^{(n)} \Big(\vec{y}_t^{(n)}\Big)^T - \vec{y}_t^{(n)} \Big(\hat{\vec{g}}_t^{(n)}\Big)^T + \Big(\mat{G}_t^{(n)}\Big)^T \\
		\hat{\vec{g}}_t^{(n)} &\approx \frac{1}{2k} \sum_{i = 1}^{2k} \vec{g}\bigg(\!\sqrt{\mat{V}_t} \vec{\xi}_i + \hat{\vec{s}}_t^{(n)}\!\bigg) \\
		\mat{G}_t^{(n)} &\approx \frac{1}{2k} \sum_{i = 1}^{2k} \vec{g}\bigg(\!\sqrt{\mat{V}_t} \vec{\xi}_i + \hat{\vec{s}}_t^{(n)}\!\bigg) \bigg(\!\vec{g}\bigg(\!\sqrt{\mat{V}_t} \vec{\xi}_i + \hat{\vec{s}}_t^{(n)}\!\bigg)\!\bigg)^T
	\end{align*}

	We proceed by deriving the E-step.
% end

\subsection{E-Step}
	To get the expectations \( \hat{\vec{s}}_t^{(n)} \), \( \mat{P}_{t} \) and \( \mat{P}_{t, t - 1} \), we need to calculate the distribution \( p\Big(\vec{s}_t^{(n)} \biggiven \vec{y}_{1:T}\Big) \) for each time step \(t\) (and, consequently, for each observation sequence \(n\)). This is exactly the posterior distribution calculated by a smoother. Thus we divide the E-step into two parts~\cite{minkaHiddenMarkovModels1999}:
	\begin{enumerate}
		\item Filtering: To calculate the posterior distribution \( p\Big(\vec{s}_t^{(n)} \biggiven \vec{y}_{1:t}\Big) \), we employ a Gaussian filter similar to the standard Kalman filter.
		\item Smoothing: To calculate the smoothed posterior distribution \( p\Big(\vec{s}_t^{(n)} \biggiven \vec{y}_{1:T}\Big) \), we employ a Gaussian \ac{rts} smoother.
	\end{enumerate}
	Note that we write \( \vec{s}_{t \subgiven t_0} \) and \( \vec{y}_{t \subgiven t_0} \) for the sequences \( \Big(\vec{s}_{t \subgiven t_0}^{(1)}, \vec{s}_{t \subgiven t_0}^{(2)}, \cdots, \vec{s}_{t \subgiven t_0}^{(N)}\Big) \) and \( \Big(\vec{y}_{t\subgiven t_0}^{(1)}, \vec{y}_{t \subgiven t_0}^{(2)}, \cdots, \vec{y}_{t \subgiven t_0}^{(N)}\Big) \), respectively.

	\subsubsection{Forward Pass}
		To derive the forward pass equations, we utilize the groundwork done in~\cite{deisenrothProbabilisticPerspectiveGaussian2011} which concludes that the filter is given as
		\begin{align*}
			\hat{\vec{s}}_{t \subgiven t}^{(n)} & = \hat{\vec{s}}_{t \subgiven t - 1}^{(n)} + \mat{K}_{t \subgiven t - 1} (\mat{S}_{t \subgiven t - 1})^{-1} \Big(\vec{y}_t^{(n)} - \hat{\vec{y}}_{t \subgiven t - 1}^{(n)}\Big) \\
			\mat{V}_{t \subgiven t}       & = \mat{V}_{t \subgiven t - 1} - \mat{K}_{t \subgiven t - 1} (\mat{S}_{t \subgiven t - 1})^{-1} \mat{K}_{t \subgiven t - 1}^T
		\end{align*}
		with the expectations and covariances
		\begin{align}
			\hat{\vec{s}}_{t \subgiven t - 1}^{(n)} & \coloneqq \E_{\vec{s}_t^{(n)}}\Big[\vec{s}_t^{(n)} \Biggiven \vec{y}_{1:t - 1}\Big]  \label{eq:filterPriorState}                                                                     \\
			\mat{V}_{t \subgiven t - 1}             & \coloneqq \frac{1}{N} \sum_{n = 1}^{N} \Cov_{\vec{s}_t^{(n)}}\!\Big[\vec{s}_t^{(n)} \Biggiven \vec{y}_{1:t - 1}\Big]  \label{eq:filterPriorCovariance}                               \\
			\hat{\vec{y}}_{t \subgiven t - 1}^{(n)} & \coloneqq \E_{\vec{y}_t^{(n)}}\Big[\vec{y}_t^{(n)} \Biggiven \vec{y}_{1:t - 1}\Big]  \label{eq:filterPriorMeas}                                                                      \\
			\mat{S}_{t \subgiven t - 1}             & \coloneqq \frac{1}{N} \sum_{n = 1}^{N} \Cov_{\vec{y}_t^{(n)}}\!\Big[\vec{y}_t^{(n)} \Biggiven \vec{y}_{1:t - 1}\Big]  \label{eq:filterPriorMeasCov}                                  \\
			\mat{K}_{t \subgiven t - 1}             & \coloneqq \frac{1}{N} \sum_{n = 1}^{N} \Cov_{\vec{s}_t^{(n)}, \vec{y}_t^{(n)}}\!\Big[\vec{s}_t^{(n)}, \vec{y}_t^{(n)} \Biggiven \vec{y}_{1:t - 1}\Big]  \label{eq:filterPriorKalman}
		\end{align}
		which are priors for the state, state covariance, measurement, measurement covariance and cross-covariance, respectively. As in our case only the measurements are nonlinear, we can easily evaluate the prior state~\eqref{eq:filterPriorState} and the prior covariance~\eqref{eq:filterPriorCovariance} by exploiting the linearity of the expectation operator:
		\begin{align*}
			\hat{\vec{s}}_{t \subgiven t - 1}^{(n)}
				&= \E_{\vec{s}_t^{(n)}}\Big[\vec{s}_t^{(n)} \Biggiven \vec{y}_{1:t - 1}\Big] \\
				&= \E_{\vec{s}_{t - 1}^{(n)}, \vec{w}_t^{(n)}}\Big[\mat{A} \vec{s}_{t - 1}^{(n)} + \vec{w}_t^{(n)} \Biggiven \vec{y}_{1:t - 1}\Big] \\
				&= \mat{A} \E_{\vec{s}_{t - 1}^{(n)}}\Big[\vec{s}_{t - 1}^{(n)} \Biggiven \vec{y}_{1:t - 1}\Big] \\
				&= \mat{A} \hat{\vec{s}}_{t - 1 \subgiven t - 1}^{(n)} \\
			\mat{V}_{t \subgiven t - 1}
				&= \frac{1}{N} \sum_{n = 1}^{N} \Cov_{\vec{s}_t^{(n)}}\!\Big[\vec{s}_t^{(n)} \Biggiven \vec{y}_{1:t - 1}\Big] \\
				&= \frac{1}{N} \sum_{n = 1}^{N} \Cov_{\vec{s}_{t - 1}^{(n)}}\!\Big[\mat{A} \vec{s}_{t - 1}^{(n)} \Biggiven \vec{y}_{1:t - 1}\Big] + \Cov_{\vec{w}_t^{(n)}}\!\Big[\vec{w}_t^{(n)} \Biggiven \vec{y}_{1:t - 1}\Big] \\
				&= \frac{1}{N} \sum_{n = 1}^{N} \E_{\vec{s}_t^{(n)}}\bigg[\vec{s}_t^{(n)} \Big(\vec{s}_t^{(n)}\Big)^T \bigggiven \vec{y}_{1:t - 1}\bigg] + \mat{Q} \\
				&= \frac{1}{N} \sum_{n = 1}^{N} \E_{\vec{s}_{t - 1}^{(n)}}\bigg[\mat{A} \vec{s}_{t - 1}^{(n)} \Big(\vec{s}_{t - 1}^{(n)}\Big)^T \mat{A}^T \bigggiven \vec{y}_{1:t - 1}\bigg] + \mat{Q} \\
				&= \mat{A} \frac{1}{N} \sum_{n = 1}^{N} \E_{\vec{s}_{t - 1}^{(n)}}\bigg[\vec{s}_{t - 1}^{(n)} \Big(\vec{s}_{t - 1}^{(n)}\Big)^T \bigggiven \vec{y}_{1:t - 1}\bigg] \mat{A}^T + \mat{Q} \\
				&= \mat{A} \mat{V}_{t - 1 \subgiven t - 1} \mat{A}^T + \mat{Q}
		\end{align*}
		
		But we are using a nonlinear measurement function \( \vec{g}(\cdot) \). Hence, it is not possible to compute the integrals produced by the expectations/covariances~\eqref{eq:filterPriorMeas},~\eqref{eq:filterPriorMeasCov} and~\eqref{eq:filterPriorKalman}. As in the derivation of the M-step, we apply cubature methods to approximate the integrals:
		\begin{align}
			\hat{\vec{y}}_{t \subgiven t - 1}^{(n)}
				&= \E_{\vec{y}_t^{(n)}}\Big[\vec{y}_t^{(n)} \Biggiven \vec{y}_{1:t - 1}\Big]  \nonumber \\
				&= \E_{\vec{s}_t^{(n)}}\Big[\vec{g}\Big(\vec{s}_t^{(n)}\Big) \Biggiven \vec{y}_{1:t - 1}\Big]  \nonumber \\
				&= \int\! \vec{g}\Big(\vec{s}_t^{(n)}\Big) p\Big(\vec{s}_t^{(n)} \Biggiven \vec{y}_{1:t - 1}\Big) \dd{\vec{s}_t^{(n)}}  \nonumber \\
				&= \int\! \vec{g}\Big(\vec{s}_t^{(n)}\Big) \,\normal\Big(\hat{\vec{s}}_{t \subgiven t - 1}^{(n)}, \mat{V}_{t \subgiven t - 1}\Big) \dd{\vec{s}_t^{(n)}}  \nonumber \\
				&\approx \frac{1}{2k} \sum_{i = 1}^{2k} \vec{g}\bigg(\! \sqrt{\mat{V}_{t \subgiven t - 1}} \vec{\xi}_i + \hat{\vec{s}}_{t \subgiven t - 1}^{(n)} \!\bigg)  \label{eq:cubatureY}
		\end{align}
		\begin{align}
			\mat{S}_{t \subgiven t - 1}
				&= \frac{1}{N} \sum_{n = 1}^{N} \Cov_{\vec{y}_t^{(n)}}\!\Big[\vec{y}_t^{(n)} \Biggiven \vec{y}_{1:t - 1}\Big]  \nonumber \\
				&= \frac{1}{N} \sum_{n = 1}^{N} \Cov_{\vec{s}_t^{(n)}}\!\Big[\vec{g}\Big(\vec{s}_t^{(n)}\Big) \Biggiven \vec{y}_{1:t - 1}\Big] + \Cov_{\vec{v}_t^{(n)}}\!\Big[\vec{v}_t^{(n)} \Biggiven \vec{y}_{1:t - 1}\Big]  \nonumber \\
				&= \frac{1}{N} \sum_{n = 1}^{N} \E_{\vec{s}_t^{(n)}}\bigg[\vec{g}\Big(\vec{s}_t^{(n)}\bigg) \Big(\vec{g}\Big(\vec{s}_t^{(n)}\Big)\Big)^T \bigggiven \vec{y}_{1:t - 1}\bigg] + \mat{R}  \nonumber \\
				&= \frac{1}{N} \sum_{n = 1}^{N} \int\! \vec{g}\Big(\vec{s}_t^{(n)}\Big) \Big(\vec{g}\Big(\vec{s}_t^{(n)}\Big)\Big)^T p\Big(\vec{s}_t^{(n)} \given \vec{y}_{1:t - 1}\Big) \dd{\vec{s}_t^{(n)}} + \mat{R}  \nonumber \\
				&= \frac{1}{N} \sum_{n = 1}^{N} \int\! \vec{g}\Big(\vec{s}_t^{(n)}\Big) \Big(\vec{g}\Big(\vec{s}_t^{(n)}\Big)\Big)^T \,\normal\Big(\hat{\vec{s}}_{t \subgiven t - 1}^{(n)}, \mat{V}_{t \subgiven t - 1}\Big) \dd{\vec{s}_t^{(n)}} + \mat{R}  \nonumber \\
				&\approx \frac{1}{2kN} \sum_{n = 1}^{N} \sum_{i = 1}^{2k} \vec{g}\bigg(\!\sqrt{\mat{V}_{t \subgiven t - 1}} \vec{\xi}_i + \hat{\vec{s}}_{t \subgiven t - 1}^{(n)}\!\bigg) \bigg(\!\vec{g}\bigg(\!\sqrt{\mat{V}_{t \subgiven t - 1}} \vec{\xi}_i + \hat{\vec{s}}_{t \subgiven t - 1}^{(n)}\!\bigg)\!\bigg)^T + \mat{R}  \label{eq:cubatureS}
		\end{align}
		\begin{align}
			\mat{K}_{t \subgiven t - 1}
				&= \frac{1}{N} \sum_{n = 1}^{N} \Cov_{\vec{s}_t^{(n)}, \vec{y}_t^{(n)}}\!\Big[\vec{s}_t^{(n)}, \vec{y}_t^{(n)} \Biggiven \vec{y}_{1:t - 1}\Big]  \nonumber \\
				&= \frac{1}{N} \sum_{n = 1}^{N} \E_{\vec{s}_t^{(n)}, \vec{y}_t^{(n)}}\bigg[\vec{s}_t^{(n)} \Big(\vec{y}_t^{(n)}\Big)^T \bigggiven \vec{y}_{1:t - 1}^{(n)}\bigg] - \frac{1}{N} \sum_{n = 1}^{N} \E_{\vec{s}_t^{(n)}}\Big[\vec{s}_t^{(n)} \Biggiven \vec{y}_{1:t - 1}\Big] \bigg(\E_{\vec{y}_t^{(n)}}\Big[\vec{y}_t^{(n)} \Biggiven \vec{y}_{1:t - 1}\Big]\bigg)^T  \nonumber \\
				&= \frac{1}{N} \sum_{n = 1}^{N} \E_{\vec{s}_t^{(n)}, \vec{y}_t^{(n)}}\bigg[\vec{s}_t^{(n)} \Big(\vec{y}_t^{(n)}\Big)^T \bigggiven \vec{y}_{1:t - 1}\bigg] - \frac{1}{N} \sum_{n = 1}^{N} \hat{\vec{s}}_{t \subgiven t - 1}^{(n)} \Big(\hat{\vec{y}}_{t \subgiven t - 1}^{(n)}\Big)^T  \nonumber \\
				&= \frac{1}{N} \sum_{n = 1}^{N} \E_{\vec{s}_t^{(n)}}\bigg[\vec{s}_t^{(n)} \Big(\vec{g}\Big(\vec{s}_t^{(n)}\Big)\Big)^T \bigggiven \vec{y}_{1:t - 1}\bigg] - \frac{1}{N} \sum_{n = 1}^{N} \hat{\vec{s}}_{t \subgiven t - 1}^{(n)} \Big(\hat{\vec{y}}_{t \subgiven t - 1}^{(n)}\Big)^T  \nonumber \\
				&= \frac{1}{N} \sum_{n = 1}^{N} \int\! \vec{s}_t^{(n)} \Big(\vec{g}\Big(\vec{s}_t^{(n)}\Big)\Big)^T p\Big(\vec{s}_t^{(n)} \Biggiven \vec{y}_{1:t - 1}\Big) \dd{\vec{s}_t^{(n)}} - \frac{1}{N} \sum_{n = 1}^{N} \hat{\vec{s}}_{t \subgiven t - 1}^{(n)} \Big(\hat{\vec{y}}_{t \subgiven t - 1}^{(n)}\Big)^T  \nonumber \\
				&= \frac{1}{N} \sum_{n = 1}^{N} \int\! \vec{s}_t \Big(\vec{g}\Big(\vec{s}_t^{(n)}\Big)\Big)^T \,\normal\Big(\hat{\vec{s}}_{t \subgiven t - 1}^{(n)}, \mat{V}_{t \subgiven t - 1}\Big) \dd{\vec{s}_t} - \frac{1}{N} \sum_{n = 1}^{N} \hat{\vec{s}}_{t \subgiven t - 1}^{(n)} \Big(\hat{\vec{y}}_{t \subgiven t - 1}^{(n)}\Big)^T  \nonumber \\
				&\approx \frac{1}{2kN} \sum_{n = 1}^{N}  \sum_{i = 1}^{2k} \bigg(\!\sqrt{\mat{V}_{t \subgiven t - 1}} \vec{\xi}_i + \hat{\vec{s}}_{t \subgiven t - 1}^{(n)}\!\bigg) \bigg(\!\vec{g}\bigg(\!\sqrt{\mat{V}_{t \subgiven t - 1}} \vec{\xi}_i + \hat{\vec{s}}_{t \subgiven t - 1}^{(n)}\!\bigg)\!\bigg)^T - \frac{1}{N} \sum_{n = 1}^{N} \hat{\vec{s}}_{t \subgiven t - 1}^{(n)} \Big(\hat{\vec{y}}_{t \subgiven t - 1}^{(n)}\Big)^T  \label{eq:cubatureK}
		\end{align}
		This completes the derivation of the forward pass equations.
		
		For completeness, we summarize all of them here:
		\begin{align*}
			\hat{\vec{s}}_{t \subgiven t - 1}^{(n)} &= \mat{A} \hat{\vec{s}}_{t - 1 \subgiven t - 1}^{(n)} \\
			\mat{V}_{t \subgiven t - 1} &= \mat{A} \mat{V}_{t - 1 \subgiven t - 1} \mat{A}^T + \mat{Q} \\
			\hat{\vec{y}}_{t \subgiven t - 1}^{(n)} &= \frac{1}{2k} \sum_{i = 1}^{2k} \vec{g}\bigg(\! \sqrt{\mat{V}_{t \subgiven t - 1}} \vec{\xi}_i + \hat{\vec{s}}_{t \subgiven t - 1}^{(n)} \!\bigg) \\
			\mat{S}_{t \subgiven t - 1} &= \frac{1}{2kN} \sum_{n = 1}^{N} \sum_{i = 1}^{2k} \vec{g}\bigg(\!\sqrt{\mat{V}_{t \subgiven t - 1}} \vec{\xi}_i + \hat{\vec{s}}_{t \subgiven t - 1}^{(n)}\!\bigg) \bigg(\!\vec{g}\bigg(\!\sqrt{\mat{V}_{t \subgiven t - 1}} \vec{\xi}_i + \hat{\vec{s}}_{t \subgiven t - 1}^{(n)}\!\bigg)\!\bigg)^T + \mat{R} \\
			\mat{K}_{t \subgiven t - 1} &= \frac{1}{2kN} \sum_{n = 1}^{N}  \sum_{i = 1}^{2k} \bigg(\!\sqrt{\mat{V}_{t \subgiven t - 1}} \vec{\xi}_i + \hat{\vec{s}}_{t \subgiven t - 1}^{(n)}\!\bigg) \bigg(\!\vec{g}\bigg(\!\sqrt{\mat{V}_{t \subgiven t - 1}} \vec{\xi}_i + \hat{\vec{s}}_{t \subgiven t - 1}^{(n)}\!\bigg)\!\bigg)^T - \frac{1}{N} \sum_{n = 1}^{N} \hat{\vec{s}}_{t \subgiven t - 1}^{(n)} \Big(\hat{\vec{y}}_{t \subgiven t - 1}^{(n)}\Big)^T \\
			\hat{\vec{s}}_{t \subgiven t}^{(n)} &= \hat{\vec{s}}_{t \subgiven t - 1}^{(n)} + \mat{K}_{t \subgiven t - 1} (\mat{S}_{t \subgiven t - 1})^{-1} \Big(\vec{y}_t^{(n)} - \hat{\vec{y}}_{t \subgiven t - 1}^{(n)}\Big) \\
			\mat{V}_{t \subgiven t} &= \mat{V}_{t \subgiven t - 1} - \mat{K}_{t \subgiven t - 1} (\mat{S}_{t \subgiven t - 1})^{-1} \mat{K}_{t \subgiven t - 1}^T
		\end{align*}
		The forward pass is initialized with \( \hat{\vec{s}}_{0 \subgiven 0}^{(n)} = \vec{m}_0 \) and \( \mat{V}_{0 \subgiven 0} = \mat{V}_0 \) for all \( n = 1, 2, \cdots, N \).
	% end
	
	\subsubsection{Backward Pass}
		To derive the backward pass equations, we utilize the groundwork done in~\cite{deisenrothProbabilisticPerspectiveGaussian2011} which concludes that the smoother is given as
		\begin{align*}
			\hat{\vec{s}}_{t - 1 \subgiven T}^{(n)} & = \hat{\vec{s}}_{t - 1 \subgiven t - 1}^{(n)} + \mat{J}_{t - 1} \Big(\hat{\vec{s}}_{t \subgiven T}^{(n)} - \hat{\vec{s}}_{t \subgiven t - 1}^{(n)}\Big) \\
			\mat{V}_{t - 1 \subgiven T}             & = \mat{V}_{t - 1 \subgiven t - 1} + \mat{J}_{t - 1} (\mat{V}_{t \subgiven T} - \mat{V}_{t \subgiven t - 1}) \mat{J}_{t - 1}^T
		\end{align*}
		with
		\begin{align}
			\mat{J}_{t - 1}                    &\coloneqq \mat{V}_{t - 1, t \subgiven t - 1} \mat{V}_{t \subgiven t - 1}^{-1}  \nonumber \\
			\mat{V}_{t - 1, t \subgiven t - 1} &\coloneqq \frac{1}{N} \sum_{n = 0}^{N} \Cov_{\vec{s}_{t - 1}^{(n)}, \vec{s}_t^{(n)}}\Big[\vec{s}_{t - 1}^{(n)}, \vec{s}_t^{(n)} \Biggiven \vec{y}_{1:t - 1}\Big]  \label{eq:smootherPriorCrossCov}
		\end{align}
		Like for the filter, we can easily evaluate the cross-covariance matrix \( \mat{V}_{t - 1, t \subgiven t - 1} \) by exploiting the linearity of the expectation:
		\begin{align*}
			\mat{V}_{t - 1, t \subgiven t - 1}
				&= \frac{1}{N} \sum_{n = 0}^{N} \Cov_{\vec{s}_{t - 1}^{(n)}, \vec{s}_t^{(n)}}\Big[\vec{s}_{t - 1}^{(n)}, \vec{s}_t^{(n)} \Biggiven \vec{y}_{1:t - 1}\Big] \\
				&= \frac{1}{N} \sum_{n = 0}^{N} \E_{\vec{s}_{t - 1}^{(n)}, \vec{s}_t^{(n)}}\bigg[\vec{s}_{t - 1}^{(n)} \Big(\vec{s}_t^{(n)}\Big)^T \bigggiven \vec{y}_{1:t - 1}\bigg] - \frac{1}{N} \sum_{n = 0}^{N} \E_{\vec{s}_{t - 1}^{(n)}}\Big[\vec{s}_{t - 1}^{(n)} \Biggiven \vec{y}_{1:t - 1}\Big] \bigg( \E_{\vec{s}_t^{(n)}}\Big[\vec{s}_t^{(n)} \Biggiven \vec{y}_{1:t - 1}\Big] \bigg)^T \\
				&\oversetfootnotemark{=} \frac{1}{N} \sum_{n = 0}^{N} \E_{\vec{s}_{t - 1}^{(n)}}\bigg[\vec{s}_{t - 1}^{(n)} \big(\vec{s}_{t - 1}^{(n)}\Big)^T \mat{A}^T \bigggiven \vec{y}_{1:t - 1}\bigg] - \frac{1}{N} \sum_{n = 0}^{N} \hat{\vec{s}}_{t - 1 \subgiven t - 1}^{(n)} \Big(\hat{\vec{s}}_{t \subgiven t - 1}^{(n)}\Big)^T \\
				&= \frac{1}{N} \sum_{n = 0}^{N} \E_{\vec{s}_{t - 1}^{(n)}}\bigg[\vec{s}_{t - 1}^{(n)} \Big(\vec{s}_{t - 1}^{(n)}\Big)^T \bigggiven \vec{y}_{1:t - 1}\Big] \mat{A}^T - \frac{1}{N} \sum_{n = 0}^{N} \hat{\vec{s}}_{t - 1 \subgiven t - 1}^{(n)} \Big(\hat{\vec{s}}_{t \subgiven t - 1}^{(n)}\Big)^T \\
				&= \mat{V}_{t - 1 \subgiven t - 1} \mat{A}^T - \frac{1}{N} \sum_{n = 0}^{N} \hat{\vec{s}}_{t - 1 \subgiven t - 1}^{(n)} \Big(\hat{\vec{s}}_{t \subgiven t - 1}^{(n)}\Big)^T
		\end{align*}
		\footnotetext{
			Due to the independence and zero mean of the noise, it disappears in the expectation:
			\( \E_{\vec{s}_{t - 1}^{(n)}, \vec{s}_t^{(n)}}\Big[\vec{s}_{t - 1}^{(n)} \big(\vec{s}_t^{(n)}\big)^T \Biggiven \vec{y}_{1:t - 1}\Big]
					= \E_{\vec{s}_{t - 1}^{(n)}, \vec{w}_t^{(n)}}\Big[\vec{s}_{t - 1}^{(n)} \Big(\!\big(\vec{s}_{t - 1}^{(n)}\big)^T \mat{A}^T + \big(\vec{w}_t^{(n)}\big)^T\Big) \Biggiven \vec{y}_{1:t - 1}\Big]
					= \E_{\vec{s}_{t - 1}^{(n)}, \vec{w}_t^{(n)}}\Big[\vec{s}_{t - 1}^{(n)} \big(\vec{s}_{t - 1}^{(n)}\big)^T \mat{A}^T + \vec{s}_{t - 1}^{(n)} \big(\vec{w}_t^{(n)}\big)^T \Biggiven \vec{y}_{1:t - 1}\Big]
					= \E_{\vec{s}_{t - 1}^{(n)}, \vec{w}_t^{(n)}}\Big[\vec{s}_{t - 1}^{(n)} \big(\vec{s}_{t - 1}^{(n)}\big)^T \mat{A}^T \Biggiven \vec{y}_{1:t - 1}\Big] \)
		}
		This completes the derivation of the backward pass equations.
		
		For completeness, we summarize all of them here:
		\begin{align}
			\mat{V}_{t - 1, t \subgiven t - 1} &= \mat{V}_{t - 1 \subgiven t - 1} \mat{A}^T - \frac{1}{N} \sum_{n = 0}^{N} \hat{\vec{s}}_{t - 1 \subgiven t - 1}^{(n)} \Big(\hat{\vec{s}}_{t \subgiven t - 1}^{(n)}\Big)^T  \nonumber \\
			\mat{J}_{t - 1} &= \mat{V}_{t - 1, t \subgiven t - 1} \mat{V}_{t \subgiven t - 1}^{-1}  \nonumber \\
			\hat{\vec{s}}_{t - 1 \subgiven T}^{(n)} &= \hat{\vec{s}}_{t - 1 \subgiven t - 1}^{(n)} + \mat{J}_{t - 1} \Big(\hat{\vec{s}}_{t \subgiven T}^{(n)} - \hat{\vec{s}}_{t \subgiven t - 1}^{(n)}\Big)  \label{eq:posteriorMean} \\
			\mat{V}_{t - 1 \subgiven T} &= \mat{V}_{t - 1 \subgiven t - 1} + \mat{J}_{t - 1} (\mat{V}_{t \subgiven T} - \mat{V}_{t \subgiven t - 1}) \mat{J}_{t - 1}^T  \label{eq:posteriorCov}
		\end{align}
		According to~\cite{minkaBayesianLinearRegression1999}, the self- and cross-correlations are given as
		\begin{align*}
			\mat{P}_t &= \mat{V}_{t \subgiven T} + \frac{1}{N} \sum_{n = 1}^{N} \hat{\vec{s}}_{t \subgiven T}^{(n)} \Big(\hat{\vec{s}}_{t \subgiven T}^{(n)}\Big)^T \\
			\mat{P}_{t, t - 1} &= \mat{J}_{t - 1} \mat{V}_{t \subgiven T} + \frac{1}{N} \sum_{n = 1}^{N} \hat{\vec{s}}_{t \subgiven T}^{(n)} \Big(\hat{\vec{s}}_{t - 1 \subgiven T}^{(n)}\Big)^T
		\end{align*}
		
		That wraps up the derivation of the E- and M-step of the EM-algorithm for partially linear dynamical Gaussian systems.
	% end
	
	\subsection{Exactness of the Cubature Rule for Linear Systems}
		If we restrict the previous generalization of the nonlinear measurement function \( \vec{g}_{\vec{\theta}} : \R^k \to \R^p \) to a linear function \( \vec{g}(\vec{s}) = \mat{C} \vec{s} \) where \( \vec{\theta} = \{ \mat{C} \} \), we can show that the approximations for \( \hat{\vec{g}}_t^{(n)} \), \( \mat{G}_t^{(n)} \), \( \hat{\vec{y}}_{t \subgiven t - 1}^{(n)} \), \( \mat{S}_{t \subgiven t - 1} \) and \( \mat{K}_{t \subgiven t - 1} \), given in equations~\eqref{eq:lgsCubatureG},~\eqref{eq:lgsCubatureGG},~\eqref{eq:cubatureY},~\eqref{eq:cubatureS} and~\eqref{eq:cubatureK} are exact, \ac{ie} they match the real expectation/covariance value.
		
		\begin{lemma}
			For any matrix \( \mat{A} \in \R^{n \times n} \), there exists exactly one skew-symmetric matrix \( \mat{B} \in \R^{n \times n} \) such that \( \mat{A}^T = \mat{A} + \mat{B} \) holds.
		\end{lemma}
		\begin{proof}
			Let \( \mat{A} \in \R^{n \times n} \) be any matrix. It then holds that
			\begin{align*}
				\mat{A} = \mat{A} + \mat{A}^T - \mat{A}^T \quad\iff\quad \mat{A}^T = \mat{A} - \mat{A} + \mat{A}^T = \mat{A} + \underbrace{(\mat{A}^T - \mat{A})}_{B \coloneqq} = \mat{A} + \mat{B}
			\end{align*}
			where \( \mat{B} \) is skew-symmetric:
			\begin{align*}
				\mat{B}^T = (\mat{A}^T - \mat{A})^T = \mat{A} - \mat{A}^T = -(\mat{A}^T - \mat{A}) = -\mat{B}
			\end{align*}
			As \(\mat{B}\) is fully determined by \( \mat{B} = \mat{A}^T - \mat{A} \), it is also unique.
		\end{proof}
		
		\begin{lemma}
			Given a symmetric and positive-semidefinite matrix \( \mat{A} \in \R^{n \times n} \), its corresponding principal square root \( \sqrt{\mat{A}} \) which is also positive-semidefinite with \( \mat{A} = \sqrt{\mat{A}} \sqrt{\mat{A}} \) is also symmetric.
		\end{lemma}
		\begin{proof}
			Let \( \mat{A} \in \R^{n \times n} \) be any symmetric and positive-semidefinite matrix and let \( \mat{B} \) be a skew-symmetric matrix such that \( \sqrt{\mat{A}}^T = \sqrt{\mat{A}} + \mat{B} \). If \( \sqrt{\mat{A}} \) is symmetric, \( \mat{B} = \mat{0} \) holds. Hence, we can proceed by massaging the matrices:
			\begin{align}
				     && \sqrt{\mat{A}}^T &= \sqrt{\mat{A}} + \mat{B} &  \nonumber \\
				\iff && \sqrt{\mat{A}}   &= \sqrt{\mat{A}}^T - \mat{B} &  \label{eq:sqrtProof1} \\
				\iff && \sqrt{\mat{A}^T} &= \sqrt{\mat{A}}^T - \mat{B} &  \nonumber \\
				\iff && \sqrt{\mat{A}}^T &= \sqrt{\mat{A}}^T - \mat{B} &  \label{eq:sqrtProof2}
			\end{align}
			By adding~\eqref{eq:sqrtProof1} and~\eqref{eq:sqrtProof2}
			\begin{align*}
				\sqrt{\mat{A}} + \sqrt{\mat{A}}^T = 2 \sqrt{\mat{A}}^T - 2 \mat{B} \quad\iff\quad \sqrt{\mat{A}} = \sqrt{\mat{A}}^T - 2 \mat{B}
			\end{align*}
			we see that \( \sqrt{\mat{A}} = \sqrt{\mat{A}}^T - 2 \mat{B} \) must hold. But according to~\eqref{eq:sqrtProof1}, \( \sqrt{\mat{A}} = \sqrt{\mat{A}}^T - \mat{B} \) must also hold. Hence, \( \mat{B} \) must zero out: \( \mat{B} = \mat{0} \). Therefore, \( \sqrt{\mat{A}} \) must be symmetric.
			
			\qedhere
		\end{proof}
	% end
% end













