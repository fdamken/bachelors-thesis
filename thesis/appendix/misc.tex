\chapter{Miscellaneous}
	\section{Solution of the Harmonic Oscillator}
		\label{app:harmonicOscillatorSolution}

		To solve the differential motion equation
		\begin{align*}
			m\ddot{x} = -kx
		\end{align*}
		of the harmonic oscillator given in~\autoref{subsec:harmonicOscillator}, we use the solution approach
		\begin{align*}
			x(t) = c e^{\lambda t} \qquad \dot{x}(t) = \lambda c e^{\lambda t} \qquad \ddot{x}(t) = \lambda^2 c e^{\lambda t}
		\end{align*}
		and insert it into the differential equation:
		\begin{align*}
			m\ddot{x} = -kx \quad\implies\quad
			m \lambda^2 c e^{\lambda t} = -k c e^{\lambda t} \quad\iff\quad
			m \lambda^2 = -k \quad\iff\quad
			\lambda = \pm \sqrt{-\frac{k}{m}}
		\end{align*}
		As both \(k\) and \(m\) are defined to be positive, we get the complex solutions:
		\begin{align*}
			x_1(t) = e^{i t \sqrt{k / m}} \qquad x_2(t) = e^{-i t \sqrt{k / m}}
		\end{align*}
		Due to the superposition principle, also \( x_1 + x_2 \) and \( x_1 - x_2 \) are solutions. Hence, we get two real solutions by using Euler's identity \( e^{\varphi i} = \cos(\varphi) + i \sin(\varphi) \):
		\begin{align*}
			x_1 + x_2
				&= e^{i t \sqrt{k / m}} + e^{-i t \sqrt{k / m}} \\
				&= \cos\Big(t \sqrt{k / m}\Big) + i \sin\Big(t \sqrt{k / m}\Big) + \cos\Big(t \sqrt{k / m}\Big) - i \sin\Big(t \sqrt{k / m}\Big) \\
				&= 2 \cos\Big(t \sqrt{k / m}\Big) \\
			x_1 - x_2
				&= e^{i \sqrt{k / m} t} - e^{-i t \sqrt{k / m}} \\
				&= \cos\Big(t \sqrt{k / m}\Big) i \sin\Big(t \sqrt{k / m}\Big) - \cos\Big(t \sqrt{k / m}\Big) + i \sin\Big(t \sqrt{k / m}\Big) \\
				&=  2i \sin\Big(t \sqrt{k / m}\Big)
		\end{align*}
		This yields the following general solution:
		\begin{align*}
			x(t) = c_1 \cos\Big(t \sqrt{k / m}\Big) + c_2 \sin\Big(t \sqrt{k / m}\Big),\quad c_1, c_2 \in \C
		\end{align*}
		As both Sine and Cosine are Sinusoidal, different \( c_1 \neq c_2 \) only lead to a phase shift. Thus we can also write the solution as
		\begin{align*}
			x(t) = A \cos\Big(t \sqrt{k / m} + \varphi\Big)
		\end{align*}
		with the amplitude \(A\) and the phase \(\varphi\).
	% end

	\section{Overview over Notations for Linear Gaussian Dynamical Systems}
		\begin{table}[ht]
			\centering
			\begin{tabular}{l|ccc}
				\textbf{Element / Definition} & \textbf{Ghahramani}~\cite{ghahramaniParameterEstimationLinear1996} & \textbf{Minka}~\cite{minkaHiddenMarkovModels1999} & \textbf{This Thesis} \\ \hline
				Time                         & \( t \)                         & \( t \)                 & \( t \)                                                    \\
				Last Timestep                & \( T \)                         & \( T \)                 & \( T \)                                                    \\
				State                        & \( \vec{x}_t \)                 & \( \vec{s}_t \)         & \( \vec{s}_t \)                                            \\
				Observations                 & \( \vec{y}_t \)                 & \( \vec{x}_t \)         & \( \vec{y}_t \)                                            \\
				State Dynamics Matrix        & \( \mat{A} \)                   & \( \mat{A} \)           & \( \mat{A} \)                                              \\
				State Noise Covariance       & \( \mat{Q} \)                   & \( \mat{\Gamma} \)      & \( \mat{Q} \)                                              \\
				Observation Matrix           & \( \mat{C} \)                   & \( \mat{C} \)           & \( \mat{C} \)                                              \\
				Observation Noise Covariance & \( \mat{R} \)                   & \( \mat{\Sigma} \)      & \( \mat{R} \)                                              \\
				Full Sequence \( \big( \vec{x}_1, \vec{x}_2, \cdots, \vec{x}_T \big) \)
				                             & \( \{ \vec{x} \} \)             & \( \vec{x}_{1:T} \)     & \( \vec{x}_{1:T} \)                                        \\
				Subsequence \( \big( \vec{x}_{t_0}, \vec{x}_{t_0 + 1}, \cdots, \vec{x}_{t_1} \big) \)
				                             & \( \{ \vec{x} \}_{t_0}^{t_1} \) & \( \vec{x}_{t_0:t_1} \) & \( \vec{x}_{t_0:t_1} \)                                    \\
				Expected Log Likelihood      & \( Q \)                         & \( Q \)                 & \( Q \)                                                    \\
				Predicted State              & \( \vec{x}_t^{t - 1} \)         & --                      & \( \hat{\vec{s}}_{t \subgiven t - 1} \)                    \\
				Predicted Covariance         & \( \mat{V}_t^{t - 1} \)         & \( \mat{P}_{t - 1} \)   & \( \hat{\mat{V}}_{t \subgiven t - 1} \)                    \\
				Filtered State               & \( \vec{x}_t^t \)               & \( \vec{m}_t \)         & \( \hat{\vec{s}}_{t \subgiven t} \)                        \\
				Filtered Covariance          & \( \mat{V}_t^t \)               & \( \mat{V}_t \)         & \( \hat{\mat{V}}_{t \subgiven t} \)                        \\
				Smoothed State               & \( \hat{\vec{x}}_t \)           & \( \hat{\vec{m}}_t \)   & \( \hat{\vec{s}}_t \), \( \hat{\vec{s}}_{t \subgiven T} \) \\
				Smoothed Covariance          & \( \mat{V}_t^T \)               & \( \hat{\mat{V}}_t \)   & \( \hat{\mat{V}}_t \), \( \hat{\mat{V}}_{t \subgiven T} \) \\
				Self-Correlation \( \E\big[\vec{x}_t \vec{x}_t^T \biggiven \{ \vec{y} \}\big] \)
				                             & \( \mat{P}_t \)                 & --                      & \( \mat{P}_t \)                                            \\
				Cross-Correlation \( \E\big[\vec{x}_t \vec{x}_{t-1}^T \biggiven \{ \vec{y} \}\big] \)
				                             & \( \mat{P}_{t, t - 1} \)        & --                      & \( \mat{P}_{t, t - 1} \)
			\end{tabular}
			\caption[Notations used for linear Gaussian dynamical systems in different papers]{Notations used for linear Gaussian dynamical systems in different papers.}
		\end{table}
	% end

	\section{Framework-Specific Implementation Problems}
		\label{app:implFrameworkProblems}

		\subsubsection{Slow QR Decomposition on GPU}
			During our implementation we found that QR decomposition is relatively slow on the \ac{gpu} due to a bug in PyTorch\footnote{See \href{https://web.archive.org/web/20201110121407/https://github.com/pytorch/pytorch/issues/22573}{\texttt{https://github.com/pytorch/pytorch/issues/22573}}.} at the time of writing. While faster QR decomposition methods on a \ac{gpu} have been proposed~\cite{andersonCommunicationAvoidingQRDecomposition2011a}, we stuck to copying the data back and forth between \ac{cpu} and \ac{gpu} in every \ac{em} iteration.

			We also address this in~\nameref{sec:futureWork}.
		% end

		\subsubsection{Issues with Numerical Integration}
			\label{subsubsec:integrationProblems}

			We should note that some environments we ran experiments on are described by stiff differential equations, \eg the pendulum or the cartpole environment. As stiff \acp{ode} can not be integrated well with explicit solution methods, implicit methods should be preferred throughout the data generation. For some environments in OpenAI's Gym~\cite{brockmanOpenAIGym2016}, this has to be explicitly configured, \eg for the cartpole environment with \lstinline|env.kinematics_integrator = 'implicit-euler'|. Not changing the integrator leads to catastrophic data. In case of the cartpole, using the explicit Euler method causes the pole to build up energy and to get significantly faster over time.
		% end
	% end
% end
