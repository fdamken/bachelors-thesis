% To add the outline to every section, add \AtBeginSection{\sectionslide} to the preamble.
\newcommand{\sectionslide}{
	\begin{frame}{\insertsectionhead \\ {\small Outline}}
		\tableofcontents[currentsection]
	\end{frame}
}

% To add the outline to every section, add \AtBeginSubsection{\subsectionslide} to the preamble.
\newcommand{\subsectionslide}{
	\begin{frame}{\insertsubsectionhead \\ {\small Outline}}
		\tableofcontents[currentsection, currentsubsection]
	\end{frame}
}


% Style definitions.

\MakeOuterQuote{"}

% TODO: Make alertblock to match TUDa design.

% Do not print "Figure:" in beamer legends.
\captionsetup{labelformat=empty}

% Matrix/Vector notation.
\renewcommand{\vec}[1]{\boldsymbol{\mathrm{#1}}}
\newcommand{\mat}[1]{\boldsymbol{\mathrm{#1}}}
% Shorthands.
\renewcommand{\C}{\mathbb{C}}
\newcommand{\R}{\mathbb{R}}
\newcommand{\E}{\mathbb{E}}
\newcommand{\normal}{\mathcal{N}}
\newcommand{\gaussianMulti}[4]{\frac{1}{\left(2\pi\right)^{#4/2} \cdot \lvert #3 \rvert^{1/2}} \exp \bigg\{\! -\frac{1}{2} \left(#1 - #2\right)^T #3^{-1} \left(#1 - #2\right) \bigg\}}
\newcommand{\logGaussianMulti}[4]{-\frac{1}{2} \log \lvert #3 \rvert - \frac{#4}{2} \log\left(2\pi\right) - \frac{1}{2} \left(#1 - #2\right)^T #3^{-1} \left(#1 - #2\right)}
\newcommand{\subgiven}{\vert}
\newcommand{\given}{\,\vert\,}
\newcommand{\biggiven}{\,\big\vert\,}
\newcommand{\Biggiven}{\,\Big\vert\,}
\newcommand{\bigggiven}{\,\bigg\vert\,}
\newcommand{\Bigggiven}{\,\Bigg\vert\,}
\newcommand{\new}{\mathrm{new}}
\newcommand{\KL}[2]{D_\mathrm{KL}\big( #1 \,\Vert\, #2 \big)}
\newcommand{\SRC}{\mathit{SRC}}
\newcommand{\rangedots}{\,\cdots\!}
\newcommand{\train}{\mathrm{train}}
% Math operators.
\DeclareMathOperator{\const}{const}
\DeclareMathOperator{\Cov}{Cov}
\DeclareMathOperator{\diag}{diag}

\newcommand{\oversetfootnotemark}[1]{\stepcounter{footnote} \overset{\mathclap{(\thefootnote)}}{#1}}
\let\realfootnote\footnote
\let\realfootnotetext\footnotetext
\renewcommand{\footnote}[1]{\realfootnote{\, #1}}
\renewcommand{\footnotetext}[1]{\realfootnotetext{\, #1}}
\newcommand{\doublefootnotetext}[2]{\addtocounter{footnote}{-1} \footnotetext{#1} \stepcounter{footnote} \footnotetext{#2}}
\newcommand{\triplefootnotetext}[3]{\addtocounter{footnote}{-2} \footnotetext{#1} \stepcounter{footnote} \footnotetext{#2} \stepcounter{footnote} \footnotetext{#3}}

\parindent0pt

% TikZ-related stuff.
\tikzset{> = { Latex[length = 2mm] }}

% algorithm2e.
\SetCommentSty{text}

% Ignore well-known acronyms in write-first-long rule.
\newcommand{\eg}{e.g.\xspace}
\newcommand{\ie}{i.e.\xspace}
\newcommand{\wrt}{w.r.t.\xspace}
